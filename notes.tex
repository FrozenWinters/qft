\documentclass{report}
\usepackage{amsmath}
\usepackage{amssymb}
\usepackage{amsthm}
\usepackage{amscd}
\usepackage{fullpage}
\usepackage{braket}
\usepackage{dsfont}
\usepackage{mdframed}

\theoremstyle{plain}
\newtheorem{theorem}{Theorem}[section]
\newtheorem{lemma}[theorem]{Lemma}
\newtheorem{proposition}[theorem]{Proposition}
\newtheorem{corollary}[theorem]{Corollary}

\theoremstyle{definition}
\newtheorem{definition}[theorem]{Definition}
\newtheorem{example}[theorem]{Example}

\theoremstyle{remark}
\newtheorem*{remark}{Remark}
\newtheorem*{note}{Note}

\newcommand{\FR}[2]{\frac{#1}{#2}}
\newcommand{\PFR}[2]{\left(\frac{#1}{#2}\right)}
\newcommand{\SFR}[2]{\sqrt{\frac{#1}{#2}}}

\newcommand{\mc}{\mathcal}
\newcommand{\lam}{\lambda}
\newcommand{\vphi}{\varphi}
\newcommand{\om}{\omega}
\newcommand{\Om}{\Omega}
\newcommand{\gam}{\gamma}
\newcommand{\di}{\partial}
\newcommand{\ddi}[2]{\FR{\partial {#1}}{\partial {#2}}}
\newcommand{\hp}{\hat p}
\newcommand{\ha} {\hat a}
\newcommand{\had}{\hat a^\dagger}

\newcommand{\elaborate}{{\color{blue} \textbf{ELABORATE}}}
\newcommand{\CHECK}{{\color{blue} \textbf{CHECK}}}

\DeclareMathOperator{\bC}{\mathbb{C}}
\DeclareMathOperator{\bR}{\mathbb{R}}
\DeclareMathOperator{\bZ}{\mathbb{Z}}
\DeclareMathOperator{\cA}{\mathcal{A}}
\DeclareMathOperator{\cB}{\mathcal{B}}
\DeclareMathOperator{\cC}{\mathcal{C}}
\DeclareMathOperator{\cD}{\mathcal{D}}
\DeclareMathOperator{\cF}{\mathcal{F}}
\DeclareMathOperator{\cI}{\mathcal{I}}
\DeclareMathOperator{\cJ}{\mathcal{J}}
\DeclareMathOperator{\cL}{\mathcal{L}}
\DeclareMathOperator{\fg}{\mathfrak{g}}
\DeclareMathOperator{\id}{id}
\DeclareMathOperator{\im}{im}
\DeclareMathOperator{\Ad}{Ad}
\DeclareMathOperator{\Sp}{Sp}
\DeclareMathOperator{\Hom}{Hom}
\DeclareMathOperator{\Aut}{Aut}

\title{Quantum Field Theory\\Fall 2015 Seminar Notes}
\author{Anton Borissov, Henry Liu}
\date{\today}

\begin{document}

\maketitle
\chapter{Klein-Gordon Field}
\section{Why Fields?}
    
    Volume 1 of Steven Weinberg's \emph{Quantum Theory of Fields} is
    devoted to answering this question. A discussion of scattering
    experiments lead him to the $S$-matrix, and then to the local behaviour
    of experiments (which he calls the cluster decomposition principle),
    and then using Lorentz invariance, fields just practically fall out.
    Weinberg does a really good job of convincing us that QFT in some form
    or another really must exist if we assume Lorentz invariance and
    unitarity.
%   First, he analyzes scattering
%   experiments and introduces us to the arena of the multiparticle Hilbert
%   space and the main player, the $S$-matrix. Next, using the cluster
%   decomposition principle, he justifies why a Hamiltonian must be written
%   as a sum of creation and annihilation operators. This cluster
%   decomposition principle makes precise what we mean by ``experiments at
%   large distances between one another are uncorrelated.'' Moreover, ``this
%   cluster decomposition principle plays a crucial part in making field
%   theory inevitable.'' (Weinberg Vol 1) This approach is very appealing
%   for it justifies why fields are important without citing the
%   ``problems'' of previous theories.
%
%   Miscelleanous remarks from the wisdom bank of Weinberg:
%   \begin{itemize}
%       \item {\color{red}``The structure and properties of any quantum field are
%           dictated by the representations of the homogenous Lorentz group
%       under which it transforms.''}
%       \item Free fields $\leftrightarrow$ trivial
%           representation, causal vector $\leftrightarrow$ 4-vector
%           representation, Dirac fields $\leftrightarrow$ dirac
%           representation, etc.
%   \end{itemize}

    Peskin and Schroeder give a slightly different motivation, one that is
    closer to the historical reason of why fields were introduced:
    \begin{itemize}
        \item Single particle relativistic wave functions $\implies$
            inconsistencies in theory (negative energy eigenstates)
            \elaborate
        \item $E=mc^2$ allows for particles to be created at high energies
        \item $\Delta E\cdot \Delta t = \hbar$ allows for virtual
            particles
        \item Causality violation. Set $H = \FR{\hp^2}{2m}$
            \begin{align*}
                U(t) &= \braket{\vec x|e^{-iHt}|\vec x_0}\\
                     &= \int \FR{d^3p}{(2\pi)^3} \braket{\vec
                         x|e^{-i(p^2/2m)t}|p}\braket{p|x}\\
 &= \int \FR{d^3p}{(2\pi)^3} e^{-i(p^2/2m)t}e^{i\vec p\cdot (\vec x-\vec x_0)}\\
                     &= \PFR{m}{2\pi i t}^{3/2} e^{im(\vec x-\vec x_0)^2/2t}
            \end{align*}
            This last quantity is non-zero, even for arbitariry $x$ that
            may by space-like separated.
    \end{itemize}
    QFT seems to solve all of these mysteries. One very good feature of the
    theory is that it predicts a lot of experiments to very high accuracy.
    QED is something we will see very soon that has been very well tested
    and agrees very well with experiments.
    
\section{Elements of Classical Field Theory}
\subsection{Lagrangian Field Theory}
    \begin{itemize}
        \item Fundamental quantity in Lagrangian field theory is the action $S$.
In high school, the Lagrangian is a function of time,
positions, and velocities of a system: $L(t,x(t),\dot x(t))$. The action
is given by $S = \int dt\, L$.
Fields can also be described in a Lagrangian formalism, for instance by
considering every point in space-time as a ``particle'' that wiggles back
and forth with the amplitude of wiggling characterizing the strength of the
field.

Let $\vphi : M \to \bR$, define a Lagrangian \emph{density} $\mc
L(t,\vphi,\di_\mu \vphi)$, the honest Lagrangian $L = \int d^3x \mc L$, and
finally define the action: \[ S = \int dt\, L = \int d^4x \mc L \]

\begin{mdframed}
    Four-vector notation:
    \begin{itemize}
        \item Greek letters $\mu,\nu,\ldots \in \{0,1,2,3\}$
        \item Roman letters $i,g,\ldots \in \{1,2,3\}$.
        \item $x^\mu = (x^0,x^1,x^2,x^3)$
        \item Signature $(+---)$
        \item $\eta_{\mu\nu} = \mathrm{diag}(1,-1,-1,-1)$
        \item $\di_\mu f = \FR{\di f}{\di x^\mu} = (\di_0 f,\di_1 f,\di_2
            f,\di_3 f)$.
    \end{itemize}
\end{mdframed}

\item Extremize the action. Let $\delta f = f(\vphi+\xi) - f(\vphi)$.
    \begin{align*}
        0 = \delta S &= \int d^4 x \left( \ddi{\mc L}{\vphi}\delta\vphi
        +\ddi{\mc L}{(\di_\mu \vphi)}
        \underbrace{\delta(\di_\mu}_{\text{commute}}\vphi)\right)\\
        &= \int d^4 x \left[ \ddi{\mc L}{\vphi}\delta\vphi
        + \di_\mu \left(  \ddi{\mc L}{\vphi}\delta\vphi \right)
    - \di_\mu \left(\ddi{\mc L}{(\di_\mu\vphi)}\right)\delta\vphi \right]
    \end{align*}
    By Stokes' theorem, we can break this integral up into two parts, one
    of which is called the boundary term. Taking a variation that is fixed
    along the boundary means $\delta\vphi \equiv 0$ on the boundary which
    means that the boundary term does not contribute to $\delta S$.
    Moreover, if we take $\delta S = 0$ for every variation, then we obtain
    the Euler Lagrange equations:
    \[ \di_\mu \left( \ddi{\mc L}{(\di_\mu \vphi)} \right) - \ddi{\mc
    L}{\vphi} = 0 \]

    \begin{remark}
        The Lagragian formalism is useful for relativistic dynamics because
        all expressions are chosen to Lorentz invariant.
    \end{remark}
    \end{itemize}
\subsection{Hamiltonian Field Theory}
    \begin{itemize}
\item Introducing this makes the transition to the quantum theory easier.
\item High school Hamiltonian formalism: 
    $p = \ddi{L}{\dot q},H=\sum p\dot q-L$.
\item Pretend that $\vec x$ enumerates points on the lattice of space-time:
    \begin{align*}
        p(\vec x) = \ddi{\mc L}{\dot\vphi(\vec x)}
        &= \ddi{}{\dot\vphi(\vec x)}\int d^3y\, \mc
        L(\vphi(y),\dot\vphi(y))\\
        &\sim \ddi{}{\dot\vphi(\vec x)}\sum \mc
        L(\vphi(y),\dot\vphi(y)) d^3y\\
        &= \ddi{\mc L}{\dot\vphi(\vec x)} d^3x\\
        &\equiv \pi(\vec x)d^3 x
    \end{align*}
    since each point on the lattice represents a different variable, so the
    derivative just picks out the one at $\vec x$. We call $\pi(\vec x)$
    the momentum \emph{density}. Therefore the Hamiltonian looks like:
    \[H = \int d^3x\, \left[\pi(\vec x)\dot\vphi(\vec x) - \mc L\right].\]
    (See the stress-energy tensor part for another derivation of the
    Hamiltonian which falls out of Noether's theorem for being the
    conserved quantity under time translations.)
\item \textbf{Important example:} Take $\mc L = \FR{1}{2}(\di_\mu\vphi)^2 -\FR{1}{2}
    m^2\vphi^2$. Euler-Lagrange equations become $\di^\mu(\di_\mu
    \vphi)+m^2\vphi=0$ which is the Klein Gordon equation. The Hamiltonian
    becomes:
    \[ H = \int d^3x \mc H
= \int d^3x \left[ \underbrace{\FR{\pi^2}{2}}_{\text{moving in time}}
    + \underbrace{\FR{(\nabla \vphi)^2}{2}}_{\text{shearing in space}}
+ \underbrace{\FR{m^2\vphi^2}{2}}_{\text{existing at all}}\right]\]
    \end{itemize}
\subsection{Noether's Theorem - How to Compute Conserved Quantities}
To every continuous transformation of the field we can assign an infinitesmal
transformation:
\[ \vphi(x) \rightarrow \vphi'(x) = \vphi(x) +
\alpha\underbrace{\Delta\vphi(x)}_{\text{deformation}}\]
Transformations might also change the Lagrangians. The interplay between
how the infinitesmal transformation changes the Lagrangian and the field is
what gives rise to conserved quantities, or sometimes known as Noether charges.
\begin{align*}
    \text{Symmetry} &\iff \text{Equations of motion -- invariant}\\
    &\iff \text{Action invariant (up to surface term)}\\
    &\iff \mc L(x) \rightarrow \mc L(x) + \alpha\di_mu \mc J^\mu(x)
\end{align*}
Taylor expanding the perturbation:
\begin{align*}
    \Delta \mc L &= \ddi{\mc L}{\vphi} \cdot \Delta \vphi + \ddi{\mc
    L}{(\di_\mu\vphi)}\di_\mu(\Delta\vphi)\\
    &= \di_\mu\left( \ddi{\mc L}{(\di_\mu\vphi)}\Delta\vphi \right) +
    \left[ \ddi{\mc L}{\vphi} - \di_\mu \left( \ddi{\mc L}{(\di_\mu \vphi)}
    \right) \right]\\
    &= \di_\mu\left( \ddi{\mc L}{(\di_\mu\vphi)}\Delta\vphi \right)\\
\end{align*}
Since we claimed that under the symmetry $\Delta\mc L = \di_\mu \mc J^\mu$
we have the following relations:
\begin{align*}
    j^\mu(x) &= \ddi{\mc L}{(\di_\mu\vphi)}\Delta\vphi - \mc J^\mu\\
    \di_\mu j^\mu &= 0\\
    \ddi{}{t} j^0 &= \di_i j^i
\end{align*}
Define the charge $Q = \int d^3x\; j^0$. Then, if we assume that space does
not have boundary, Stokes' theorem implies that $\di Q/\di t = 0$. Often,
$j^0$ is called the charge density, and $j^\mu$ is called the current
density.

\begin{mdframed}
Therefore to compute a conserved quantity we compare the deformation of the
Lagrangian due to the $\vphi$ changing with the deformation of the
Lagrangian due to the symmetry transformation. 
\end{mdframed}

\textbf{Examples:}
\begin{enumerate}
    \item $\mc L = \FR{1}{2}(\di_\mu \vphi)^2$ has the following field
        symmetry, $\vphi \to \vphi+\alpha$, ie. $\Delta\vphi \equiv$ const.
        There is no change to the Lagrangian, so $j^\mu = \di^\mu \vphi$.
    \item Space-time transformation, $x^\mu \to x^\mu-a^\mu$, implies
        \begin{align*}
        \vphi(x) &\to \vphi(x+a) = \vphi(x) + a^\nu\di_\nu\vphi(x)\\
        \mc L(x) &\to \mc L(x+a) = \mc L(x) + a^\mu\di_\mu \mc L\\
                 &. \quad\qquad\qquad = \mc L(x) +
                 a^\nu\di_\mu(\delta^\mu_\nu \mc L)
        \end{align*}
        Therefore we write 
        \[T^\mu_\nu = \ddi{\mc L}{(\di_\mu\vphi)}\di_\nu\vphi -
        \delta^\mu_\nu\mc L\]
        we get four separately conserved quantities. \CHECK

        This is called the stress-energy tensor or the energy-momentum
        tensor in various contexts.
        The $T^{\bullet 0}$ quantity gives rise to the Hamiltonian:
        \begin{align*}
            \int d^3x T^{00} = \int d^3x \mc H \equiv H
        \end{align*}

\end{enumerate}

\subsection{Summary of Computing Noether Charges}
Field or coordinate transformation $\leadsto$ $\{\Delta\phi,\Delta\mc L\}$
$\leadsto j^\mu(x) = \ddi{\mc L}{(\di_\mu\vphi)}\Delta\vphi - \mc J^\mu$
$\leadsto Q = \int j^0 d^3x$ conserved charge.

\section{Quantizing Klein Gordon Field}
Quantization of field theories involves two steps:
\begin{enumerate}
    \item Promote $\phi$ and $\pi$ to operators,
    \item Specify commutation relations:
        \begin{align*}
            [\phi(\vec x),\pi(\vec y)] &= i\delta^{(3)}(\vec x-\vec y)\\
            [\phi(\vec x),\phi(\vec y)] &= [\pi(\vec x),\pi(\vec y)] = 0
        \end{align*}
\end{enumerate}
Note that here we are in the Schrodinger picture, so that $\phi,\pi$ are
operators independent of time. Next, we are going to introduce a
basis that will diagonalize the Hamiltonian, and we shall express $\phi$
and $\pi$ in terms of these operators.

\elaborate

The exact form of the following is a little tricky to motivate (but this is
done in Weinberg, chapter 5). One way to motivate this is to look at the
Klein-Gordon equation in the Fourier representation. This is exactly the
harmonic oscillator equation and so it is reasonable to assume that the
operators $\phi,\pi$ can be manipulated in a similar way to obtain the
following ansatz:

\begin{align*}
    \phi(\vec x) &= \\
    \pi(\vec x) &=
\end{align*}

%   \chapter{Review}
%
%   \section{Field Theory}
%
%   \chapter{Free Field Theory}
%
%   \section{Lorentz Invariance}
%
%   \section{Canonical Quantization}
%
%   \section{Path Integral}
%
%   \chapter{Interacting Field Theory}
%
%   \chapter{Feynman Rules}
%
%   \chapter{Renormalization Theory}
%
%   \section{Renormalization Group}
%
%   \section{Effective Field Theories}
%
\end{document}
