\documentclass{report}
\usepackage{amsmath}
\usepackage{amssymb}
\usepackage{amsthm}
\usepackage{amscd}
\usepackage{fullpage}
\usepackage{braket}
\usepackage{dsfont}
\usepackage{mdframed}

\theoremstyle{plain}
\newtheorem{theorem}{Theorem}[section]
\newtheorem{lemma}[theorem]{Lemma}
\newtheorem{proposition}[theorem]{Proposition}
\newtheorem{corollary}[theorem]{Corollary}

\theoremstyle{definition}
\newtheorem{definition}[theorem]{Definition}
\newtheorem{axiom}{Axiom}
\newtheorem{example}[theorem]{Example}

\theoremstyle{remark}
\newtheorem*{remark}{Remark}
\newtheorem*{note}{Note}

\newcommand{\FR}[2]{\frac{#1}{#2}}
\newcommand{\PFR}[2]{\left(\frac{#1}{#2}\right)}
\newcommand{\SFR}[2]{\sqrt{\frac{#1}{#2}}}

\newcommand{\mc}{\mathcal}
\newcommand{\lam}{\lambda}
\newcommand{\vphi}{\varphi}
\newcommand{\om}{\omega}
\newcommand{\Om}{\Omega}
\newcommand{\gam}{\gamma}
\newcommand{\di}{\partial}
\newcommand{\ddi}[2]{\FR{\partial {#1}}{\partial {#2}}}
\newcommand{\hp}{\hat p}
\newcommand{\ha} {\hat a}
\newcommand{\had}{\hat a^\dagger}

\newcommand{\elaborate}{{\color{blue} \textbf{ELABORATE}}}
\newcommand{\CHECK}{{\color{blue} \textbf{CHECK}}}

\DeclareMathOperator{\bC}{\mathbb{C}}
\DeclareMathOperator{\bR}{\mathbb{R}}
\DeclareMathOperator{\bP}{\mathbb{P}}
\DeclareMathOperator{\bN}{\mathbb{N}}
\DeclareMathOperator{\bZ}{\mathbb{Z}}
\DeclareMathOperator{\cA}{\mathcal{A}}
\DeclareMathOperator{\cB}{\mathcal{B}}
\DeclareMathOperator{\cC}{\mathcal{C}}
\DeclareMathOperator{\cD}{\mathcal{D}}
\DeclareMathOperator{\cF}{\mathcal{F}}
\DeclareMathOperator{\cH}{\mathcal{H}}
\DeclareMathOperator{\cI}{\mathcal{I}}
\DeclareMathOperator{\cJ}{\mathcal{J}}
\DeclareMathOperator{\cL}{\mathcal{L}}
\DeclareMathOperator{\fg}{\mathfrak{g}}
\DeclareMathOperator{\id}{id}
\DeclareMathOperator{\im}{im}
\DeclareMathOperator{\Ad}{Ad}
\DeclareMathOperator{\Sp}{Sp}
\DeclareMathOperator{\SO}{SO}
\DeclareMathOperator{\Hom}{Hom}
\DeclareMathOperator{\Aut}{Aut}

\begingroup
    \makeatletter
    \@for\theoremstyle:=definition,remark,plain\do{%
        \expandafter\g@addto@macro\csname th@\theoremstyle\endcsname{%
            \addtolength\thm@preskip\parskip
            }%
        }
\endgroup

\edef\restoreparindent{\parindent=\the\parindent\relax}
\usepackage{parskip}
\restoreparindent

\setcounter{chapter}{-1}

\title{Quantum Field Theory\\Fall 2015 Seminar Notes}
\author{Anton Borissov, Henry Liu}
\date{\today}

\begin{document}

\maketitle

\tableofcontents

\chapter{Review}

Let's review some notation and concepts from quantum mechanics and
special relativity. Note that not all these concepts carry over to
quantum field theory. For example, in quantum mechanics, we are always
working with states within some Hilbert space, but in quantum field
theory, there is no suitable Hilbert space.

\section{Quantum Mechanics}

The fundamentals of quantum mechanics have had almost a century to be
formalized, and indeed they have been! Here we give a somewhat
axiomatic presentation of QM.

\begin{axiom}[States]
  Let $\cH$ be a (complex) Hilbert space. Its projectivization
  $\bP\cH$ is the {\bf state space} of our system.
  \begin{enumerate}
  \item An element of $\cH$, i.e. a state, is called a {\bf ket}, and
    is written $\ket{x}$.
  \item An element of $\cH^*$, i.e. a functional, is called a {\bf
      bra}. The bra associated to $\ket{x}$ (under the identification
    $\cH \cong \cH^*$ given by the inner product) is denoted $\bra{x}$.
  \end{enumerate}
  Consequently, $\braket{x|x} = \|x\|^2$, which we usually want to
  normalize to be $1$.

  Note that the symbol inside the ket or bra is somewhat arbitrary.
  For example, the states of a quantum harmonic oscillator are written
  $\ket{n}$, for $n \in \bN$.
\end{axiom}

Given a space of states, we can look at the operators that act on the
states. These operators must be unitary, so that normalized states go
to normalized states. 

\begin{axiom}[Observables]
  To every classical observable (i.e. property of a system) is
  associated a quantum operator, called an {\bf observable}.
  Observables are (linear) self-adjoint operators whose (real!)
  eigenvalues are possible values of the corresponding classical
  property of the system. For example,
  \begin{enumerate}
  \item $\hat{H}$ is the {\bf Hamiltonian} of the system, which
    classically represents the ``total energy'' (kinetic + potential)
    in the system,
  \item $\hat{x}$ is the {\bf position operator},
  \item $\hat{p}$ is the {\bf momentum operator}.
  \end{enumerate}
  The convention in QM is that observables are denoted by symbols with
  hats on them. The process of ``moving'' from a classical picture of
  a system to a quantum picture by making classical observables into
  operators is called {\bf quantization}, because the possible values
  of the observables are often quantized, i.e. made discrete, whereas
  previously they formed a continuum.
\end{axiom}

A classical observable is simple: it is just a function $f$ defined on
the classical phase space, so in order to make a measurement of the
observable, we simply apply $f$ to the current state of the system.
In QM it is not as simple, in most part due to its inherently
probabilistic nature. But it is still straightforward.

\begin{axiom}[Measurement]
  If $\hat{A}$ is the observable and $\hat{A}\ket{k} = a_k\ket{k}$,
  i.e. $\ket{k}$ is an eigenstate with eigenvalue $a_k \in \bR$, then
  the probability of obtaining $a_k$ as the value of the measurement
  on $\ket{\psi}$ is $|\braket{k|\psi}|^2$. But not only is the
  outcome probabilistic, the state of the system after the measurement
  is $\ket{k}$. In other words, {\bf measurement is projection}. This
  is fundamental to QM and cannot be emphasized enough.

  There are some conventions for position and momentum eigenstates.
  Since $\hat{x}$ and $\hat{p}$ are conventional symbols to use for
  position and momentum respectively, the states $\ket{x}$ and
  $\ket{p}$ are position and momentum eigenstates with eigenvalues $x$
  and $p$ respectively.
\end{axiom}

What about states that we don't measure? What are they doing as time
passes? We need to specify the {\bf dynamics} of our system, and this
is where the quantum analog of the Hamiltonian comes into play.

\begin{axiom}[Dynamics]
  The time-evolution of the state $\ket{\psi}$ is specified by the
  Hamiltonian $\hat{H}$ of the system, and is given by the {\bf
    Schr\"odinger equation}
  $$ i\hbar \frac{d\ket{\psi}}{dt} = \hat{H}\ket{\psi}, $$
  where $\hbar$ is Planck's constant (later we will be working in
  units where $\hbar = 1$). Note that we can solve this first-order ODE:
  $$ \ket{\psi(t)} = \exp(-i\hat{H}t)\ket{\psi(0)}. $$
  The operator $U(t) = \exp(-i\hat{H}t)$ is known as the {\bf
    time-evolution operator}.
\end{axiom}

That's it! There are some quick consequences of these axioms we should
explore before moving on. First, although measurement is
probabilistic, we often work with states whose observables tend to
take on values clumped around a certain value, which corresponds to
the classical value of that observable for the system. So given a
state $\ket{\psi}$ and observable $\hat{A}$, it is reasonable to
define the {\bf expectation value} and {\bf standard deviation}
$$ \braket{\hat{A}} = \braket{\psi|\hat{A}|\psi}, \qquad \Delta \hat{A} = \sqrt{\braket{\hat{A}^2} - \braket{\hat{A}}^2}. $$

\begin{proposition}[Heisenberg's uncertainty principle]
  Let $\hat{A}$ and $\hat{B}$ be self-adjoint operators. Then
  $$ \Delta \hat{A} \Delta \hat{B} \ge \frac{1}{2}\left|\braket{[\hat{A}, \hat{B}]}\right|. $$
\end{proposition}

\begin{proof}
  Note that the variance can also be written 
  $$ \Delta \hat{A} = \braket{\psi|(\hat{A} - \braket{A})^2|\psi}. $$
  Without loss of generality, assume
  $\braket{\hat{A}} = \braket{\hat{B}} = 0$, since we can shift
  $\hat{A}$ and $\hat{B}$ by constants without affecting
  $\Delta \hat{A}$ and $\Delta \hat{B}$. Then an application of
  Cauchy-Schwarz (using braket notation) gives
  $$ \Delta \hat{A} \Delta \hat{B} = \|\hat{A}\ket{\psi}\| \|\hat{B}\ket{\psi}\| \ge \left|\braket{\psi|\hat{A}\hat{B}|\psi}\right|. $$
  Now note that if $z = \braket{\psi|\hat{A}\hat{B}|\psi}$, then
  $|z| \ge |\im z| = |z - z^*|/2$. Hence
  $$ \left|\braket{\psi|\hat{A}\hat{B}|\psi}\right| \ge \frac{1}{2}\left|\braket{\psi|\hat{A}\hat{B}|\psi} - \braket{\psi|\hat{A}\hat{B}|\psi}^*\right| = \frac{1}{2}\left|\braket{\psi|\hat{A}\hat{B} - (\hat{A}\hat{B})^\dagger|\psi}\right| = \frac{1}{2}\left|\braket{\psi|[\hat{A}, \hat{B}]|\psi}\right|, $$
  where the last equality follows from the observables being
  self-adjoint:
  $(\hat{A}\hat{B})^\dagger = \hat{B}^\dagger\hat{A}^\dagger =
  \hat{B}\hat{A}$.
\end{proof}

For example, if we have a particle in $\bR^n$, the Hilbert space
underlying the state space is $\cH = L^2(\bR^n)$, and the position and
momentum operators are given by
$$ \hat{x}: \psi(x) \mapsto x\psi(x), \quad \hat{p}: \psi(x) \mapsto -i\hbar\nabla\psi(x). $$
A short calculation gives the {\bf fundamental commutation relation}
between $\hat{x}$ and $\hat{p}$:
$$ [\hat{x}, \hat{p}] = i\hbar, $$
which we interpret as saying that we cannot know both the exact
position and exact momentum of a particle at the same time.

\section{Special Relativity}

\setcounter{axiom}{0}

Special relativity describes the structure of spacetime. It says that
spacetime is $\bR^{1+3}$, known as {\bf Minkowski space} (as opposed
to $\bR^4$, Euclidean space) and equipped with the {\bf Minkowski
  metric}
$$ ds^2 = c^2 dt^2 - dx^2 - dy^2 - dz^2 $$
where $c$ is the speed of light (later we will work in units where
$c = 1$). As with QM, there is a nice axiomatic presentation of SR,
which is essentially just the following axiom.

\begin{axiom}[Lorentz invariance]
  The fundamental laws of physics must be invariant under isometries
  of Minkowski space. These isometries form the {\bf Poincar\'e group}
  $\bR^{1+3} \rtimes \SO(1,3)$. The subgroup $\SO(1,3)$ is known as
  the {\bf Lorentz group}; its elements are called {\bf Lorentz
    transformations}, and are precisely the isometries leaving the
  origin fixed.
\end{axiom}

So any Hamiltonian, Lagrangian, or physical expression we write down
from now on had better be Lorentz invariant (we will usually work
locally with nicely-behaved objects that are automatically invariant
under the full Poincar\'e group if they are Lorentz invariant).

Along with special relativity, Einstein introduced his {\bf summation
  notation} for tensors:
\begin{enumerate}
\item Components of (contravariant) vectors $\vec{v}$ are written with
  superscripts, i.e. $\vec{v} = v^1 e_1 + \cdots + v^n e_n$, and those
  of (covariant) covectors with subscripts;
\item An index which appears both as a subscript and a superscript is
  implicitly summed over, i.e. $\vec{v} = v^i e_i$;
\item Unbound indices (the ones not summed over) must appear on both
  sides of an equation.
\end{enumerate}
For example, $T^{\mu \alpha} = g^{\mu \nu} T^\alpha_\nu$ demonstrates
contraction with the metric tensor. When there is superscript that
should be a subscript, or vice versa, the metric tensor is implicitly
being used to raise and lower indices.

There are several conventions regarding Einstein's summation notation.
Spacetime variables are indexed by Greek letters, e.g. $\mu$ or $\nu$,
which run from $0$ to $3$, while space-only variables are indexed by
Roman letters, e.g. $i$ or $j$, which run from $1$ to $3$. Given a
$4$-vector $v = v^\nu e_\nu$, we let $\vec{v} = v^i e_i$ be the
space-only component, and $v^2$ generally denotes $v^\mu v_\mu$
whereas $\vec{v}^2$ generally denotes $v^i v_i$.

\chapter{Klein-Gordon Field}
\section{Why Fields?}
    
    Volume 1 of Steven Weinberg's \emph{Quantum Theory of Fields} is
    devoted to answering this question. A discussion of scattering
    experiments lead him to the $S$-matrix, and then to the local behaviour
    of experiments (which he calls the cluster decomposition principle),
    and then using Lorentz invariance, fields just practically fall out.
    Weinberg does a really good job of convincing us that QFT in some form
    or another really must exist if we assume Lorentz invariance and
    unitarity.
%   First, he analyzes scattering
%   experiments and introduces us to the arena of the multiparticle Hilbert
%   space and the main player, the $S$-matrix. Next, using the cluster
%   decomposition principle, he justifies why a Hamiltonian must be written
%   as a sum of creation and annihilation operators. This cluster
%   decomposition principle makes precise what we mean by ``experiments at
%   large distances between one another are uncorrelated.'' Moreover, ``this
%   cluster decomposition principle plays a crucial part in making field
%   theory inevitable.'' (Weinberg Vol 1) This approach is very appealing
%   for it justifies why fields are important without citing the
%   ``problems'' of previous theories.
%
%   Miscelleanous remarks from the wisdom bank of Weinberg:
%   \begin{itemize}
%       \item {\color{red}``The structure and properties of any quantum field are
%           dictated by the representations of the homogenous Lorentz group
%       under which it transforms.''}
%       \item Free fields $\leftrightarrow$ trivial
%           representation, causal vector $\leftrightarrow$ 4-vector
%           representation, Dirac fields $\leftrightarrow$ dirac
%           representation, etc.
%   \end{itemize}

    Peskin and Schroeder give a slightly different motivation, one that is
    closer to the historical reason of why fields were introduced:
    \begin{itemize}
        \item Single particle relativistic wave functions $\implies$
            inconsistencies in theory (negative energy eigenstates)
            \elaborate

            The Dirac equation comes from forcing the Schr\"odinger
            equation $i(d\Psi/dt) = \hat{H}\Psi$ to be Lorentz
            invariant. As it stands, it is first-order in time, but
            second-order in space. Suppose instead that
            $$ \hat{H} = \frac{1}{i} \alpha^j \partial_j + m\beta. $$
            Since $E^2 = \vec{p}^2 + m^2$, we want $\hat{H}^2 =
            -\nabla^2 + m^2$, which gives
            $$ \alpha^j \alpha^k + \alpha^k \alpha^j = 2\delta^{jk},
            \quad \alpha^j \beta + \beta \alpha^j = 0, \quad \beta^2 =
            1. $$ Hence $\{\alpha^1, \alpha^2, \alpha^3, \beta\}$ are
            not scalars, but instead are the generators of a Clifford
            algebra; we take their simplest representation as
            matrices, which is as $4 \times 4$ complex matrices
            $$ \alpha^j = \begin{pmatrix} 0 & \sigma_j \\ \sigma_j &
              0 \end{pmatrix}, \quad \beta = \begin{pmatrix} I & 0
              \\ 0 & -I \end{pmatrix}, $$ where $\sigma_j$ are the
            Pauli matrices. Now compute in momentum-space that
            $$ \widehat{H\psi}(\vec{p}) = (-i \vec{p} \cdot
            \vec{\alpha} + m\beta) \hat{\psi}(\vec{p})
            = \begin{pmatrix} mI & \vec{p} \cdot \vec{\sigma}
              \\ \vec{p} \cdot \vec{\sigma} & -mI \end{pmatrix}
            \hat{\psi}(\vec{p}), $$ and a straightforward calculation
            shows that $\hat{H}$ has eigenvalues $\pm \sqrt{\vec{p}^2
              + m^2}$. In particular, the energy can be negative!

            Dirac attempted to resolve this issue by appealing to the
            Pauli exclusion principle and positing that there existed
            a whole ``sea of negative-energy states'' that were
            already occupied. Consequently, the holes in this sea
            would be antiparticles. This makes sense until we realize
            that that a particle falling into a negative-energy state
            would represent particle-antiparticle annihilation, but
            the Dirac equation is supposed to be modeling a single
            particle (an electron, actually). So philosophical issues
            aside, there are technical issues here.
            
        \item $E=mc^2$ allows for particles to be created at high energies
        \item $\Delta E\cdot \Delta t = \hbar$ allows for virtual
            particles
        \item Causality violation. Set $H = \FR{\hp^2}{2m}$
            \begin{align*}
                U(t) &= \braket{\vec x|e^{-iHt}|\vec x_0}\\
                     &= \int \FR{d^3p}{(2\pi)^3} \braket{\vec
                         x|e^{-i(p^2/2m)t}|p}\braket{p|x}\\
 &= \int \FR{d^3p}{(2\pi)^3} e^{-i(p^2/2m)t}e^{i\vec p\cdot (\vec x-\vec x_0)}\\
                     &= \PFR{m}{2\pi i t}^{3/2} e^{im(\vec x-\vec x_0)^2/2t}
            \end{align*}
            This last quantity is non-zero, even for arbitariry $x$ that
            may by space-like separated.
    \end{itemize}
    QFT seems to solve all of these mysteries. One very good feature of the
    theory is that it predicts a lot of experiments to very high accuracy.
    QED is something we will see very soon that has been very well tested
    and agrees very well with experiments.
    
\section{Elements of Classical Field Theory}
\subsection{Lagrangian Field Theory}
    \begin{itemize}
        \item Fundamental quantity in Lagrangian field theory is the action $S$.
In high school, the Lagrangian is a function of time,
positions, and velocities of a system: $L(t,x(t),\dot x(t))$. The action
is given by $S = \int dt\, L$.
Fields can also be described in a Lagrangian formalism, for instance by
considering every point in space-time as a ``particle'' that wiggles back
and forth with the amplitude of wiggling characterizing the strength of the
field.

Let $\vphi : M \to \bR$, define a Lagrangian \emph{density} $\mc
L(t,\vphi,\di_\mu \vphi)$, the honest Lagrangian $L = \int d^3x \mc L$, and
finally define the action: \[ S = \int dt\, L = \int d^4x \mc L \]

\begin{mdframed}
    Four-vector notation:
    \begin{itemize}
        \item Greek letters $\mu,\nu,\ldots \in \{0,1,2,3\}$
        \item Roman letters $i,g,\ldots \in \{1,2,3\}$.
        \item $x^\mu = (x^0,x^1,x^2,x^3)$
        \item Signature $(+---)$
        \item $\eta_{\mu\nu} = \mathrm{diag}(1,-1,-1,-1)$
        \item $\di_\mu f = \FR{\di f}{\di x^\mu} = (\di_0 f,\di_1 f,\di_2
            f,\di_3 f)$.
    \end{itemize}
\end{mdframed}

\item Extremize the action. Let $\delta f = f(\vphi+\xi) - f(\vphi)$.
    \begin{align*}
        0 = \delta S &= \int d^4 x \left( \ddi{\mc L}{\vphi}\delta\vphi
        +\ddi{\mc L}{(\di_\mu \vphi)}
        \underbrace{\delta(\di_\mu}_{\text{commute}}\vphi)\right)\\
        &= \int d^4 x \left[ \ddi{\mc L}{\vphi}\delta\vphi
        + \di_\mu \left(  \ddi{\mc L}{\vphi}\delta\vphi \right)
    - \di_\mu \left(\ddi{\mc L}{(\di_\mu\vphi)}\right)\delta\vphi \right]
    \end{align*}
    By Stokes' theorem, we can break this integral up into two parts, one
    of which is called the boundary term. Taking a variation that is fixed
    along the boundary means $\delta\vphi \equiv 0$ on the boundary which
    means that the boundary term does not contribute to $\delta S$.
    Moreover, if we take $\delta S = 0$ for every variation, then we obtain
    the Euler Lagrange equations:
    \[ \di_\mu \left( \ddi{\mc L}{(\di_\mu \vphi)} \right) - \ddi{\mc
    L}{\vphi} = 0 \]

    \begin{remark}
        The Lagragian formalism is useful for relativistic dynamics because
        all expressions are chosen to Lorentz invariant.
    \end{remark}
    \end{itemize}
\subsection{Hamiltonian Field Theory}
    \begin{itemize}
\item Introducing this makes the transition to the quantum theory easier.
\item High school Hamiltonian formalism: 
    $p = \ddi{L}{\dot q},H=\sum p\dot q-L$.
\item Pretend that $\vec x$ enumerates points on the lattice of space-time:
    \begin{align*}
        p(\vec x) = \ddi{\mc L}{\dot\vphi(\vec x)}
        &= \ddi{}{\dot\vphi(\vec x)}\int d^3y\, \mc
        L(\vphi(y),\dot\vphi(y))\\
        &\sim \ddi{}{\dot\vphi(\vec x)}\sum \mc
        L(\vphi(y),\dot\vphi(y)) d^3y\\
        &= \ddi{\mc L}{\dot\vphi(\vec x)} d^3x\\
        &\equiv \pi(\vec x)d^3 x
    \end{align*}
    since each point on the lattice represents a different variable, so the
    derivative just picks out the one at $\vec x$. We call $\pi(\vec x)$
    the momentum \emph{density}. Therefore the Hamiltonian looks like:
    \[H = \int d^3x\, \left[\pi(\vec x)\dot\vphi(\vec x) - \mc L\right].\]
    (See the stress-energy tensor part for another derivation of the
    Hamiltonian which falls out of Noether's theorem for being the
    conserved quantity under time translations.)
\item \textbf{Important example:} Take $\mc L = \FR{1}{2}(\di_\mu\vphi)^2 -\FR{1}{2}
    m^2\vphi^2$. Euler-Lagrange equations become $\di^\mu(\di_\mu
    \vphi)+m^2\vphi=0$ which is the Klein Gordon equation. The Hamiltonian
    becomes:
    \[ H = \int d^3x \mc H
= \int d^3x \left[ \underbrace{\FR{\pi^2}{2}}_{\text{moving in time}}
    + \underbrace{\FR{(\nabla \vphi)^2}{2}}_{\text{shearing in space}}
+ \underbrace{\FR{m^2\vphi^2}{2}}_{\text{existing at all}}\right]\]
    \end{itemize}
\subsection{Noether's Theorem - How to Compute Conserved Quantities}
To every continuous transformation of the field we can assign an infinitesmal
transformation:
\[ \vphi(x) \rightarrow \vphi'(x) = \vphi(x) +
\alpha\underbrace{\Delta\vphi(x)}_{\text{deformation}}\]
Transformations might also change the Lagrangians. The interplay between
how the infinitesmal transformation changes the Lagrangian and the field is
what gives rise to conserved quantities, or sometimes known as Noether charges.
\begin{align*}
    \text{Symmetry} &\iff \text{Equations of motion -- invariant}\\
    &\iff \text{Action invariant (up to surface term)}\\
    &\iff \mc L(x) \rightarrow \mc L(x) + \alpha\di_mu \mc J^\mu(x)
\end{align*}
Taylor expanding the perturbation:
\begin{align*}
    \Delta \mc L &= \ddi{\mc L}{\vphi} \cdot \Delta \vphi + \ddi{\mc
    L}{(\di_\mu\vphi)}\di_\mu(\Delta\vphi)\\
    &= \di_\mu\left( \ddi{\mc L}{(\di_\mu\vphi)}\Delta\vphi \right) +
    \left[ \ddi{\mc L}{\vphi} - \di_\mu \left( \ddi{\mc L}{(\di_\mu \vphi)}
    \right) \right]\\
    &= \di_\mu\left( \ddi{\mc L}{(\di_\mu\vphi)}\Delta\vphi \right)\\
\end{align*}
Since we claimed that under the symmetry $\Delta\mc L = \di_\mu \mc J^\mu$
we have the following relations:
\begin{align*}
    j^\mu(x) &= \ddi{\mc L}{(\di_\mu\vphi)}\Delta\vphi - \mc J^\mu\\
    \di_\mu j^\mu &= 0\\
    \ddi{}{t} j^0 &= \di_i j^i
\end{align*}
Define the charge $Q = \int d^3x\; j^0$. Then, if we assume that space does
not have boundary, Stokes' theorem implies that $\di Q/\di t = 0$. Often,
$j^0$ is called the charge density, and $j^\mu$ is called the current
density.

\begin{mdframed}
Therefore to compute a conserved quantity we compare the deformation of the
Lagrangian due to the $\vphi$ changing with the deformation of the
Lagrangian due to the symmetry transformation. 
\end{mdframed}

\textbf{Examples:}
\begin{enumerate}
    \item $\mc L = \FR{1}{2}(\di_\mu \vphi)^2$ has the following field
        symmetry, $\vphi \to \vphi+\alpha$, ie. $\Delta\vphi \equiv$ const.
        There is no change to the Lagrangian, so $j^\mu = \di^\mu \vphi$.
    \item Space-time transformation, $x^\mu \to x^\mu-a^\mu$, implies
        \begin{align*}
        \vphi(x) &\to \vphi(x+a) = \vphi(x) + a^\nu\di_\nu\vphi(x)\\
        \mc L(x) &\to \mc L(x+a) = \mc L(x) + a^\mu\di_\mu \mc L\\
                 &. \quad\qquad\qquad = \mc L(x) +
                 a^\nu\di_\mu(\delta^\mu_\nu \mc L)
        \end{align*}
        Therefore we write 
        \[T^\mu_\nu = \ddi{\mc L}{(\di_\mu\vphi)}\di_\nu\vphi -
        \delta^\mu_\nu\mc L\]
        we get four separately conserved quantities. \CHECK

        This is called the stress-energy tensor or the energy-momentum
        tensor in various contexts.
        The $T^{\bullet 0}$ quantity gives rise to the Hamiltonian:
        \begin{align*}
            \int d^3x T^{00} = \int d^3x \mc H \equiv H
        \end{align*}

\end{enumerate}

\subsection{Summary of Computing Noether Charges}
Field or coordinate transformation $\leadsto$ $\{\Delta\phi,\Delta\mc L\}$
$\leadsto j^\mu(x) = \ddi{\mc L}{(\di_\mu\vphi)}\Delta\vphi - \mc J^\mu$
$\leadsto Q = \int j^0 d^3x$ conserved charge.

\section{Quantizing Klein Gordon Field}
Quantization of field theories involves two steps:
\begin{enumerate}
    \item Promote $\phi$ and $\pi$ to operators,
    \item Specify commutation relations:
        \begin{align*}
            [\phi(\vec x),\pi(\vec y)] &= i\delta^{(3)}(\vec x-\vec y)\\
            [\phi(\vec x),\phi(\vec y)] &= [\pi(\vec x),\pi(\vec y)] = 0
        \end{align*}
\end{enumerate}
Note that here we are in the Schrodinger picture, so that $\phi,\pi$ are
operators independent of time. Next, we are going to introduce a
basis that will diagonalize the Hamiltonian, and we shall express $\phi$
and $\pi$ in terms of these operators.

\elaborate

The exact form of the following is a little tricky to motivate (but this is
done in Weinberg, chapter 5). One way to motivate this is to look at the
Klein-Gordon equation in the Fourier representation. This is exactly the
harmonic oscillator equation and so it is reasonable to assume that the
operators $\phi,\pi$ can be manipulated in a similar way to obtain the
following ansatz:

\begin{align*}
    \phi(\vec x) &= \\
    \pi(\vec x) &=
\end{align*}

%   \chapter{Review}
%
%   \section{Field Theory}
%
%   \chapter{Free Field Theory}
%
%   \section{Lorentz Invariance}
%
%   \section{Canonical Quantization}
%
%   \section{Path Integral}
%
%   \chapter{Interacting Field Theory}
%
%   \chapter{Feynman Rules}
%
%   \chapter{Renormalization Theory}
%
%   \section{Renormalization Group}
%
%   \section{Effective Field Theories}
%
\end{document}
