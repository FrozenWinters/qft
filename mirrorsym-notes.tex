\documentclass{report}
\usepackage{amsmath}
\usepackage{amssymb}
\usepackage{amsthm}
\usepackage{amscd}
\usepackage[usenames,dvipsnames,svgnames,table]{xcolor}
\usepackage[colorlinks=true,urlcolor=blue,bookmarks=true,citecolor=blue]{hyperref}
\usepackage{fullpage}
\usepackage{braket}
\usepackage{slashed}
\usepackage{verbatim}
\usepackage{mathrsfs}
\usepackage{mathtools}
\usepackage{todonotes}

\theoremstyle{plain}
\newtheorem{theorem}{Theorem}[section]
\newtheorem{lemma}[theorem]{Lemma}
\newtheorem{proposition}[theorem]{Proposition}
\newtheorem{corollary}[theorem]{Corollary}

\theoremstyle{definition}
\newtheorem{definition}[theorem]{Definition}
\newtheorem{example}[theorem]{Example}
\newtheorem{exercise}{Exercise}[section]

\newcommand{\di}{\partial}
\newcommand{\sdi}{\slashed\partial}
\newcommand{\del}{\partial}
\newcommand{\delbar}{\bar\partial}
 % normal ordering
\newcommand{\NO}[1]{\vcentcolon\mathrel{#1}\vcentcolon\,}
% creation annihilation normal ordering
\newcommand{\circcolon}{\mathbin{\raise 0.75ex\hbox{\oalign{$\scriptscriptstyle\mathrm{o}$\cr$\scriptscriptstyle\mathrm{o}$}}}}
\newcommand{\CANO}[1]{\,\circcolon\mathrel{#1}\circcolon\,}

\newcommand{\bC}{\mathbb{C}}
\newcommand{\bP}{\mathbb{P}}
\newcommand{\bR}{\mathbb{R}}
\newcommand{\bZ}{\mathbb{Z}}
\newcommand{\cF}{\mathcal{F}}
\newcommand{\cM}{\mathcal{M}}
\newcommand{\cO}{\mathcal{O}}
\newcommand{\cU}{\mathcal{U}}
\DeclareMathOperator{\id}{id}
\newcommand{\Morse}{\mathrm{Morse}}
\DeclareMathOperator{\Hom}{Hom}
\DeclareMathOperator{\grad}{grad}
\DeclareMathOperator{\im}{im}
\DeclareMathOperator{\ind}{ind}
\DeclareMathOperator{\codim}{codim}
\DeclareMathOperator{\coker}{coker}
\newcommand{\chH}{\check{H}}
\newcommand{\dR}{\mathrm{dR}}
\newcommand{\dder}[2]{\frac{d #1}{d #2}}
\newcommand{\pder}[2]{\frac{\partial #1}{\partial #2}}
\newcommand{\fder}[2]{\frac{\delta #1}{\delta #2}}
\newcommand{\pdder}[2]{\frac{\partial^2 #1}{\partial #2^2}}
\newcommand{\bz}{\bar{z}}
\newcommand{\bu}{\bar{u}}
\newcommand{\bdi}{\bar{\di}}
\newcommand{\rep}[1]{\mathbf{#1}}

\begingroup
    \makeatletter
    \@for\theoremstyle:=definition,remark,plain\do{%
        \expandafter\g@addto@macro\csname th@\theoremstyle\endcsname{%
            \addtolength\thm@preskip\parskip
            }%
        }
\endgroup

\edef\restoreparindent{\parindent=\the\parindent\relax}
\usepackage{parskip}
\restoreparindent

\title{Mirror Symmetry\\Summer 2016 Seminar Notes}
\author{Anton Borissov, Henry Liu}
\date{\today}

\begin{document}
\hypersetup{pageanchor=false}
\maketitle
\hypersetup{pageanchor=true}

\tableofcontents

\chapter{Mathematical Preliminaries}

The aim of this chapter is to give a brief review of the required
mathematical background for mirror symmetry.

\section{Cohomology Theories}

Throughout this section, $X$ is a complex manifold, and $H_{\dR}$ is
de Rham cohomology. We examine the relationships between some common
cohomology theories on $X$.

\subsection{Sheaf Cohomology}

All of our sheaves take values in abelian groups. Let $\cF$ be a
presheaf on $X$.

\begin{definition}
  Recall the definition of a {\bf presheaf} $\cF$ on $X$:
  \begin{enumerate}
  \item (presheaf) every open set $U$ in $X$ is assigned an abelian
    group $\cF(U)$, such that if $V \subseteq U$ are two open sets,
    there is a restriction map $-|_{U \to V}\colon \cF(U) \to \cF(V)$
    compatible with inclusion, i.e. $(-|_{U \to V})|_{V \to W} = -|_{U
      \to W}$ for any $W \subseteq V \subseteq U$.
  \end{enumerate}
  If in addition $\cF$ satisfies the following two properties, it is a
  {\bf sheaf}:
  \begin{enumerate}
  \setcounter{enumi}{1}
  \item (locality) if $\{U_\alpha\}$ is an open cover of $X$ and $f, g
    \in \cF(X)$ such that $f|_{U \to U_\alpha} = g|_{U \to U_\alpha}$
    for every $U_\alpha$, then $f = g$;
  \item (gluing) if $\{U_\alpha\}$ is an open cover of $X$ and
    $f_\alpha \in \cF(U_\alpha)$ for every $U_\alpha$ are elements
    agreeing on overlaps, i.e. such that $f_\alpha|_{U_\alpha \to
      U_\alpha \cap U_\beta} = f_\beta|_{U_\beta \to U_\alpha \cap
      U_\beta}$, then we can glue the $f_\alpha$ together to get $f
    \in \cF(X)$, i.e. $f|_{X \to U_\alpha} = f_\alpha$ for every
    $U_\alpha$.
  \end{enumerate}
\end{definition}

\begin{definition}
  Let $\cU = \{U_\alpha\}$ be an {\bf ordered open cover} of $X$, i.e.
  with a partial order such that if $\alpha$ and $\beta$ are
  incomparable then $U_\alpha \cap U_\beta$ is empty. A {\bf
    $p$-simplex} $\sigma$ of $\cU$ is a totally ordered collection of
  open sets $U_{\alpha_0}, \ldots, U_{\alpha_p} \in \cU$; we call
  $U_{\alpha_0, \ldots, \alpha_p} \coloneqq U_{\alpha_0} \cap \cdots
  \cap U_{\alpha_p}$ its {\bf support}, and often refer to $\sigma$ by
  it instead. The {\bf $k$-th boundary component} of a $p$-simplex
  $U_{\alpha_0, \ldots, \alpha_p}$ is given by $\di_k U_{\alpha_0,
    \ldots, \alpha_p} \coloneqq U_{\alpha_0, \ldots, \hat{\alpha}_k,
    \ldots, \alpha_p}$. Cochains are maps from simplices to sheaf
  sections, and form a cochain complex:
  \[ C^p(\cU, \cF) \coloneqq \prod_{\alpha_0 < \cdots < \alpha_p} \cF(U_{\alpha_0, \ldots, \alpha_p}), \quad (\delta^p \omega)(\sigma) \coloneqq \sum_{k=0}^{p+1} (-1)^k \omega(\di_k \sigma)|_{\sigma} \colon C^p(\cU, \cF) \to C^{p+1}(\cU, \cF). \]
  The {\bf \v Cech cohomology of $\cU$} with coefficients in $\cF$,
  denoted $\chH^\bullet(\cU, \cF)$, is the cohomology of this complex.
\end{definition}

\begin{example}
  Let $\cF$ be a sheaf. By the gluing condition for a sheaf, a global
  section $f \in \cF(X)$ is defined by its values $f_\alpha \coloneqq
  f|_{X \to U_\alpha} \in \cF(U_\alpha)$ on every $U_\alpha$ in an
  open cover. These $f_\alpha$ form precisely the data for an element
  of $C^0(\cU, \cF)$, and satisfy the gluing condition $f_\alpha =
  f_\beta$ on $U_\alpha \cap U_\beta$, which is precisely the
  statement $\delta_0 f = 0$. Hence $\chH^0(\cU, \cF) = \cF(X)$ for a
  sheaf $\cF$.
\end{example}

\begin{example}
  Let $\cO$ denote the sheaf of holomorphic functions (and $\cO^*$ the
  nowhere-zero ones) on $\bP^1$. Recall that on $\bP^1$ we have the
  charts $U = \bP^1 \setminus \{S\}$ and $V = \bP^1 \setminus \{N\}$,
  with coordinates $u$ and $v$ respectively. To look at sections of
  $\cO(U)$ versus $\cO(V)$, we use the transition map $v = u^{-1}$ on
  $U \cap V$. The cochains for this open cover are
  \[ C^0(\cU, \cO) = \cO(U) \times \cO(V), \quad C^1(\cU, \cO) = \cO(U \cap V), \quad C^k(\cU, \cO) = 0 \; \forall k \ge 2. \]
  We compute the sheaf cohomology.
  \begin{enumerate}
  \item ($\chH^0(\cU, \cO)$) The boundary map $\delta_0$ maps $(f, g)
    \in C^0(\cU, \cO)$ to $g - f$. But
    \[ f = \sum_{k=0}^\infty f_k u^k, \quad g = \sum_{k=0}^\infty g_k v^k = \sum_{k=0}^\infty g_k u^{-k}, \]
    so $g - f = 0$ iff $f_k = g_k = 0$ for all $k > 0$, and $f_0 =
    g_0$. Hence $\chH^0(\cU, \cO) \cong \bC$, consisting of all
    constant functions.
  \item ($\chH^1(\cU, \cO)$) Given $h \in C^1(\cU, \cO)$, rewrite its
    Laurent expansion:
    \[ h = \sum_{k=-\infty}^\infty h_k u^k = \sum_{k=0}^\infty h_k u^k + \sum_{k=1}^{\infty} h_k v^k = -f + g \]
    where $f \in \cO(U)$ and $g \in \cO(V)$. Hence $h \in \im
    \delta_0$, and $\chH^1(\cU, \cO) = 0$.
  \item ($\chH^k(\cU, \cO)$) Trivially, $\chH^k(\cU, \cO) = 0$ for $k
    \ge 2$.
  \end{enumerate}
  Note that $\chH^0(\cU, \cO) \cong \bC$ is consistent with what we
  know so far, since $\chH^0(\cU, \cO) = \cO(\bP^1)$, and Liouville's
  theorem shows that $\cO(\bP^1)$ can only contain constant functions.
\end{example}

\begin{example}
  Recall the tautological line bundle $\cO(-1)$ and its dual $\cO(1)$
  on $\bP^n$; we have $\cO(n) = \cO(1)^n$. On the same charts on
  $\bP^1$, since $\cO(1)$ has transition function $u = v^{-1}$, we
  know $\cO(n)$ has transition function $u^n = v^{-n}$. To construct a
  global section of $\cO(n)$, given a monomial $v^k$ on $V$, we
  require $u^n v^k = u^{n-k}$ to be well-defined on $U$, so $k \le n$.
  In homogeneous coordinates $[x_0 : x_1]$, the global sections are
  therefore $x_0^n, x_0^{n-1}x_1, \ldots, x_1^n$, the homogeneous
  polynomials of degree $n$. The same story holds on $\bP^N$. Hence
  $\dim H^0(\bP^N, \cO(n)) = \binom{N+n-1}{n-1}$. In particular, there
  are $\binom{9}{5} = 126$ independent global sections of
  $\cO_{\bP^4}(5)$.
\end{example}

\begin{definition}
  The set of all open covers of $X$ form a directed set under
  refinement. The {\bf \v Cech cohomology of $X$} with coefficients in
  $\cF$ is the direct limit $\chH^n(X, \cF) \coloneqq \varinjlim_{\cU}
  \chH^n(\cU, \cF)$.
\end{definition}

\begin{definition}
  An ordered open cover $\{U_\alpha\}$ is {\bf good} if it is
  countable and every finite intersection $U_{\alpha_0, \ldots,
    \alpha_p}$ is either empty or contractible.
\end{definition}

\begin{theorem}[{\cite[Corollary of Theorem 5.4.1]{Godement1997}}]
  The \v Cech cohomology of a good cover $\cU$ is isomorphic to the \v
  Cech cohomology of $X$.
\end{theorem}

One can define {\bf sheaf cohomology} $H^n(X, \cF)$ as the right
derived functors of the global sections functor $\Gamma_X$ (i.e. $\cF
\mapsto \cF(X)$). For us, \v Cech and sheaf cohomology are
indistinguishable as long as we work with sheaves.

\begin{theorem}[{\cite[Theorem 5.10.1]{Godement1997}}]
  If $X$ is a paracompact topological space, then \v Cech cohomology
  $\chH^n(X, \cF)$ and sheaf cohomology $H^n(X, \cF)$ are isomorphic
  for any sheaf $\cF$.
\end{theorem}

\v Cech cohomology is also directly related to de Rham cohomology,
and, as we shall see, Dolbeault cohomology in the complex case. So we
can think of \v Cech cohomology classes as forms.

\begin{theorem}[\v Cech--de Rham isomorphism]
  Let $\bR$ denote the constant sheaf. There is a canonical
  isomorphism $\chH^k(X, \bR) \cong H_{\dR}^k(X)$ for each $k$.
\end{theorem}

\begin{proof}
  By the Poincar\'e lemma, the {\bf de Rham complex} of sheaves
  \[ 0 \to \bR \xrightarrow{\subset} \Omega^0(X) \xrightarrow{d} \Omega^1(X) \xrightarrow{d} \Omega^2(X) \xrightarrow{d} \cdots \]
  is exact (by checking exactness on the stalks). Let $Z^n \coloneqq
  \ker(d\colon \Omega^k \to \Omega^{k+1})$. The de Rham complex splits
  into a bunch of short exact sequences:
  \[ 0 \to d\Omega^{k-1} \cong Z^k \xrightarrow{\subset} \Omega^k \xrightarrow{d} Z^{k+1} \to 0. \]
  To each such short exact sequence is associated a long exact
  sequence of (sheaf) cohomology:
  \[ 0 \to H^0(X, Z^k) \to H^0(X, \Omega^k) \to H^0(X, Z^{k+1}) \to H^1(X, Z^k) \to H^1(X, \Omega^k) \to H^1(X, Z^{k+1}) \to \cdots. \]
  Fact: $H^i(X, \Omega^k) = 0$ for every $k$ and $i > 0$ (since
  $\Omega^k$ is a fine sheaf). Hence we obtain isomorphisms
  \[ H^{i+1}(X, Z^0) \cong H^i(X, Z^1) \cong \cdots \cong H^1(X, Z^i). \]
  But $Z^0 \cong \bR$ and we more commonly write
  \[ H^1(X, Z^i) = \coker(H^0(X, \Omega^i) \to H^0(X, Z^{i+1})) = Z^{i+1}(X)/d\Omega^k(X) = H^{i+1}_{\dR}(X). \]
  Hence $\chH^{i+1}(X, \bR) \cong H^{i+1}_{\dR}(X)$.
\end{proof}

\begin{definition}
  Let $E$ be a holomorphic vector bundle on $X$ and $\Omega^{0,q}(E)
  \coloneqq \Omega^{0,q}(X) \otimes \Gamma(E)$ denote the space of
  $E$-valued $(0,q)$-forms. {\bf Dolbeault cohomology} $H^{p,q}(E)$ is
  the cohomology of the complex
  \[ \cdots \xrightarrow{\bdi} \Omega^{0,q}(E) \xrightarrow{\bdi} \Omega^{0,q+1}(E) \xrightarrow{\bdi} \Omega^{0,q+2}(E) \xrightarrow{\bdi} \cdots. \]
  We write $H^{p,q}(X)$ for $E = \Lambda^pT^{1,0}X$. The dimensions
  $h^{p,q}(E) \coloneqq \dim_{\bC} H^{p,q}(E)$ are the {\bf Hodge
    numbers} of $X$ with respect to $E$.
\end{definition}

\begin{lemma}[$\bdi$-Poincar\'e lemma]
  $\bdi$-closed, i.e. holomorphic, implies $\bdi$-exact on $\bC^n$.
\end{lemma}

\begin{theorem}[\v Cech--Dolbeault isomorphism]
  Let $\Omega^{p,0}$ denote the sheaf of holomorphic $p$-forms on $X$.
  There is a natural isomorphism $\chH^q(X, \Omega^{p,0}) \cong
  H^{p,q}(X)$.
\end{theorem}

\begin{proof}
  Analogous to the proof of the \v Cech--de Rham isomorphism, except
  now using the $\bdi$-Poincar\'e lemma to establish the exactness of
  the complex
  \[ 0 \to \ker(\bdi\colon\Omega^{p,0}(X) \to \Omega^{p,1}(X)) \xrightarrow{\subset} \Omega^{p,0}(X) \xrightarrow{\bdi} \Omega^{p,1}(X) \xrightarrow{\bdi} \cdots. \qedhere \]
\end{proof}

\begin{definition}
  Consider the double complex $\Omega^{\bullet,\bullet}$ with
  differentials $\di$ and $\bdi$. The {\bf Fr\"olicher spectral
    sequence} is the spectral sequence of a double complex associated
  to $\Omega^{\bullet,\bullet}$. Since the total complex of
  $\Omega^{\bullet,\bullet}$ is $\Omega^\bullet(X)$, the Fr\"olicher
  spectral sequence converges to complex de Rham cohomology
  $H^\bullet_{\dR}(X, \bC)$.
\end{definition}

\subsection{Morse Homology}

Throughout this subsection, $f\colon X \to \bR$ is a smooth function,
and we equip $X$, viewed as a real manifold, with a Riemannian metric
$g$. We also assume $(f, g)$ is Morse--Smale, defined below.

\begin{definition}
  A {\bf critical point} of $f$ is a point $p \in X$ with $df_p = 0$.
  Define the {\bf Hessian}
  \[ H(f)_p\colon T_pX \to T_p^*X, \quad v \mapsto \nabla_v(df), \]
  which is independent of the choice of connection $\nabla$. (In
  coordinates, we recover the usual $\di^2f/\di x_i \di x_j$.) The
  critical point $p$ is {\bf non-degenerate} if the Hessian does not
  have zero eigenvalues. A non-degenerate critical point $p$ has {\bf
    Morse index} $\ind(p)$ the number of negative eigenvalues of the
  Hessian. The function $f$ is {\bf Morse} if all of its critical
  points are non-degenerate.
\end{definition}

\begin{definition}
  Recall that the {\bf gradient} of $f$ with respect to a metric $g$
  is the vector field $\grad f$ such that $g(\grad f, X) = Xf$.
  Equivalently, $\grad f = (df)^\sharp$. Let $\psi_t\colon X \to X$ be
  the one-parameter group of diffeomorphisms associated to the flow of
  $-\grad f$. The {\bf descending manifold} $D(p)$ and {\bf ascending
    manifold} $A(p)$ at a critical point $p$ are
  \begin{align*}
    D(p) &\coloneqq \{x \in X : \lim_{t \to -\infty} \psi_t(x) = p\} \\
    A(p) &\coloneqq \{x \in X : \lim_{t \to +\infty} \psi_t(x) = p\}.
  \end{align*}
  The pair $(f, g)$ is {\bf Morse--Smale} if $f$ is Morse and $D(p)$
  is transverse to $A(q)$ for every pair of critical points $p$ and
  $q$ (i.e. tangent spaces of $D(p)$ and $A(q)$ generate the tangent
  space at every intersection point).
\end{definition}

Here are two useful and easy-to-prove facts: every flow line
asymptotically approaches critical points, and $\dim D(p) = \ind(p)$
(so $\dim A(p) = \dim X - \ind p$ by the Morse--Smale condition).

\begin{definition}
  Fix critical points $p$ and $q$. A {\bf flow line} from $p$ to $q$
  is an integral curve $\gamma(t)$ of $-\grad f$ with $\lim_{t \to
    -\infty} \gamma(t) = p$ and $\lim_{t \to +\infty} \gamma(t) = q$.
  The {\bf moduli space of flow lines} from $p$ to $q$ is
  \begin{align*}
    \cM(p, q) &\coloneqq \{\text{flow lines from } p \text{ to } q\}/\sim,
    \quad \alpha \sim \beta \text{ if } \exists c \in \bR : \alpha(t) = \beta(t + c) \\
    &= (D(p) \cap A(q)) / \bR.
  \end{align*}
  A {\bf broken flow line} consists, piecewise, of flow lines.
\end{definition}

The Morse--Smale condition implies $D(p) \cap A(q)$ is a submanifold
of $X$ with dimension $\ind(p) - \ind(q)$. Since $\sim$ is a smooth,
proper, free $\bR$-action, $\cM(p, q)$ is a manifold of dimension
$\ind(p) - \ind(q) - 1$ when $p \neq q$ (otherwise the $\bR$-action is
trivial). Note that if $\ind(p) = k$ and $\ind(q) = k-1$ then $\cM(p,
q)$ is zero-dimensional. In fact, in this case, $\cM(p, q)$ is compact
as a corollary of the following theorem, and therefore is a finite set
of points.

\begin{theorem}[{\cite[Theorem 2.1]{Hutchings2012}}]
  Let $X$ be closed and $(f, g)$ Morse--Smale. Then $\cM(p, q)$ has a
  natural compactification to a smooth manifold with corners
  $\overline{\cM(p, q)}$ where
  \[ \overline{\cM(p, q)} \setminus \cM(p, q) = \bigcup_{k \ge 1} \bigcup_{\substack{p,r_1,\ldots,r_k,q\\\text{distinct crit pts}}} \cM(p, r_1) \times \cM(r_1, r_2) \times \cdots \times \cM(r_{k-1}, r_k) \times \cM(r_k, q). \]
\end{theorem}

\begin{corollary}
  If $\ind(p) - \ind(q) = 1$, then $\overline{\cM(p, q)} = \cM(p, q)$
  is compact. If $\ind(p) - \ind(q) = 2$, then
  \[ \di \overline{\cM(p, q)} = \bigcup_{\ind(r) = \ind(p)-1} \cM(p, r) \times \cM(r, q). \]
\end{corollary}

\begin{proof}
  Since $\dim \cM(r, s) = \ind(r) - \ind(s) - 1$, the space $\cM(r,
  s)$ is non-empty only if $\ind(r) - \ind(s) \ge 1$. Hence
  $\overline{\cM(p, q)}\setminus \cM(p, q) = \emptyset$ when $\ind(p)
  - \ind(q) = 1$. Similar reasoning shows the $\ind(p) - \ind(q) = 2$
  case.
\end{proof}

\begin{definition}
  Fix orientations for $D(p)$ at every critical point $p$. There is an
  isomorphism at $x \in \gamma \in \cM(p, q)$ given by
  \begin{align*}
    T_xD(p) &\cong T_x(D(p) \cap A(q)) \oplus (T_xX/T_xA(q)) && \text{transversality from Morse--Smale} \\
    &\cong T_\gamma \cM(p, q) \oplus T_x\gamma \oplus (T_xX/T_xA(q)) && \text{definition of $\cM(p, q)$} \\
    &\cong T_\gamma \cM(p, q) \oplus T_x\gamma \oplus T_qD(q) && \text{translating $T_qD(q)$ along $\gamma$.}
  \end{align*}
  The {\bf orientation} on $\cM(p, q)$ is such that this isomorphism
  is orientation-preserving. Let $C_k$ be the free abelian group
  generated by critical points of index $k$, and define the {\bf
    Morse--Smale--Witten boundary map}
  \[ \di_k^{\Morse}\colon C_k \to C_{k-1}, \quad p \mapsto \sum_{\ind q = k-1} \# \cM(p, q) q \]
  where $\# \cM(p, q) \in \bZ$ is counted with sign according to the
  orientation of $\cM(p, q)$, which here is a discrete set of points.
\end{definition}

\begin{lemma}
  $(\di_k^\Morse)^2 = 0$, so $(C_\bullet, \di^\Morse)$ is a chain
  complex.
\end{lemma}

\begin{proof}
  Let $\ind(p) - \ind(q) = 2$. The coefficient of $q$ in
  $(\di^\Morse)^2p$ is
  \[ \sum_{\ind(r) = \ind(p)-1} \# \cM(p, r) \cdot \# \cM(r, q) = \# \bigcup_{\ind(r) = \ind(p)-1} \cM(p, r) \times \cM(r, q) = \# \di \overline{\cM(p, q)}. \]
  Since $\overline{\cM(p, q)}$ is an oriented $1$-manifold with
  boundary, this quantity, the number of boundary points, is
  zero.
\end{proof}

\begin{definition}
  {\bf Morse homology} $H_\bullet^\Morse(f, g)$ is the homology of the
  {\bf Morse--Smale--Witten complex} $(C_\bullet, \di^{\Morse})$.
\end{definition}

\begin{example}
  The (upright) torus $T^2$ has four critical points with $f$ the
  height function: $p$ (index 2), $q$ and $r$ (index 1), and $s$
  (index 0). This choice of $f$ is Morse, but with the induced metric
  $g$ from $\bR^3$, the pair $(f, g)$ is not Morse--Smale: $D(q) \cap
  A(r)$ is non-empty, but transversality forces it to be. The solution
  is to tilt the torus a little; equivalently, perturb $g$. There are
  two flow lines, of opposite sign, for each relevant pair of critical
  points. Hence $\di_k^\Morse = 0$ for $k = 1, 2$. It follows that
  \[ H_2^\Morse(f, g) = \bZ, \quad H_1^\Morse(f,g) = \bZ^2, \quad H_0^\Morse(f, g) = \bZ. \]
\end{example}

\begin{theorem}[{\cite[Theorem 3.1]{Hutchings2012}}]
  Let $X$ be a closed smooth manifold, $H_\bullet(X)$ denote singular
  homology on $X$, and $(f, g)$ be a Morse--Smale pair on $X$. Then
  there is a canonical isomorphism $H_n^\Morse(f, g) \cong H_n(X)$.
\end{theorem}

\begin{corollary}
  The number of critical points of a Morse function is at least the
  sum $\sum_k \dim H_k(X)$ of the Betti numbers.
\end{corollary}

\begin{proof}
  The number of critical points is the sum of the dimensions of the
  Morse chain groups, which is at least the sum of the dimensions of
  the Morse homology groups, which is equal to the sum of the
  dimensions of the singular homology groups.
\end{proof}

The infinite-dimensional analogue of Morse homology is known as {\bf
  Floer homology}. We shall primarily be concerned with Floer homology
for mirror symmetry.

\subsection{Equivariant Cohomology}



\section{Differential Topology}

We stop distinguishing between isomorphic cohomology theories now. In
particular, since $X$ is always at least a smooth manifold, we think
of singular cohomology $H^k(X)$ as de Rham cohomology.

\subsection{Poincar\'e Duality}

Unless otherwise stated, $X$ in this section is a compact oriented
$n$-manifold.

\begin{theorem}[Poincar\'e duality, \cite{Bott1982}]
  Let $X$ be a compact oriented $n$-manifold. The map
  \[ \int_X\colon H^k(X) \otimes H^{n-k}(X) \to \bR, \quad \omega \otimes \eta \mapsto \int_X \omega \wedge \eta \]
  is a perfect pairing, and hence $H^k(X) \cong H^{n-k}(X)^*$.
\end{theorem}

If we relax the assumption that $X$ is compact, then the issue is that
$\int_X$ may not be well-defined. We work around this by using
de Rham cohomology with compact support.

\begin{definition}
  Let $\Omega_c^k(X)$ denote the $k$-forms on $X$ with compact
  support. The {\bf de Rham cohomology groups with compact support}
  $H^n_c(X)$ are the cohomology of the chain complex
  $(\Omega_c^\bullet(X), d)$.
\end{definition}

\begin{theorem}[Poincar\'e duality for non-compact manifolds, \cite{Bott1982}]
  Let $X$ be an oriented $n$-manifold without boundary. The map
  \[ \int_X\colon H^k(X) \otimes H^{n-k}_c(X) \to \bR, \quad \omega \otimes \eta \mapsto \int_X \omega \wedge \eta \]
  is a perfect pairing, and hence $H^k(X) \cong H^{n-k}_c(X)^*$.
\end{theorem}

\begin{definition}
  Fix $C \subset X$ a closed $(n-k)$-submanifold. Then Poincar\'e
  duality identifies the map $\int_C\colon H^{n-k}(X) \to \bR$
  with a $k$-form $\eta_C \in H^k(X)$, called the {\bf
    Poincar\'e dual class}. Explicitly, $\int_C \omega = \int_X \omega
  \wedge \eta_C$.
\end{definition}

There is a relation between the Poincar\'e dual class and the Thom
class, which we define below. Namely, the Poincar\'e dual class of $C$
can be constructed as the Thom class of the normal bundle of $C$ in
$X$.

\begin{theorem}[{\cite[Theorem 10.4]{Milnor1974}}]
  Let $\pi\colon E \to B$ be an oriented rank-$n$ real vector bundle
  and $B$ is embedded into $E$ as the zero section. Then
  \begin{enumerate}
  \item there exists a unique cohomology class $\Phi \in H^n(E, E
    \setminus B)$ called the {\bf Thom class} such that for every
    $x \in B$, the restriction of $u$ to $H^n(E_x, E_x \setminus
    \{0\})$ is the preferred generator specified by the
    orientation of $E_x$ in $E$;
  \item the {\bf Thom isomorphism} $\colon H^k(E) \to H^{k+n}(E, E
    \setminus B)$, given by $\omega \mapsto \omega \wedge \Phi$, is an
    isomorphism for every $k$.
  \end{enumerate}
\end{theorem}

Note that since $B$ is a deformation retract of $E$, the rings
$H^*(E)$ and $H^*(B)$ are isomorphic. Hence $\pi^*\Phi = 1 \in H^*(B)$,
which shall be very important in the upcoming proof.

\begin{theorem}[Tubular neighborhood theorem, {\cite[Theorem 11.1]{Milnor1974}}]
  Let $C \subset X$ be a $k$-submanifold embedded in $X$. There exists
  an open neighborhood, called a {\bf tubular neighborhood}, of $C$ in
  $X$ diffeomorphic to the total space of the normal bundle of $C$.
  This diffeomorphism maps points in $C$ to zero vectors.
\end{theorem}

\begin{proposition}[{\cite[Proposition 6.24a]{Bott1982}}]
  Let $C \subset X$ be a closed $(n-k)$-submanifold. The Poincar\'e
  dual class $\eta_C \in H^k(X)$ of $C$ is the Thom class of the
  normal bundle of $C$ in $X$.
\end{proposition}

\begin{proof}
  Let $NC$ denote the normal bundle of $C$ in $X$, which has rank $k$
  because $C$ is codimension $k$. Use the tubular neighborhood theorem
  to identify $NC$ with an open neighborhood $T$ of $C$ in $X$, and
  then extend by zero to get $\Phi \in H^k(X)$ supported on $T$.

  We shall show that $\int_X \omega \wedge \Phi = \int_C \omega$ for
  any $\omega \in H^{n-k}_c(X)$. The maps $\pi\colon T \to C$ and
  $\iota\colon C \to T$ induce isomorphisms of cohomology, so on forms
  $\omega$ and $\pi^*\iota^*\omega$ differ by at most an exact form
  $d\tau$. Then
  \begin{align*}
    \int_X \omega \wedge \Phi
    &= \int_T \omega \wedge \Phi = \int_T (\pi^*\iota^*\omega + d\tau) \wedge \Phi \\
    &= \int_T \pi^*\iota^*\omega \wedge \Phi = \int_C \iota^*\omega \wedge \pi^* \Phi = \int_C \iota^*\omega. \qedhere
  \end{align*}
\end{proof}

\begin{corollary}
  Transverse intersection is Poincar\'e dual to the wedge product,
  i.e. for $C, D \subset X$ closed submanifolds intersecting
  transversally, $\eta_{C \cap D} = \eta_C \wedge \eta_D$.
\end{corollary}

\begin{proof}
  For transversal intersections, codimension is additive: $\codim C
  \cap D = \codim C + \codim D$. So the normal bundle of the
  intersection is $N(C \cap D) = NC \oplus ND$. Let $\Phi(E)$ denote
  the Thom class associated to the vector bundle $E$. By the
  characterization of the Thom class, for vector bundles $E$ and $F$
  we have $\Phi(E \oplus F) = \Phi(E) \wedge \Phi(F)$; check that
  $\Phi(E) \oplus \Phi(F)$ restricts on each fiber to the preferred
  generator. Hence
  \[ \eta_{C \cap D} = \Phi(N_{C \cap D}) = \Phi(NC \oplus ND) = \Phi(NC) \wedge \Phi(ND) = \eta_C \wedge \eta_D. \qedhere \]
\end{proof}

\subsection{Serre Duality}


\subsection{Characteristic Classes}

\subsection{The Grothendieck--Riemann--Roch Formula}

\section{Calabi--Yau Manifolds}

\section{Toric Geometry}

\addcontentsline{toc}{chapter}{Bibliography}
\bibliographystyle{unsrt}
\bibliography{mirrorsym-notes}

\end{document}
