\documentclass{report}
\usepackage{amsmath}
\usepackage{amssymb}
\usepackage{amsthm}
\usepackage{amscd}
\usepackage[usenames,dvipsnames,svgnames,table]{xcolor}
\usepackage[colorlinks=true,urlcolor=blue,bookmarks=true,citecolor=blue]{hyperref}
\usepackage{fullpage}
\usepackage{braket}
\usepackage{slashed}
\usepackage{verbatim}
\usepackage{mathrsfs}
\usepackage{mathtools}
\usepackage{todo}

\theoremstyle{plain}
\newtheorem{theorem}{Theorem}[section]
\newtheorem{lemma}[theorem]{Lemma}
\newtheorem{proposition}[theorem]{Proposition}
\newtheorem{corollary}[theorem]{Corollary}

\theoremstyle{definition}
\newtheorem{definition}[theorem]{Definition}
\newtheorem{example}[theorem]{Example}
\newtheorem{exercise}{Exercise}[section]

\theoremstyle{remark}
\newtheorem*{remark}{Remark}

\newcommand{\di}{\partial}
\newcommand{\sdi}{\slashed\partial}
\newcommand{\del}{\partial}
\newcommand{\delbar}{\bar\partial}
 % normal ordering
\newcommand{\NO}[1]{\vcentcolon\mathrel{#1}\vcentcolon\,}
% creation annihilation normal ordering
\newcommand{\circcolon}{\mathbin{\raise 0.75ex\hbox{\oalign{$\scriptscriptstyle\mathrm{o}$\cr$\scriptscriptstyle\mathrm{o}$}}}}
\newcommand{\CANO}[1]{\,\circcolon\mathrel{#1}\circcolon\,}

\newcommand{\bC}{\mathbb{C}}
\newcommand{\bP}{\mathbb{P}}
\newcommand{\bR}{\mathbb{R}}
\newcommand{\bZ}{\mathbb{Z}}
\newcommand{\cF}{\mathcal{F}}
\newcommand{\cM}{\mathcal{M}}
\newcommand{\cO}{\mathcal{O}}
\newcommand{\cU}{\mathcal{U}}
\DeclareMathOperator{\id}{id}
\newcommand{\Morse}{\mathrm{Morse}}
\DeclareMathOperator{\Hom}{Hom}
\DeclareMathOperator{\Pic}{Pic}
\DeclareMathOperator{\End}{End}
\DeclareMathOperator{\grad}{grad}
\DeclareMathOperator{\im}{im}
\DeclareMathOperator{\tr}{tr}
\DeclareMathOperator{\ch}{ch}
\DeclareMathOperator{\td}{td}
\DeclareMathOperator{\ind}{ind}
\DeclareMathOperator{\codim}{codim}
\DeclareMathOperator{\coker}{coker}
\DeclareMathOperator{\Ad}{Ad}
\newcommand{\chH}{\check{H}}
\newcommand{\dR}{\mathrm{dR}}
\newcommand{\dder}[2]{\frac{d #1}{d #2}}
\newcommand{\pder}[2]{\frac{\partial #1}{\partial #2}}
\newcommand{\fder}[2]{\frac{\delta #1}{\delta #2}}
\newcommand{\pdder}[2]{\frac{\partial^2 #1}{\partial #2^2}}
\newcommand{\bz}{\bar{z}}
\newcommand{\bu}{\bar{u}}
\newcommand{\bdi}{\bar{\di}}
\newcommand{\rep}[1]{\mathbf{#1}}


\newcommand{\mc}{\mathcal}
\newcommand{\ms}{\mathscr}
\newcommand{\mf}{\mathfrak}
\newcommand{\cnj}{\overline}
\newcommand{\sg}{\sigma}
\newcommand{\dsum}{\oplus}
\newcommand{\ten}{\otimes}
\newcommand{\non}{\nonumber}
\newcommand{\hook}{\,\lrcorner\,}

\newcommand{\lam}{\lambda}
\newcommand{\om}{\omega}
\newcommand{\Om}{\Omega}
\newcommand{\gam}{\gamma}

\newcommand{\FR}[2]{\frac{#1}{#2}}
\newcommand{\PFR}[2]{\left(\frac{#1}{#2}\right)}
\newcommand{\SFR}[2]{\sqrt{\frac{#1}{#2}}}

\DeclareMathOperator{\Tr}{tr}
\DeclareMathOperator{\sign}{sign}
\DeclareMathOperator{\Harm}{Harm}


\begingroup
    \makeatletter
    \@for\theoremstyle:=definition,remark,plain\do{%
        \expandafter\g@addto@macro\csname th@\theoremstyle\endcsname{%
            \addtolength\thm@preskip\parskip
            }%
        }
\endgroup

\edef\restoreparindent{\parindent=\the\parindent\relax}
\usepackage{parskip}
\restoreparindent

\title{Mirror Symmetry\\Summer 2016 Seminar Notes}
\author{Anton Borissov, Henry Liu}
\date{\today}

\begin{document}
\hypersetup{pageanchor=false}
\maketitle
\hypersetup{pageanchor=true}

\tableofcontents

\chapter{Mathematical Preliminaries}

The aim of this chapter is to give a brief review of the required
mathematical background for mirror symmetry.

\section{Cohomology Theories}

Throughout this section, $X$ is a complex manifold, and $H_{\dR}$ is
de Rham cohomology. We examine the relationships between some common
cohomology theories on $X$.

\subsection{Sheaf Cohomology}

All of our sheaves take values in abelian groups. Let $\cF$ be a
presheaf on $X$.

\begin{definition}
  Recall the definition of a {\bf presheaf} $\cF$ on $X$:
  \begin{enumerate}
  \item (presheaf) every open set $U$ in $X$ is assigned an abelian
    group $\cF(U)$, such that if $V \subseteq U$ are two open sets,
    there is a restriction map $-|_{U \to V}\colon \cF(U) \to \cF(V)$
    compatible with inclusion, i.e. $(-|_{U \to V})|_{V \to W} = -|_{U
      \to W}$ for any $W \subseteq V \subseteq U$.
  \end{enumerate}
  If in addition $\cF$ satisfies the following two properties, it is a
  {\bf sheaf}:
  \begin{enumerate}
  \setcounter{enumi}{1}
  \item (locality) if $\{U_\alpha\}$ is an open cover of $X$ and $f, g
    \in \cF(X)$ such that $f|_{U \to U_\alpha} = g|_{U \to U_\alpha}$
    for every $U_\alpha$, then $f = g$;
  \item (gluing) if $\{U_\alpha\}$ is an open cover of $X$ and
    $f_\alpha \in \cF(U_\alpha)$ for every $U_\alpha$ are elements
    agreeing on overlaps, i.e. such that $f_\alpha|_{U_\alpha \to
      U_\alpha \cap U_\beta} = f_\beta|_{U_\beta \to U_\alpha \cap
      U_\beta}$, then we can glue the $f_\alpha$ together to get $f
    \in \cF(X)$, i.e. $f|_{X \to U_\alpha} = f_\alpha$ for every
    $U_\alpha$.
  \end{enumerate}
\end{definition}

\begin{definition}
  Let $\cU = \{U_\alpha\}$ be an {\bf ordered open cover} of $X$, i.e.
  with a partial order such that if $\alpha$ and $\beta$ are
  incomparable then $U_\alpha \cap U_\beta$ is empty. A {\bf
    $p$-simplex} $\sigma$ of $\cU$ is a totally ordered collection of
  open sets $U_{\alpha_0}, \ldots, U_{\alpha_p} \in \cU$; we call
  $U_{\alpha_0, \ldots, \alpha_p} \coloneqq U_{\alpha_0} \cap \cdots
  \cap U_{\alpha_p}$ its {\bf support}, and often refer to $\sigma$ by
  it instead. The {\bf $k$-th boundary component} of a $p$-simplex
  $U_{\alpha_0, \ldots, \alpha_p}$ is given by $\di_k U_{\alpha_0,
    \ldots, \alpha_p} \coloneqq U_{\alpha_0, \ldots, \hat{\alpha}_k,
    \ldots, \alpha_p}$. Cochains are maps from simplices to sheaf
  sections, and form a cochain complex:
  \[ C^p(\cU, \cF) \coloneqq \prod_{\alpha_0 < \cdots < \alpha_p} \cF(U_{\alpha_0, \ldots, \alpha_p}), \quad (\delta^p \omega)(\sigma) \coloneqq \sum_{k=0}^{p+1} (-1)^k \omega(\di_k \sigma)|_{\sigma} \colon C^p(\cU, \cF) \to C^{p+1}(\cU, \cF). \]
  The {\bf \v Cech cohomology of $\cU$} with coefficients in $\cF$,
  denoted $\chH^\bullet(\cU, \cF)$, is the cohomology of this complex.
\end{definition}

\begin{example}
  Let $\cF$ be a sheaf. By the gluing condition for a sheaf, a global
  section $f \in \cF(X)$ is defined by its values $f_\alpha \coloneqq
  f|_{X \to U_\alpha} \in \cF(U_\alpha)$ on every $U_\alpha$ in an
  open cover. These $f_\alpha$ form precisely the data for an element
  of $C^0(\cU, \cF)$, and satisfy the gluing condition $f_\alpha =
  f_\beta$ on $U_\alpha \cap U_\beta$, which is precisely the
  statement $\delta_0 f = 0$. Hence $\chH^0(\cU, \cF) = \cF(X)$ for a
  sheaf $\cF$.
\end{example}

\begin{example}
  Let $\cO$ denote the sheaf of holomorphic functions (and $\cO^*$ the
  nowhere-zero ones) on $\bP^1$. Recall that on $\bP^1$ we have the
  charts $U = \bP^1 \setminus \{S\}$ and $V = \bP^1 \setminus \{N\}$,
  with coordinates $u$ and $v$ respectively. To look at sections of
  $\cO(U)$ versus $\cO(V)$, we use the transition map $v = u^{-1}$ on
  $U \cap V$. The cochains for this open cover are
  \[ C^0(\cU, \cO) = \cO(U) \times \cO(V), \quad C^1(\cU, \cO) = \cO(U \cap V), \quad C^k(\cU, \cO) = 0 \; \forall k \ge 2. \]
  We compute the sheaf cohomology.
  \begin{enumerate}
  \item ($\chH^0(\cU, \cO)$) The boundary map $\delta_0$ maps $(f, g)
    \in C^0(\cU, \cO)$ to $g - f$. But
    \[ f = \sum_{k=0}^\infty f_k u^k, \quad g = \sum_{k=0}^\infty g_k v^k = \sum_{k=0}^\infty g_k u^{-k}, \]
    so $g - f = 0$ iff $f_k = g_k = 0$ for all $k > 0$, and $f_0 =
    g_0$. Hence $\chH^0(\cU, \cO) \cong \bC$, consisting of all
    constant functions.
  \item ($\chH^1(\cU, \cO)$) Given $h \in C^1(\cU, \cO)$, rewrite its
    Laurent expansion:
    \[ h = \sum_{k=-\infty}^\infty h_k u^k = \sum_{k=0}^\infty h_k u^k + \sum_{k=1}^{\infty} h_k v^k = -f + g \]
    where $f \in \cO(U)$ and $g \in \cO(V)$. Hence $h \in \im
    \delta_0$, and $\chH^1(\cU, \cO) = 0$.
  \item ($\chH^k(\cU, \cO)$) Trivially, $\chH^k(\cU, \cO) = 0$ for $k
    \ge 2$.
  \end{enumerate}
  Note that $\chH^0(\cU, \cO) \cong \bC$ is consistent with what we
  know so far, since $\chH^0(\cU, \cO) = \cO(\bP^1)$, and Liouville's
  theorem shows that $\cO(\bP^1)$ can only contain constant functions.
\end{example}

\begin{example}
  Recall the tautological line bundle $\cO(-1)$ and its dual $\cO(1)$
  on $\bP^n$; we have $\cO(n) = \cO(1)^n$. On the same charts on
  $\bP^1$, since $\cO(1)$ has transition function $u = v^{-1}$, we
  know $\cO(n)$ has transition function $u^n = v^{-n}$. To construct a
  global section of $\cO(n)$, given a monomial $v^k$ on $V$, we
  require $u^n v^k = u^{n-k}$ to be well-defined on $U$, so $k \le n$.
  In homogeneous coordinates $[x_0 : x_1]$, the global sections are
  therefore $x_0^n, x_0^{n-1}x_1, \ldots, x_1^n$, the homogeneous
  polynomials of degree $n$. The same story holds on $\bP^N$. Hence
  $\dim H^0(\bP^N, \cO(n)) = \binom{N+n-1}{n-1}$. In particular, there
  are $\binom{9}{5} = 126$ independent global sections of
  $\cO_{\bP^4}(5)$.
\end{example}

\begin{definition}
  The set of all open covers of $X$ form a directed set under
  refinement. The {\bf \v Cech cohomology of $X$} with coefficients in
  $\cF$ is the direct limit $\chH^n(X, \cF) \coloneqq \varinjlim_{\cU}
  \chH^n(\cU, \cF)$.
\end{definition}

\begin{definition}
  An ordered open cover $\{U_\alpha\}$ is {\bf good} if it is
  countable and every finite intersection $U_{\alpha_0, \ldots,
    \alpha_p}$ is either empty or contractible.
\end{definition}

\begin{theorem}[{\cite[Corollary of Theorem 5.4.1]{Godement1997}}]
  The \v Cech cohomology of a good cover $\cU$ is isomorphic to the \v
  Cech cohomology of $X$.
\end{theorem}

One can define {\bf sheaf cohomology} $H^n(X, \cF)$ as the right
derived functors of the global sections functor $\Gamma_X$ (i.e. $\cF
\mapsto \cF(X)$). For us, \v Cech and sheaf cohomology are
indistinguishable as long as we work with sheaves.

\begin{theorem}[{\cite[Theorem 5.10.1]{Godement1997}}]
  If $X$ is a paracompact topological space, then \v Cech cohomology
  $\chH^n(X, \cF)$ and sheaf cohomology $H^n(X, \cF)$ are isomorphic
  for any sheaf $\cF$.
\end{theorem}

\v Cech cohomology is also directly related to de Rham cohomology,
and, as we shall see, Dolbeault cohomology in the complex case. So we
can think of \v Cech cohomology classes as forms.

\begin{theorem}[\v Cech--de Rham isomorphism]
  Let $\bR$ denote the constant sheaf. There is a canonical
  isomorphism $\chH^k(X, \bR) \cong H_{\dR}^k(X)$ for each $k$.
\end{theorem}

\begin{proof}
  By the Poincar\'e lemma, the {\bf de Rham complex} of sheaves
  \[ 0 \to \bR \xrightarrow{\subset} \Omega^0(X) \xrightarrow{d} \Omega^1(X) \xrightarrow{d} \Omega^2(X) \xrightarrow{d} \cdots \]
  is exact (by checking exactness on the stalks). Let $Z^n \coloneqq
  \ker(d\colon \Omega^k \to \Omega^{k+1})$. The de Rham complex splits
  into a bunch of short exact sequences:
  \[ 0 \to d\Omega^{k-1} \cong Z^k \xrightarrow{\subset} \Omega^k \xrightarrow{d} Z^{k+1} \to 0. \]
  To each such short exact sequence is associated a long exact
  sequence of (sheaf) cohomology:
  \[ 0 \to H^0(X, Z^k) \to H^0(X, \Omega^k) \to H^0(X, Z^{k+1}) \to H^1(X, Z^k) \to H^1(X, \Omega^k) \to H^1(X, Z^{k+1}) \to \cdots. \]
  Fact: $H^i(X, \Omega^k) = 0$ for every $k$ and $i > 0$ (since
  $\Omega^k$ is a fine sheaf). Hence we obtain isomorphisms
  \[ H^{i+1}(X, Z^0) \cong H^i(X, Z^1) \cong \cdots \cong H^1(X, Z^i). \]
  But $Z^0 \cong \bR$ and we more commonly write
  \[ H^1(X, Z^i) = \coker(H^0(X, \Omega^i) \to H^0(X, Z^{i+1})) = Z^{i+1}(X)/d\Omega^k(X) = H^{i+1}_{\dR}(X). \]
  Hence $\chH^{i+1}(X, \bR) \cong H^{i+1}_{\dR}(X)$.
\end{proof}

\begin{definition}
  Let $E$ be a holomorphic vector bundle on $X$ and $\Omega^{0,q}(E)
  \coloneqq \Omega^{0,q}(X) \otimes \Gamma(E)$ denote the space of
  $E$-valued $(0,q)$-forms. {\bf Dolbeault cohomology} $H^q_{\bdi}(E)$
  is the cohomology of the complex
  \[ \cdots \xrightarrow{\bdi} \Omega^{0,q}(E) \xrightarrow{\bdi} \Omega^{0,q+1}(E) \xrightarrow{\bdi} \Omega^{0,q+2}(E) \xrightarrow{\bdi} \cdots. \]
  We write $H^{p,q}_{\bdi}(X)$ for $E = \Lambda^pT^{1,0}X$. The
  dimensions $h^{p,q}(X) \coloneqq \dim_{\bC} H^{p,q}_{\bdi}(X)$ are
  the {\bf Hodge numbers} of $X$.
\end{definition}

\begin{lemma}[$\bdi$-Poincar\'e lemma]
  $\bdi$-closed, i.e. holomorphic, implies $\bdi$-exact on $\bC^n$.
\end{lemma}

\begin{theorem}[\v Cech--Dolbeault isomorphism] \label{thm:cech-dolbeault}
  Let $\Omega^{p,0}$ denote the sheaf of holomorphic $p$-forms on $X$.
  There is a natural isomorphism $\chH^q(X, \Omega^{p,0}) \cong
  H^{p,q}_{\bdi}(X)$.
\end{theorem}

\begin{proof}
  Analogous to the proof of the \v Cech--de Rham isomorphism, except
  now using the $\bdi$-Poincar\'e lemma to establish the exactness of
  the complex
  \[ 0 \to \ker(\bdi\colon\Omega^{p,0}(X) \to \Omega^{p,1}(X)) \xrightarrow{\subset} \Omega^{p,0}(X) \xrightarrow{\bdi} \Omega^{p,1}(X) \xrightarrow{\bdi} \cdots. \qedhere \]
\end{proof}

\begin{definition}
  Consider the double complex $\Omega^{\bullet,\bullet}$ with
  differentials $\di$ and $\bdi$. The {\bf Fr\"olicher spectral
    sequence} is the spectral sequence of a double complex associated
  to $\Omega^{\bullet,\bullet}$. Since the total complex of
  $\Omega^{\bullet,\bullet}$ is $\Omega^\bullet(X)$, the Fr\"olicher
  spectral sequence converges to complex de Rham cohomology
  $H^\bullet_{\dR}(X, \bC)$.
\end{definition}

\subsection{Morse Homology}

Throughout this subsection, $f\colon X \to \bR$ is a smooth function,
and we equip $X$, viewed as a real manifold, with a Riemannian metric
$g$. We also assume $(f, g)$ is Morse--Smale, defined below.

\begin{definition}
  A {\bf critical point} of $f$ is a point $p \in X$ with $df_p = 0$.
  Define the {\bf Hessian}
  \[ H(f)_p\colon T_pX \to T_p^*X, \quad v \mapsto \nabla_v(df), \]
  which is independent of the choice of connection $\nabla$. (In
  coordinates, we recover the usual $\di^2f/\di x_i \di x_j$.) The
  critical point $p$ is {\bf non-degenerate} if the Hessian does not
  have zero eigenvalues. A non-degenerate critical point $p$ has {\bf
    Morse index} $\ind(p)$ the number of negative eigenvalues of the
  Hessian. The function $f$ is {\bf Morse} if all of its critical
  points are non-degenerate.
\end{definition}

\begin{definition}
  Recall that the {\bf gradient} of $f$ with respect to a metric $g$
  is the vector field $\grad f$ such that $g(\grad f, X) = Xf$.
  Equivalently, $\grad f = (df)^\sharp$. Let $\psi_t\colon X \to X$ be
  the one-parameter group of diffeomorphisms associated to the flow of
  $-\grad f$. The {\bf descending manifold} $D(p)$ and {\bf ascending
    manifold} $A(p)$ at a critical point $p$ are
  \begin{align*}
    D(p) &\coloneqq \{x \in X : \lim_{t \to -\infty} \psi_t(x) = p\} \\
    A(p) &\coloneqq \{x \in X : \lim_{t \to +\infty} \psi_t(x) = p\}.
  \end{align*}
  The pair $(f, g)$ is {\bf Morse--Smale} if $f$ is Morse and $D(p)$
  is transverse to $A(q)$ for every pair of critical points $p$ and
  $q$ (i.e. tangent spaces of $D(p)$ and $A(q)$ generate the tangent
  space at every intersection point).
\end{definition}

Here are two useful and easy-to-prove facts: every flow line
asymptotically approaches critical points, and $\dim D(p) = \ind(p)$
(so $\dim A(p) = \dim X - \ind p$ by the Morse--Smale condition).

\begin{definition}
  Fix critical points $p$ and $q$. A {\bf flow line} from $p$ to $q$
  is an integral curve $\gamma(t)$ of $-\grad f$ with $\lim_{t \to
    -\infty} \gamma(t) = p$ and $\lim_{t \to +\infty} \gamma(t) = q$.
  The {\bf moduli space of flow lines} from $p$ to $q$ is
  \begin{align*}
    \cM(p, q) &\coloneqq \{\text{flow lines from } p \text{ to } q\}/\sim,
    \quad \alpha \sim \beta \text{ if } \exists c \in \bR : \alpha(t) = \beta(t + c) \\
    &= (D(p) \cap A(q)) / \bR.
  \end{align*}
  A {\bf broken flow line} consists, piecewise, of flow lines.
\end{definition}

The Morse--Smale condition implies $D(p) \cap A(q)$ is a submanifold
of $X$ with dimension $\ind(p) - \ind(q)$. Since $\sim$ is a smooth,
proper, free $\bR$-action, $\cM(p, q)$ is a manifold of dimension
$\ind(p) - \ind(q) - 1$ when $p \neq q$ (otherwise the $\bR$-action is
trivial). Note that if $\ind(p) = k$ and $\ind(q) = k-1$ then $\cM(p,
q)$ is zero-dimensional. In fact, in this case, $\cM(p, q)$ is compact
as a corollary of the following theorem, and therefore is a finite set
of points.

\begin{theorem}[{\cite[Theorem 2.1]{Hutchings2012}}]
  Let $X$ be closed and $(f, g)$ Morse--Smale. Then $\cM(p, q)$ has a
  natural compactification to a smooth manifold with corners
  $\overline{\cM(p, q)}$ where
  \[ \overline{\cM(p, q)} \setminus \cM(p, q) = \bigcup_{k \ge 1} \bigcup_{\substack{p,r_1,\ldots,r_k,q\\\text{distinct crit pts}}} \cM(p, r_1) \times \cM(r_1, r_2) \times \cdots \times \cM(r_{k-1}, r_k) \times \cM(r_k, q). \]
\end{theorem}

\begin{corollary}
  If $\ind(p) - \ind(q) = 1$, then $\overline{\cM(p, q)} = \cM(p, q)$
  is compact. If $\ind(p) - \ind(q) = 2$, then
  \[ \di \overline{\cM(p, q)} = \bigcup_{\ind(r) = \ind(p)-1} \cM(p, r) \times \cM(r, q). \]
\end{corollary}

\begin{proof}
  Since $\dim \cM(r, s) = \ind(r) - \ind(s) - 1$, the space $\cM(r,
  s)$ is non-empty only if $\ind(r) - \ind(s) \ge 1$. Hence
  $\overline{\cM(p, q)}\setminus \cM(p, q) = \emptyset$ when $\ind(p)
  - \ind(q) = 1$. Similar reasoning shows the $\ind(p) - \ind(q) = 2$
  case.
\end{proof}

\begin{definition}
  Fix orientations for $D(p)$ at every critical point $p$. There is an
  isomorphism at $x \in \gamma \in \cM(p, q)$ given by
  \begin{align*}
    T_xD(p) &\cong T_x(D(p) \cap A(q)) \oplus (T_xX/T_xA(q)) && \text{transversality from Morse--Smale} \\
    &\cong T_\gamma \cM(p, q) \oplus T_x\gamma \oplus (T_xX/T_xA(q)) && \text{definition of $\cM(p, q)$} \\
    &\cong T_\gamma \cM(p, q) \oplus T_x\gamma \oplus T_qD(q) && \text{translating $T_qD(q)$ along $\gamma$.}
  \end{align*}
  The {\bf orientation} on $\cM(p, q)$ is such that this isomorphism
  is orientation-preserving. Let $C_k$ be the free abelian group
  generated by critical points of index $k$, and define the {\bf
    Morse--Smale--Witten boundary map}
  \[ \di_k^{\Morse}\colon C_k \to C_{k-1}, \quad p \mapsto \sum_{\ind q = k-1} \# \cM(p, q) q \]
  where $\# \cM(p, q) \in \bZ$ is counted with sign according to the
  orientation of $\cM(p, q)$, which here is a discrete set of points.
\end{definition}

\begin{lemma}
  $(\di_k^\Morse)^2 = 0$, so $(C_\bullet, \di^\Morse)$ is a chain
  complex.
\end{lemma}

\begin{proof}
  Let $\ind(p) - \ind(q) = 2$. The coefficient of $q$ in
  $(\di^\Morse)^2p$ is
  \[ \sum_{\ind(r) = \ind(p)-1} \# \cM(p, r) \cdot \# \cM(r, q) = \# \bigcup_{\ind(r) = \ind(p)-1} \cM(p, r) \times \cM(r, q) = \# \di \overline{\cM(p, q)}. \]
  Since $\overline{\cM(p, q)}$ is an oriented $1$-manifold with
  boundary, this quantity, the number of boundary points, is
  zero.
\end{proof}

\begin{definition}
  {\bf Morse homology} $H_\bullet^\Morse(f, g)$ is the homology of the
  {\bf Morse--Smale--Witten complex} $(C_\bullet, \di^{\Morse})$.
\end{definition}

\begin{example}
  The (upright) torus $T^2$ has four critical points with $f$ the
  height function: $p$ (index 2), $q$ and $r$ (index 1), and $s$
  (index 0). This choice of $f$ is Morse, but with the induced metric
  $g$ from $\bR^3$, the pair $(f, g)$ is not Morse--Smale: $D(q) \cap
  A(r)$ is non-empty, but transversality forces it to be. The solution
  is to tilt the torus a little; equivalently, perturb $g$. There are
  two flow lines, of opposite sign, for each relevant pair of critical
  points. Hence $\di_k^\Morse = 0$ for $k = 1, 2$. It follows that
  \[ H_2^\Morse(f, g) = \bZ, \quad H_1^\Morse(f,g) = \bZ^2, \quad H_0^\Morse(f, g) = \bZ. \]
\end{example}

\begin{theorem}[{\cite[Theorem 3.1]{Hutchings2012}}]
  Let $X$ be a closed smooth manifold, $H_\bullet(X)$ denote singular
  homology on $X$, and $(f, g)$ be a Morse--Smale pair on $X$. Then
  there is a canonical isomorphism $H_n^\Morse(f, g) \cong H_n(X)$.
\end{theorem}

\begin{corollary}
  The number of critical points of a Morse function is at least the
  sum $\sum_k \dim H_k(X)$ of the Betti numbers.
\end{corollary}

\begin{proof}
  The number of critical points is the sum of the dimensions of the
  Morse chain groups, which is at least the sum of the dimensions of
  the Morse homology groups, which is equal to the sum of the
  dimensions of the singular homology groups.
\end{proof}

The infinite-dimensional analogue of Morse homology is known as {\bf
  Floer homology}. We shall primarily be concerned with Floer homology
for mirror symmetry.

\subsection{Equivariant Cohomology}



\section{Algebraic Topology}

We stop distinguishing between isomorphic cohomology theories now. In
particular, since $X$ is always at least a smooth manifold, we think
of singular cohomology $H^k(X)$ as de Rham cohomology. For a bundle
$E$, $H^k(E)$ refers to sheaf cohomology.

\subsection{Poincar\'e and Serre Duality}

Unless otherwise stated, $X$ in this section is a compact oriented
$n$-manifold.

\begin{theorem}[Poincar\'e duality, \cite{Bott1982}]
  Let $X$ be a compact oriented $n$-manifold. The map
  \[ \int_X\colon H^k(X) \otimes H^{n-k}(X) \to \bR, \quad \omega \otimes \eta \mapsto \int_X \omega \wedge \eta \]
  is a perfect pairing, and hence $H^k(X) \cong H^{n-k}(X)^*$.
\end{theorem}

If we relax the assumption that $X$ is compact, then the issue is that
$\int_X$ may not be well-defined. We work around this by using
de Rham cohomology with compact support.

\begin{definition}
  Let $\Omega_c^k(X)$ denote the $k$-forms on $X$ with compact
  support. The {\bf de Rham cohomology groups with compact support}
  $H^n_c(X)$ are the cohomology of the chain complex
  $(\Omega_c^\bullet(X), d)$.
\end{definition}

\begin{theorem}[Poincar\'e duality for non-compact manifolds, \cite{Bott1982}]
  Let $X$ be an oriented $n$-manifold without boundary. The map
  \[ \int_X\colon H^k(X) \otimes H^{n-k}_c(X) \to \bR, \quad \omega \otimes \eta \mapsto \int_X \omega \wedge \eta \]
  is a perfect pairing, and hence $H^k(X) \cong H^{n-k}_c(X)^*$.
\end{theorem}

\begin{definition}
  Fix $C \subset X$ a closed $(n-k)$-submanifold. Then Poincar\'e
  duality identifies the map $\int_C\colon H^{n-k}(X) \to \bR$
  with a $k$-form $\eta_C \in H^k(X)$, called the {\bf
    Poincar\'e dual class}. Explicitly, $\int_C \omega = \int_X \omega
  \wedge \eta_C$.
\end{definition}

There is a relation between the Poincar\'e dual class and the Thom
class, which we define below. Namely, the Poincar\'e dual class of $C$
can be constructed as the Thom class of the normal bundle of $C$ in
$X$.

\begin{theorem}[{\cite[Theorem 10.4]{Milnor1974}}] \label{thm:thom}
  Let $\pi\colon E \to B$ be an oriented rank $n$ real vector bundle
  and $B$ is embedded into $E$ as the zero section. Then
  \begin{enumerate}
  \item there exists a unique cohomology class $\Phi \in H^n(E, E
    \setminus B)$ called the {\bf Thom class} such that for every $x
    \in B$, the restriction of $\Phi$ to $H^n(E_x, E_x \setminus
    \{0\})$ is the preferred generator specified by the orientation of
    $E_x$ in $E$;
  \item the {\bf Thom isomorphism} $\colon H^k(E) \to H^{k+n}(E, E
    \setminus B)$, given by $\omega \mapsto \omega \wedge \Phi$, is an
    isomorphism for every $k$.
  \end{enumerate}
\end{theorem}

Note that since $B$ is a deformation retract of $E$, the rings
$H^*(E)$ and $H^*(B)$ are isomorphic. Hence $\pi^*\Phi = 1 \in H^*(B)$,
which shall be very important in the upcoming proof.

\begin{theorem}[Tubular neighborhood theorem, {\cite[Theorem 11.1]{Milnor1974}}]
  Let $C \subset X$ be a $k$-submanifold embedded in $X$. There exists
  an open neighborhood, called a {\bf tubular neighborhood}, of $C$ in
  $X$ diffeomorphic to the total space of the normal bundle of $C$.
  This diffeomorphism maps points in $C$ to zero vectors.
\end{theorem}

\begin{proposition}[{\cite[Proposition 6.24a]{Bott1982}}] \label{thm:thom-normal}
  Let $C \subset X$ be a closed $(n-k)$-submanifold. The Poincar\'e
  dual class $\eta_C \in H^k(X)$ of $C$ is the Thom class of the
  normal bundle of $C$ in $X$.
\end{proposition}

\begin{proof}
  Let $NC$ denote the normal bundle of $C$ in $X$, which has rank $k$
  because $C$ is codimension $k$. Use the tubular neighborhood theorem
  to identify $NC$ with an open neighborhood $T$ of $C$ in $X$, and
  then extend by zero to get $\Phi \in H^k(X)$ supported on $T$.

  We shall show that $\int_X \omega \wedge \Phi = \int_C \omega$ for
  any $\omega \in H^{n-k}_c(X)$. The maps $\pi\colon T \to C$ and
  $\iota\colon C \to T$ induce isomorphisms of cohomology, so on forms
  $\omega$ and $\pi^*\iota^*\omega$ differ by at most an exact form
  $d\tau$. Then
  \begin{align*}
    \int_X \omega \wedge \Phi
    &= \int_T \omega \wedge \Phi = \int_T (\pi^*\iota^*\omega + d\tau) \wedge \Phi \\
    &= \int_T \pi^*\iota^*\omega \wedge \Phi = \int_C \iota^*\omega \wedge \pi^* \Phi = \int_C \iota^*\omega. \qedhere
  \end{align*}
\end{proof}

\begin{corollary} \label{thm:wedge-dual}
  Transverse intersection is Poincar\'e dual to the wedge product,
  i.e. for $C, D \subset X$ closed submanifolds intersecting
  transversally, $\eta_{C \cap D} = \eta_C \wedge \eta_D$.
\end{corollary}

\begin{proof}
  For transversal intersections, codimension is additive: $\codim C
  \cap D = \codim C + \codim D$. So the normal bundle of the
  intersection is $N(C \cap D) = NC \oplus ND$. Let $\Phi(E)$ denote
  the Thom class associated to the vector bundle $E$. By the
  characterization of the Thom class, for vector bundles $E$ and $F$
  we have $\Phi(E \oplus F) = \Phi(E) \wedge \Phi(F)$; check that
  $\Phi(E) \oplus \Phi(F)$ restricts on each fiber to the preferred
  generator. Hence
  \[ \eta_{C \cap D} = \Phi(N_{C \cap D}) = \Phi(NC \oplus ND) = \Phi(NC) \wedge \Phi(ND) = \eta_C \wedge \eta_D. \qedhere \]
\end{proof}

Let $X$ be a complex $n$-fold now. In the complex setting, we can
refine Poincar\'e duality. The \v Cech--Dolbeault isomorphism
\ref{thm:cech-dolbeault} works for the more general setting in which
we defined Dolbeault cohomology: if $E$ is a holomorphic vector bundle
over $X$, then $H^k(X, E) \cong H^k_{\bdi}(E)$. So we think of \v Cech
cohomology classes $H^k(X, E)$ as $E$-valued $(0,k)$-forms.

\begin{definition}
  The {\bf canonical bundle} $K_X$ of a complex $n$-fold $X$ is the
  vector bundle of $(n, 0)$-forms. (Also commonly denoted
  $\Omega^n(X)$.)
\end{definition}

Hence $H^{n-k}(X, E^* \otimes K_X)$ consists of $E^*$-valued $(n,
n-k)$-forms. Given such a form $\omega$ and another form $\eta \in
H^k(X, E)$, the form $\omega \wedge \eta$ is an $(n, n)$-form with
complex coefficients. We can integrate it to get something in $\bC$.
At this point it is impossible not to wonder about whether the pairing
$H^k(E) \otimes H^{n-k}(X, E^* \otimes K_X) \to \bC$ given by wedging
and then integrating is perfect.

\begin{theorem}[Serre duality, {\cite[Corollary III.7.13]{Hartshorne1997}}]
  The pairing $H^k(X, E) \otimes H^{n-k}(X, E^* \otimes K_X) \to \bC$
  is perfect, so $H^k(X, E) \cong H^{n-k}(X, E^* \otimes K_X)^*$.
\end{theorem}

Poincar\'e duality combined with Hodge decomposition gives
\[ \bigoplus_{p+q=k} H^q(X, \Omega^p) = H^k(X, \bC) \cong H^{2n-k}(X, \bC) = \bigoplus_{p'+q'=2n-k} H^{q'}(X, \Omega^{p'}) = \bigoplus_{p+q=k} H^{n-q}(X, \Omega^{n-p}). \]
Serre duality says that in fact, each of the terms in the sum are
isomorphic: $H^q(X, \Omega^p) \cong H^{n-q}(X, \Omega^{n-p})$.

\subsection{Chern Classes via Chern--Weil Theory}

For this subsection, we work over $\bC$, and every vector bundle we
consider is smooth and complex. We define Chern classes using the
Chern--Weil approach. There are other equivalent approaches in more
general settings. But for us, we take $\pi\colon E \to X$ to be a
rank-$n$ smooth complex vector bundle over a smooth manifold $X$. A
connection $A \in \Omega^1(X, \Ad E)$ on $E$ gives a curvature $F_A
\coloneqq dA + A \wedge A \in \Omega^2(X, \Ad E)$.

\begin{definition}
  The {\bf total Chern class} of $E$ is
  \begin{align*}
    c(E) \coloneqq \det\left(1 + \frac{i}{2\pi}F\right)
    &= 1 + \frac{i}{2\pi} \tr(F) + \frac{1}{8\pi}(\tr(F^2) - \tr(F)^2) + \cdots \\
    &= 1 + c_1(E) + c_2(E) + \cdots \in H^0(X, \bR) \oplus H^2(X, \bR) \oplus \cdots.
  \end{align*}
  Its terms $c_k(E) \in H^{2k}(X, \bR)$ are the {\bf Chern classes}.
  The total Chern class $c(X)$ of $X$ is defined as $c(X) \coloneqq
  c(T^{1,0}X)$.
\end{definition}

\begin{theorem}[Chern--Weil theorem, {\cite[Corollary 4.4.5, Lemma 4.4.6]{Huybrechts2005}}]
  The total Chern class $c(E)$ is closed and independent of the choice
  of connection $A$ on $E$.
\end{theorem}

\begin{example}
  A magnetic monopole at the origin in $U(1)$ Maxwell theory is given
  by the trivial line bundle on $\bR^3$ with connection $A =
  i\frac{1}{2r} \frac{1}{z-r} (x dy - y dx)$, where $r$ is the
  coordinate on the fibers of the bundle. Then
  \[ F_A = i\frac{1}{2r^3} (x dy \wedge dz + y dz \wedge dx + z dx \wedge dy) = -\frac{i}{2r^2}(r^2 \sin \theta d\theta \wedge d\phi). \]
  We easily check that $\int_{S^2} c_1 = \frac{i}{4\pi} \int_{S^2} F_A
  = 1$ for any $2$-sphere around the origin.
\end{example}

\begin{theorem} \label{thm:chern-axioms}
  The Chern classes satisfy and are uniquely determined by the
  following properties:
  \begin{enumerate}
  \item $c_0(E) = 1$ and $c_k(E) = 0$ if $k > \dim E$;
  \item (Naturality) if $f\colon Y \to X$ is continuous, then $f^*c(E)
    = c(f^*E)$;
  \item (Whitney product formula) $c(E \oplus F) = c(E) \wedge c(F)$;
  \item $c_1(\cO_{\bP^1}(-1))$ is minus the preferred generator (given
    by the orientation) of $H^2(\bP^1)$.
  \end{enumerate}
\end{theorem}

\begin{proof}
  Property 1 is clear from the definition of $c_k(E)$. Property 2
  follows from the multiplicative property of the determinant for
  block diagonal matrices; the curvature on $E \oplus F$ splits as a
  curvature on $E$ and a curvature on $F$. Property 3 follows from
  pulling back a connection $A$ on $E$ to a connection $f^*A$ on
  $f^*E$, and then using that pullbacks commute with everything.

  Property 4 is more tedious and serves as our second explicit
  calculation of a Chern class. On $\bP^1$, take the usual charts $(U,
  u)$ and $(V, v)$ with $u = v^{-1}$ on $U \cap V$. The local
  $1$-forms
  \[ A_U = \frac{\bar{u} du}{1 + u\bar{u}}, \quad A_V = \frac{\bar{v} dv}{1 + v \bar{v}} \]
  form the globally-defined {\bf Chern connection} on
  $\cO_{\bP^1}(-1)$, the tautological bundle. We cheat a little and
  work only over $U$ instead of all of $\bP^1$. The curvature is
  \[ F_{A_U} = \frac{(1 + u\bar{u}) d\bar{u} \wedge du - \bar{u} (\bar{u} du + u d\bar{u}) \wedge du}{(1 + u\bar{u})^2} = -\frac{du \wedge d\bar{u}}{(1 + u\bar{u})^2}. \]
  Hence, in real coordinates, $c_1(\cO_{\bP^1}(-1)) = -\frac{dx \wedge
    dy}{\pi(1 + x^2 + y^2)^2}$. To compare this with the preferred
  generator, we simply integrate both and compare the result. (This is
  valid since $\dim H^2(\bP^1) = 1$.) We know the preferred generator
  integrates to $1$, whereas
  \[ \int_{\bP^1} c_1(\cO_{\bP^1}(-1)) = -\frac{1}{\pi} \int_{-\infty}^\infty \int_{-\infty}^\infty \frac{dx \wedge dy}{(1 + x^2 + y^2)^2} = -\frac{1}{\pi} \int_0^\infty \int_0^{2\pi} \frac{r d\theta \wedge dr}{(1 + r^2)^2} = -1. \qedhere \]
\end{proof}

\begin{proposition}[Splitting principle]
  If $0 \to A \to B \to C \to 0$ is a short exact sequence, then $c(B)
  = c(A) \wedge c(C)$.
\end{proposition}

\begin{proof}
  Short exact sequences of smooth vector bundles always split: pick a
  metric on $B$ and show that $C \cong A^\perp$. Hence $c(B) = c(A
  \oplus C)$, and then we use the Whitney product formula.
\end{proof}

Note: this is {\bf not} the usual ``splitting principle''. The usual
splitting principle says that to prove an identity on Chern classes,
it suffices to pretend that the bundle completely splits into line
bundles and prove the identity for that case. For more detail, see
\cite[Section 21]{Bott1982}.

\begin{example} \label{ex:chern-proj}
  We compute the total Chern class of $\bP^n$. We first construct the
  {\bf Euler sequence} on $\bP^n$, given by
  \[ 0 \to \bC \to \cO_{\bP^n}(1)^{n+1} \to T^{1,0}\bP^n \to 0. \]
  Since $X = \bP^n$ is $Y = \bC^{n+1} \setminus \{0\}$ mod a $\bC^*$
  action, given $n+1$ linear functionals $v_i$ on $\bC^{n+1}$, the
  vector field $\sum_i v_i \di_i$ on $X$ is invariant under this
  $\bC^*$ action and descends to a vector field on $\bP^n$. The $v_i$
  are sections of $\cO_{\bP^n}(1)$, so this construction is the map
  $\cO_{\bP^n}(1)^{n+1} \to T^{1,0}\bP^n$. Its kernel is the line
  bundle associated to $Z = \sum_i x_i \di_i$ (here $x_i$ are the
  coordinates on $Y$): for homogeneous polynomials $f$, we have
  $\frac{1}{d} \sum_i x_i \di_i f = f$. Another way to see this is to
  visualize $\bP^n$ as a sphere in $Y$, so when we project, the radial
  vector field $Z$ and its multiples are precisely the kernel.

  Clearly $c(\bC) = 1$, so by the splitting principle, $c(\bP^n) =
  c(\cO_{\bP^n}(1)^{n+1}) = c(\cO_{\bP^n}(1))^{n+1}$. Let $x =
  c_1(\cO_{\bP^n}(1))$. Then $c(\bP^n) = (1 + x)^{n+1}$.
\end{example}

Using the symbol $x$ to stand for $c_1(\cO_{\bP^n}(1))$ is fairly
common. We shall do so from now on. (The reason is that $x$ generates
the cohomology ring $H^*(\bP^n)$.)

\begin{example} \label{ex:chern-hypersurface}
  Let $X = V(p)$ be a smooth projective variety in $\bP^n$ with $p$ a
  degree $d$ homogeneous polynomial, i.e. a section of
  $\cO_{\bP^n}(d)$. To compute the Chern class of $X$, we use the {\bf
    adjunction formula} $NX \cong \cO(d)|_X$ (see \cite[Proposition
    2.2.17]{Huybrechts2005} for details), so that
  \[ 0 \to TX \to T\bP^n|_X \to NX \cong \cO(d)|_X \to 0 \]
  is a short exact sequence. Since $\cO(d) = \cO(1)^{\otimes d}$, we
  can't use the Whitney sum property of the Chern class, but we can
  use the Chern character:
  \[ \ch(\cO(d)) = \ch(\cO(1))^d = \exp(x)^d = \exp(dx), \]
  so $c(\cO(d)) = 1 + dx$. Hence $c(X) = (1 + x)^{n+1}/(1 + dx)$.
\end{example}

\begin{definition}
  If we formally factorize the total Chern class as $c(E) =
  \prod_{k=1}^r (1 + a_k)$, then the {\bf Chern character class} is
  \[ \ch(E) \coloneqq \sum_{k=1}^r \exp(a_i) = r + c_1(E) + \frac{1}{2}(c_1(E)^2 - 2c_2(E)) + \cdots. \]
\end{definition}

\begin{proposition}
  In the Chern--Weil setting, $\ch(E) = \tr \exp(iF/2\pi)$.
\end{proposition}

Of course, the Chern character class, being a combination of Chern
classes, does not contain more information than the Chern class. The
reason we work with it instead of the Chern class is the following
proposition.

\begin{proposition}
  The Chern character satisfies
  \[ \ch(E \oplus F) = \ch(E) + \ch(F), \quad \ch(E \otimes F) = \ch(E) \wedge \ch(F). \]
\end{proposition}

Finally, we need to connect the theory of Chern classes with sheaf
cohomology. Consider the short exact sequence of sheaves given by
\[ 0 \to \bZ \to \cO \xrightarrow{\exp} \cO^* \to 0. \]
Its associated long exact sequence of cohomology contains
\[ \cdots \to H^1(X, \bZ) \to H^1(X, \cO) \to H^1(X, \cO^*) \xrightarrow{\delta} H^2(X, \bZ) \to \cdots. \]

\begin{definition}
  The {\bf Picard group} of $X$ is the group of isomorphism classes of
  holomorphic line bundles under tensor product.
\end{definition}

\begin{theorem}[{\cite[Corollary 2.2.10]{Huybrechts2005}}]
  Let $\cO^*$ be the sheaf of nowhere-zero holomorphic functions. Then
  $\Pic(X) \cong H^1(X, \cO^*)$.
\end{theorem}

\begin{theorem}[{\cite[Proposition 4.4.12]{Huybrechts2005}}]
  Under the identification of elements of $H^1(X, \cO^*)$ with
  isomorphism classes of holomorphic line bundles, the connecting map
  $\delta\colon H^1(X, \cO^*) \to H^2(X, \bZ)$ is the first Chern
  class $c_1$.
\end{theorem}

\subsection{The Euler Class and Euler Characteristic}

Recall that the (holomorphic) Euler characteristic of a sheaf $\cF$ is
$\chi(\cF) \coloneqq \sum_k (-1)^k \dim H^k(X, \cF)$.

\begin{definition}
  Let $\pi\colon E \to B$ be an oriented rank $n$ vector bundle over a
  smooth $n$-fold $B$. In \ref{thm:thom} we defined its Thom class
  $\Phi \in H^n(E, E \setminus B)$. The inclusion $(E, \emptyset)
  \subset (E, E \setminus B)$ gives a homomorphism $H^k(E, E \setminus
  B) \to H^k(E)$ which we denote by $\omega \mapsto \omega|_E$. The
  {\bf Euler class} $e(E)$ of $E$ is the image of the Thom class
  $\Phi$ under the composition
  \[ H^n(E, E \setminus B) \xrightarrow{-|_E} H^n(E) \xrightarrow{(\pi^*)^{-1}} H^n(B) \]
  where the last isomorphism is canonical and comes from $B$ being a
  deformation retract of $E$. Again, if $X$ is a manifold, $e(X)
  \coloneqq e(TX)$.
\end{definition}

\begin{proposition} \label{thm:euler-chern}
  Whenever both are defined, $e(E) = c_n(E)$ for $E$ of rank $n$.
\end{proposition}

\begin{proof}[Proof sketch.]
  Given $E$ a smooth complex vector bundle, $e(E)$ is well-defined
  because the complex structure on $E$ induces an orientation. We can
  use the Euler class to construct the Chern classes \cite[Section
    14]{Milnor1974}. Then it suffices to verify that the Chern classes
  we constructed this way satisfy the four axioms
  \ref{thm:chern-axioms} of Chern classes. For example, the Whitney
  sum formula comes from $\Phi(E \oplus F) = \Phi(E) \wedge \Phi(F)$
  being preserved throughout the construction.
\end{proof}

We like to distinguish between the Euler class and the top Chern class
for several reasons. One is that the Euler class is topological,
whereas the Chern classes are differential geometric. Another is that
it is sometimes easier to prove properties of the Euler class using
properties of the Thom class rather than all the Chern classes,
especially from the Chern--Weil approach.

\begin{proposition}[{\cite[Property 9.3, Property 9.7]{Milnor1974}}]
  Properties of the Euler class $e(E)$ that do not directly follow
  from $e(E) = c_n(E)$:
  \begin{enumerate}
  \item if the orientation of $E$ is flipped, $e(E)$ changes sign;
  \item if $E$ has a nowhere zero global section, then $e(E) = 0$.
  \end{enumerate}
\end{proposition}

\begin{proof}
  Property 1 is obvious: flipping the orientation of $E$ flips the
  sign of the Thom class $\Phi$, since $\Phi|_{E_x}$ is the preferred
  generator. Property 2 comes from $B \xrightarrow{s} E \setminus B
  \subset E \xrightarrow{\pi} B$ being the identity for a non-zero
  global section $s$. Then
  \[ H^n(B) \xrightarrow{\pi^*} H^n(E) \to H^n(E \setminus B) \xrightarrow{s^*} H^n(B) \]
  is the identity on $H^n(B)$. But $\pi^*e(E) = \Phi|_E$, the
  restriction of the Thom class, by definition, so we have
  $s^*((\Phi|_E)|_{E \setminus B}) = e(E)$. Since $\Phi \in H^n(E, E
  \setminus B)$, the composition of these two restrictions is zero.
  Hence $e(E) = s^*0 = 0$.
\end{proof}

\begin{proposition} \label{thm:euler-dual}
  Let $E \to M$ be a smooth oriented real vector bundle of rank $r$
  over the smooth compact oriented manifold $M$ of dimension $n \ge
  r$. Let $Z$ be the zero set of a smooth section $s\colon M \to E$
  that is transversal to the zero section $\iota\colon M \to E$. Then
  $Z$ is a smooth submanifold of $M$ of codimension $r$ and there is a
  natural bundle isomorphism $T_ZM \cong E|_M$. Consequently, $e(E)$
  is Poincar\'e dual to $Z$.
\end{proposition}

\begin{proof}
  A straightforward exercise. Hint: remember that $e(E)$ is the
  restriction of the Thom class, and then use \ref{thm:thom-normal}
  and \ref{thm:wedge-dual}.
\end{proof}

The main purpose of this subsection is the following generalization of
the Gauss--Bonnet theorem. We shall use it extensively when
calculating Euler characteristic. To determine the Euler class
explicitly, we often use many of the preceding results identifying it
with various other objects.

\begin{theorem}[Generalized Gauss--Bonnet]
  Let $X$ be a compact complex manifold. Then
  \[ \int_X e(X) = \chi(X). \]
\end{theorem}

\begin{proof}
  We shall prove this in the next subsection, as a consequence of the
  Hirzebruch--Riemann--Roch formula. (There are much easier proofs,
  though.)
\end{proof}

\begin{example}
  We can continue \ref{ex:chern-proj} to compute the Euler
  characteristic of $\bP^n$. Note that every hyperplane $H \cong
  \bP^{n-1} \subset \bP^n$ is Poincar\'e dual to $x$. So $x^n$ is
  Poincar\'e dual to the intersection of $n$ generic hyperplanes,
  which is a point. In other words, $x^n$ is the preferred generator
  given by the orientation, and hence $\int_{\bP^n} x^n = 1$. (For a
  more explicit calculation of this, see \cite[Theorem
    14.10]{Milnor1974}. The explicit form for $x$ is the obvious
  generalization of $-c_1(\cO_{\bP^1}(-1))$, which we computed in
  \ref{thm:chern-axioms}.)

  Since $c(\bP^n) = (1 + x)^{n+1}$, we have $c_n = (n+1) x^n$. Hence
  $\int_{\bP^n} c_n = n+1$. By the generalized Gauss--Bonnet theorem,
  $\chi(\bP^n) = n+1$.
\end{example}

\begin{example}
  Recall that the {\bf degree} of a curve in $\bP^2$ is just the
  degree of the defining homogeneous polynomial. (For a more general
  definition of the degree of a variety, see \cite[Section
    I.7]{Hartshorne1997}.) Fact: A degree $d$ curve $X$ in $\bP^2$ has
  Chern class $1 + (3 - d)x$. (Remember we write $x$ for
  $c_1(\cO_{\bP^n}(1))$.) Then
  \[ \chi(X) = \int_X c_1(X) = \int_{\bP^2} c_1(X) (xd) = \int_{\bP^2} d(3-d)x^2 = d(3-d). \]
  But for nonsingular curves $X$, we have $\chi(X) = 2 - 2g$. Hence $g
  = (d-1)(d-2)/2 = \binom{d-1}{2}$.
\end{example}

\begin{example}
  By \ref{ex:chern-hypersurface}, a quintic hypersurface $Q$ in
  $\bP^4$ has total Chern class $c(Q) = (1 + x)^5/(1 + 5x) = 1 + 10x^2
  - 40x^3$. (Note that $c_1(Q) = 0$, so $Q$ is Calabi--Yau.) We want
  to compute its Euler characteristic using HRR, but integrating over
  $Q$ is hard. Instead, we use \ref{thm:euler-dual}: $Q$ is defined as
  the zero set of a section of $\cO(5) \to \bP^4$, so $e(\cO(5))$ is
  Poincar\'e dual to $Q$. The Euler class is just the top Chern class
  (see \ref{thm:euler-chern}), so $e(\cO(5)) = c_1(\cO(5))$,
  \[ \chi(Q) = \int_Q e(Q) = \int_Q c_3(Q) = \int_{\bP^4} c_3(Q) \wedge c_1(\cO(5)) = \int_{\bP^4} (-40x^3)(5x) = -200 \int_{\bP^4} x^4 = -200. \]
\end{example}

\subsection{The Hirzebruch--Riemann--Roch Formula}

Here $E$ is a rank $r$ holomorphic vector bundle over a compact
complex $n$-fold $X$. The Hirzebruch--Riemann--Roch formula is part of
a long sequence of generalizations of Gauss--Bonnet, relating
geometric quantities to topological quantities.

\begin{definition}
  Again, formally factor $c(E) = \prod_{k=1}^r (1 + a_k)$. The {\bf
    Todd class} is
  \[ \td(E) \coloneqq \prod_{i=1}^r \frac{a_i}{1 - \exp(-a_i)} = 1 + \frac{1}{2}c_1(E) + \frac{1}{2}(c_1(E)^2 + c_2(E)) + \cdots. \]
  The Todd class $\td(X)$ of $X$ is defined as $\td(X) \coloneqq
  \td(TX)$.
\end{definition}

\begin{theorem}[Hirzebruch--Riemann--Roch, {\cite[Theorem A.4.1]{Hartshorne1997}}]
  Let $E$ be a holomorphic vector bundle over a compact complex
  manifold $X$. Then
  \[ \chi(E) = \int_X \ch(E) \wedge \td(X). \]
  where on the right hand side we only integrate the top form, i.e.
  $\sum_k \ch_k \wedge \td_{n-k}$.
\end{theorem}

We can often use the Hirzebruch--Riemann--Roch (HRR) formula to
compute the dimension of a specific cohomology group, either because
some other cohomology groups vanish, or because we know their
dimensions. Before we begin calculating anything, we need the
following helpful result.

\begin{theorem}[Grothendieck's vanishing theorem, {\cite[Theorem 2.7]{Hartshorne1997}}]
  Let $X$ be a Noetherian topological space of dimension $n$. Then
  $H^i(X, \cF) = 0$ for $i > n$ and any sheaf of abelian groups $\cF$.
\end{theorem}

The remainder of this section is examples of the HRR formula. Keep in
mind that we are always working with sheaf cohomology, not de Rham
cohomology. For example, $H^0(TX)$ is by no means equal to $H^0(X)$,
and is not freely generated by connected components of $TX$. Instead,
$H^0(TX) = \Gamma(TX)$, and since global sections generate
automorphisms, $H^0(TX)$ consists of holomorphic automorphisms of $X$.

\begin{example}
  Let $\cM_g$ denote the {\bf moduli space of complex structures} on a
  genus $g$ closed surface. We shall see later (or recall from
  Teichm\"uller theory) that $\dim \cM_g = \dim_{\bC} H^1(TX)$ where
  $X$ is a genus $g$ closed Riemann surface. HRR gives
  \begin{align*}
    \dim_{\bC} H^0(TX) - \dim_{\bC} H^1(TX)
    &= \chi(TX) = \int_X \ch(TX) \wedge \td(TX) \\
    &= \int_X (1 + c_1(TX)) \wedge (1 + (1/2)c_1(TX))
    = \frac{3}{2} \int_X c_1(TX) = 3 - 3g.
  \end{align*}
  The last equality comes from $c_1$ being the top Chern class for
  $X$, i.e. the Euler class, so applying generalized Gauss--Bonnet
  gives $\int_X c_1(TX) = \chi(X) = 2 - 2g$. For $g \ge 2$, the
  Riemann surface $X$ has no non-trivial automorphisms, so $\dim_{\bC}
  H^0(TX) = 0$. Hence $\dim \cM_g = 3g - 3$.
\end{example}

\begin{example}
  An important object in mirror symmetry is the space of holomorphic
  maps from a Riemann surface $\Sigma$ to a Calabi--Yau $n$-fold $M$,
  i.e. a K\"ahler $n$-fold $M$ with $c_1(M) = 0$. (We'll be more
  careful about the definition of Calabi--Yau later.) An infinitesimal
  deformation of a holomorphic map, given by a vector field $\chi^i$,
  must satisfy $\bdi \chi^i = 0$ if we want the deformed map to still
  be holomorphic. Hence $\chi \in H^0_{\bdi}(\phi^* TM)$, the space of
  such deformations. By HRR,
  \begin{align*}
    \dim_{\bC} H^0(\phi^*TM) - \dim_{\bC} H^1(\phi^*TM)
    &= \int_X \ch(\phi^*TM) \wedge \td(\Sigma) \\
    &= \int_X (n + \phi^*c_1(TM)) \wedge (1 + (1/2)c_1(\Sigma)) = n(1-g).
  \end{align*}
  We assume for now that $H^1(\phi^* TM) = 0$, so the space of
  deformations is $n(1-g)$-dimensional. For $n = 3$ and $g = 0$, the
  dimension is $3$, but there is also a $3$-dimensional group of
  automorphisms of the genus-zero Riemann surface $\bP^1$ which does
  not affect the image curve. Hence the space of genus $0$ holomorphic
  curves inside a Calabi--Yau $3$-fold is zero, and we may be able to
  count them!
\end{example}

\begin{example}
  Let $X$ be a connected compact curve and $L$ a holomorphic line
  bundle on $X$. Fact: the natural isomorphism $H^2(X, \bZ) = \bZ$ is
  given by integration over $X$, and under this isomorphism we have
  $c_1(L) \mapsto \deg(L)$ (\cite[Exercise 4.4.1]{Huybrechts2005}). By
  HRR,
  \[ \chi(L) = \int_X c_1(L) + \frac{1}{2} c_1(X) = \deg(L) + 1 - g. \]
  For the case $L = \cO(D)$ for a divisor $D$ on $X$, we know
  $\dim_{\bC} H^0(L) = \ell(D)$, and by Serre duality $\dim_{\bC}
  H^1(L) = \dim_{\bC} H^0(\cO(K - D)) = \ell(K - D)$ where $K$ is the
  canonical divisor. Hence we recover the classical {\bf Riemann--Roch
    formula} $\ell(D) - \ell(K-D) = \deg D + 1 - g$.
\end{example}

\todo{Huybrecht def 5.1.3 of Hirzebruch $\chi_y$-genus}

\todo{Prove Huybrecht cor 5.1.4 in detail to get generalized Gauss--Bonnet}

\section{Fixed-Point Theorems}

%% \begin{theorem}[Poincar\'e--Hopf index theorem]
%%   Let $M$ be a compact differentiable manifold, and $X$ a vector field
%%   on $M$ with isolated zeros pointing in the outward normal direction
%%   along $\di M$. Then $\sum_i \ind_X(x_i) = \chi(M)$.
%% \end{theorem}

%% \begin{example}
%%   Consider the holomorphic vector field $u \pder{}{u}$ on $\bP^1$.
%%   Since $\pder{}{u} = -v^2 \pder{}{v}$, we have $u \pder{}{u} = -v
%%   \pder{}{v}$. Hence this vector field has two zeros at $u = 0$ and $v
%%   = 0$, and these zeros both have index $1$. Note that $\chi(\bP^1) =
%%   2$.
%% \end{example}

\section{Calabi--Yau Manifolds}

\section{Toric Geometry}

\chapter{Physics Preliminaries}
\section{Overview}
\todo{Make a shorter introduction.}
\begin{enumerate}
    \item Choose Manifold $(M,g)$ with or without boundary. If with, then
        specify additional data. $g$ may be Riemannian or Lorentzian.
    \item \textbf{Objects:} fields. Here are some examples:
        \begin{itemize}
        \item Gauge fields: connections over a principal bundle over $M$.
        \item Matter fields: sections of a vector bundle over $M$.
        \end{itemize}
        A \textbf{quantum gauge theory} has fields which are sections of
        associated vector bundles.
        \begin{itemize}
        \item Sigma-model fields: $\phi\colon M \to X$, for some target
            manifold $X$.
        \end{itemize}
        A \textbf{quantum gravity theory} is obtained by integrating over
        various choices of metrics on $M$.
    \item The \textbf{action} allows us to define the path-integral by
        weighting the fields with $e^{-S}$ or $e^{iS}$ depending on whether
        or not $M$ is Riemannian or Lorentzian.
    \item Boundaries. If $\di M = \bigcup_i B_i$ then the set of field
        configurations on the boundary give rise to Hilbert spaces: $\mc
        H_i$. 
        The path integral can be viewed as a map $\bigotimes_i\mc H_i\to
        \bC$.
    \item Boundaries (evolution).
        Suppose $M=N\times [0,t]$ where $\di N = \varnothing$.
        This corresponds to a multi-linear map $\mc H^*\ten\mc H\to\bC$ or
        equivalently a linear map $U(t):\mc H\to \mc H$. Moreover, by
        gluing two manifolds together we get the following relation:
        \[ U(t_1+t_2)=U(t_1)U(t_2).\] By a theorem from functional analysis
        there exists a Hermitian operator $H$, which we call the
        Hamiltonian, which is given by $U(t) = e^{-tH}$ or $U(t) =
        e^{-itH}$ in the Euclidean or Minkowski case, respectively.
    \item Dimensionality.
        \begin{enumerate}
            \item Kaluza--Klein reduction is one way to acheive the
                effective 4-dimensional space-time that we see.
            \item Damping non-constant modes with action $e^{-S}$ effective
                path integral on the larger component.

                ``Luckily for us, the study of mirror symmetry entails
                studying QFTs with $d=2$, so our aim is to study mainly
                low-dimensional QFTs.''
        \end{enumerate}
    \item $d=0$.
        \begin{enumerate}
            \item Old ingredients: fermionic fields, supersymmetry.
            \item New ingredients: localization and deformation invariance.
        \end{enumerate}
    \item $d=1$, Quantum Mechanics.
        \begin{enumerate}
            \item New ingredients: SUSY $\sg$-models, Landau--Ginzburg
                ($\sg$-models with extra potential functions on target)
        \end{enumerate}
    \item $d=2$
        \begin{enumerate}
            \item Almost free theories: $\sg$-models with a flat torus
                being the target. We review T-duality here.
            \item $\sg$-models on K\"ahler manifold: Gauge theoretic
                description, connection to Landau--Ginzburg. (Toric
                geometry is needed here.)
            \item Superspace review and a tickle of Mirror Symmetry.
        \end{enumerate}
\end{enumerate}
\section{QFT with $d=0$}
First, let us discuss the fields. For a real-valued theory with $\dim M=0$, 
we may identify functions $X:M\to\bR$ with just an element $X\in\bR$. To
model fermions we shall ask our variables to anti-commute:
$\psi_i\psi_j=-\psi_j\psi_i$.

The path-integral, or what Hori et al. calls the \textbf{partition
function}, is given by the expression \[Z \coloneqq \int e^{-S[X,\psi]}dX
d\psi.\] Whether it's for correlation appearing in statistical mechanics or
scattering amplitudes, many physical quantities can be expressed in terms
of \textbf{correlation functions}. Using the path integral approach, a
correlation function is an expectation value of an operator weighted by
$e^{-S}$: \[\braket{F(X,\psi)} = \int F(X,\psi) e^{-S}  dXd\psi.\] Let's be
clever. For the moment, assume that $F$ is a polynomial in $X$ and $\psi$.
Modifying the argument of the exponential $\exp(-S + J_1X+J_2\psi)$ and
applying the correct number of derivatives we can write:
\[ \braket{F} = \FR{\di}{\di J_1}\FR{\di}{\di J_2}\cdots
\biggr|_{J_1=J_2=0} \left(\int e^{-S+J_1X+J_2\psi}\, dXd\psi\right). \]
The integral on the right side may be denoted by $Z[X,\psi;J_1,J_2]$.

\todo{For completeness: feynman diagrams?}
\todo{Recall: Fermionic path integral rules
\label{fermionic-path-integral-rules}.}

\subsection{Supersymmetry}

Supersymmetry is both mathematically and physically a useful tool. From a
mathematical point of view, ``the classical field equations have
non-trivial odd symmetries: eg. gradient flow lines in Morse theory,
holomorphic curves, gauge theory instantons, monopoles, Seiberg--Witten
solutions, hyperKahler structures, Calabi-Yau metrics, metrics of G2 and
Spin7 holonomy.'' From a physical point of view, ``[supersymmetric]
theories offer a possible way of solving the 'hierarchy problem,' the
mystery of the enormous ratio of the Planck mass to the 300 GeV energy
scale of electroweak symmetry breaking. Supersymmetry also has the quality
of uniqueness that we search for in fundamental physical theories. There is
an infinite number of Lie groups that can be used to combine particles of
the same spin in ordinary symmetry multiplets, but there are only eight
kinds of supersymmetry in four spacetime dimensions, of which only one, the
simplest, could be directly relevant to observed particles.'' \todo{Revamp
this introduction.}


Let us consider a simple QFT with one bosonic field $X$ and two fermionic
fields $\psi_1,\psi_2$ with an equally simple action: 
\begin{align} S = S_0(X) - \psi_1\psi_2 S_1(X).
    \label{susy-simple-action}
\end{align}
Taylor expanding the path-integral, and using the fermionic path-integral
rules (\ref{fermionic-path-integral-rules}), we get the following
expression for the integral: \[ \int e^{-S_0(X)+\psi_1\psi_2S_1(X)}
dXd\psi_1d\psi_2 = \int e^{-S_0}S_1(X).\]
For the rest of the section, we are going to pick a convenient choice for
the functions $S_0(X), S_1(X)$ so that we get a feeling for what a
supersymmetric theories have to offer. Fix a function $h:\bR\to \bR$, and let
\[ S_0(X) = \FR{1}{2} (\di h)^2, \qquad\qquad S_1(X) = \di^2 h.\] We use
the notation $\di h = h'(X)$ to refer to the derivative of $h$.  Together
with the form of the action \eqref{susy-simple-action}, we get a theory
that is invariant under the transformations:
\begin{align}
\delta X = \epsilon^1\psi_1 + \epsilon^2\psi_2,
\qquad \delta\psi_1 = \epsilon^2\di h,
\qquad \delta\psi_2 =-\epsilon^1\di h.
\label{susy-simple-variation}
\end{align}
Let's check this explicitly. Notice that the variational operator $\delta$
can be treated as if it were a derivation: $\delta(AB) = (\delta A)B +
A(\delta B)$. \todo{Proof that variation is a derivation.} Then,
\begin{align*}
\delta(S) &= \delta(S_0) - \delta(\psi_1)\psi_2S_1 -
\psi_1\delta(\psi_2)S_1 - \psi_1\psi_2\delta(S_1)\\
&= (\delta h)(\di^2 h(\epsilon^1\psi_1+\epsilon^2\psi_2))
      -(\epsilon^2\di h)\psi_2\di^2 h
      -\psi_1(-\epsilon^1\di h)\di^2 h
      -\psi_1\psi_2(\di^3 h (\epsilon^1\psi_1+\epsilon^2\psi_2))\\
&=\di h\di^2 h
(\epsilon^1\psi_1+\epsilon^2\psi_2-\epsilon^2\psi_2+\psi_1\epsilon^1)
+\di^3 h (-\psi_1\psi_1\psi_2\epsilon^1 + \psi_1\epsilon^2\psi_2\psi_2)\\
&=0
\end{align*}
To show that the measure is also invariant, we proceed analogously except
we must keep in mind that we \emph{pullback} the measure and do not push it
forward:
\begin{align*}
f^*(dX\wedge d\psi_1\wedge d\psi_2)
&= df^*(X)\wedge df^*(\psi_1) \wedge df^*(\psi_2)\\
&= dX\wedge d\psi_1\wedge d\psi_2
  -d(\epsilon^1\psi_1+\epsilon^2\psi_2)\wedge d\psi_1\wedge d\psi_2\\
  &\qquad-dX\wedge d(\epsilon^2\di h)\wedge d\psi_2\\
  &\qquad-dX\wedge d\psi_1\wedge d(-\epsilon^1\di h)\\
&= 0
\end{align*}
The last two terms are $0$ because $d(\di h) = (\di^2 h)dX$ and by wedgeing
with another $dX$ zeros out the entire term.

\subsection{Localization}
The group of supersymmetry transformations is surprisingly large. We
continue with the model described in the previous section and in particular
show that if $\di h$ is nowhere zero then the path integral vanishes.
The idea is to use a supersymmetry transformation to make one of the
fermionic fields vanish: $S(X,\psi_1,\psi_2) = S(\hat X,0,\hat \psi_2)$
which will allow us to simplify the path-integral. Choose,
\[ \hat X = X-\FR{\psi_1\psi_2}{\di h(X)},\qquad
   \hat\psi_1 = \alpha(X)\psi_1,\qquad
   \hat\psi_2 = \psi_1+\psi_2.\]
In particular, taking $\alpha(X)=0$ this is a susy transformation with
$\epsilon^1=\epsilon^2=-\FR{\psi_1}{\di h}$. In the new coordinates,
$\hat\psi_1=0$ just as we were looking for. The path integral can be
evaluated:
\begin{align*}
    \int e^{-S(X,\psi_1,\psi_2)}\,dXd\psi_1d\psi_2
&= \int e^{-S(\hat X(X),0,\hat\psi_2(\psi_1+\psi_2))}\,dXd\psi_1d\psi_2\\
&= \int e^{-S(\hat X,0,\hat\psi_2)}\,\left( \alpha - 
(\di^2 h)(\di h)^{-2}\hat\psi_1\hat\psi_2 \right)\,
d\hat Xd\hat \psi_1d\hat \psi_2
\end{align*}
The first term in the sum, $\int d\hat\psi_1 \left(\int e^{-\hat
S}\alpha\right) = 0$ because the argument is independent of $\hat\psi_1$.
The second term vanishes because \todo{Show that the integral vanishes
because the argument is a total derivative.}

This transformation only is valid away from the critical points of $h$.
Taking the path integral as a sum over critical points and expanding 
$h(X) = h(X_c)+\FR{\alpha_c}{2}(X-X_c)^2 + \cdots$, we get
\begin{align*}
    Z &= \sum \int \FR{dXd\psi_1d\psi_2}{\sqrt{2\pi}}
    \exp\left(-\FR{1}{2}\alpha_c^2(X-X_c)^2 + \alpha_c\psi_1\psi_2\right)\\
    &= \sum_{X_c} \FR{1}{\sqrt{2\pi}}\int dX
    \exp\left(-\FR{1}{2}\alpha_c^2(X-X_c)^2\right) \alpha_c\\
&= \sum_{X_c} \alpha_c \SFR{1}{\alpha_c^2}
\end{align*}
The factor of $\sqrt{2\pi}$ can be factored out by normalizing the measure
appropriately. This normalization corresponds to defining the correlation
functions as $\FR{Z[\lambda,0]}{Z[0,0]}$, which in our case,
$Z[0,0]=\sqrt{2\pi}$. Therefore the path-integral becomes an integer:
\[ Z = \sum_{X \colon h'(X)=0} \FR{h''(X)}{|h''(X)|}.\]
\begin{remark}[Localization] The path integral is localized at loci where
the fermionic variation under supersymmetry vanishes. 
\end{remark}
\begin{remark} The function $h$ in our model is called the
\textbf{superpotential}, the critical points of which characterize the
path-integral.
\end{remark}

\subsection{Deformation Invariance}
\begin{proposition} Suppose $\rho$ is a function such that $\rho$ and
$\di\rho$ vanishes at infinity. Then $h\to h+\rho$ leaves the action
invariant.
\end{proposition}
\begin{proof}
Take $g = \di\rho(X)\psi_1$ and consider susy variation $\delta_\epsilon
g$, with $\epsilon^1=\epsilon^2=\epsilon$. Now consider a deformation of
the superpotential: $h\to h+\rho$. Then the action $S = \FR{1}{2}(\di h)^2
- \di^2h \psi_1\psi_2$ transforms with $\epsilon\delta_\rho S =
\delta_\epsilon g$. Therefore $\braket{\delta_\rho S} =\epsilon^{-1}
\braket{\delta_\epsilon g}=0$.
\end{proof}

\subsection{Landau--Ginzburg}
After delving into the details of the previous model it is comforting to
note that the complex analogue is what is known as the Landau--Ginzburg
theory: $(X,\psi_1,\psi_2) \leadsto (z,\bar z, \psi_1, \psi_2, \bar\psi_1,
\bar\psi_2)$, and action:
\begin{align}
S = |\di W|^2-(\di^2 W)\psi_1\psi_2 - \overline{\di^2W}\bar\psi_1\bar\psi_2.
\label{LandauGinzburgaction}
\end{align}
We call the holomorphic function $W(z)$ the \textbf{superpotential} and it
will play the role $h$ did in the previous sections. The complex susy
transformations then take the form of:
\begin{align}
\delta z &= \epsilon^1\psi_1+\epsilon^2\psi_2, \qquad
\delta\psi_1 = \epsilon^2\overline{\di W},\qquad
\delta\psi_2 =-\epsilon^1\overline{\di W},\qquad
\delta \bar z=\delta\bar\psi_1 = \delta\bar\psi_2=0\\
\bar\delta\bar z &= \bar\epsilon^1\bar\psi_1+\bar\epsilon^2\bar\psi_2, \qquad
\bar\delta\bar \psi_1 = \bar\epsilon^2\di W,\qquad
\bar\delta\bar \psi_2 =-\bar\epsilon^1\di W,\qquad
\bar\delta z=\bar\delta\psi_1 = \bar\delta\psi_2=0
\label{complexsusytransformations}
\end{align}
Once again, the localization principle applies by choosing
$\epsilon^1=\epsilon^2=-\FR{\psi_1}{\overline{\di W}}$, and writing $W(z) =
W(z_c)+\FR{\alpha_c}{2}(z-z_c)^2+\cdots$, then with proper normalization,
the path-integral integrates to:
\begin{align*}
    Z &= \FR{1}{2\pi}\int
    e^{-|\alpha(z-z_c)|^2}(\alpha\psi^1\psi^2+\bar\alpha\bar\psi_1\bar\psi_2)\,
    dzd\bar z\,d\psi_1d\psi_2\,d\bar\psi_1d\bar\psi_2\\
&= \FR{|\alpha|^2}{2\pi} \sum_{z_c} \int e^{-|\alpha(z-z_c)|^2}\,dzd\bar z\\
&= \FR{|\alpha|^2}{2\pi} \sum_{z_c} \int e^{-|\alpha|^2(x^2+y^2)} \,dxdy
= \sum_{z_c} 1 = \text{\# of critical points of $W$}
\end{align*}
\subsection{Holomorphicity and Correlation functions}
%Existence of supersymmetry does not allow us to trivially compute arbitrary
%correlation functions but it does help with some. In particular, if we have
%an observable $\mc O$ that is invariant under one of the supersymmetry
%transformations then perhaps we can use localization to annihilate that
%fermionic 

\todo{Prove that the Localization principle applies to holomorphic
observables.}
Let $f(z)$ be a holomorphic observable. Then by the localization principle,
\begin{align*}
    \braket{f(z)} &= \int f(z) e^{-S}\\
    &= \int f(z)|\di^2 W|^2 e^{-\FR{1}{2}|\di W|^2}\\
    &= \sum_{z_c} f(z_c)
\end{align*}

The set of all fields that vanish under $\bar\delta$ are called
\textbf{chiral fields}. In particular because $\bar\delta$ is a derivation
it follows that this set is closed under multiplication. If we take
$\epsilon^1=\epsilon^2$ then $\bar\delta^2 = 0$ and so we can consider then
cohomology ring $\FR{\ker\bar\delta}{\im\,\bar\delta}$ is often called the
\textbf{chiral ring}.
%\begin{ex}(Chiral Ring for $d=0$ Landau--Ginzburg)
%Applying $\bar\delta$ to the generators of fields, $z,\bar z,
%\psi_i,\bar\psi_i$, we will obtain the image
%In the multivariable case LG action looks like
%\[ S(z_i,\psi^i_1,\psi^i_2)=\sum
%|\di_iW(z_1,\dots,z_N)|^2-\di_i\di_jW\psi^i_1\psi^i_2 -
%\overline{\di_i\di_jW}\bar\psi^i_1\bar\psi^j_2.\]
%Similarly, we can show that $\mc R = \bC[z_1,\dots,z_N]/\braket{\di_i W}$.
%Explore the connections with Poincar\'e duality here and show that this
%describes some of the structure on the chiral ring.
%\end{ex}

\section{QFT with $d=1$}
Let $M$ be a $1$-dimensional manifold. We know that there are only two
diffeomorphism classes of $1$-manifolds without boundary: $\bR$ and $S^1$.
For simplicity, the only manifold with boundary that we will consider is
the unit interval $[0,1]$.

\subsection{Quantum Mechanics}
\begin{enumerate}
    \item $S = \int Ldt = \int \left[\FR{1}{2}\PFR{dX}{dt}^2-V(X)\right]dt$
    \item Hilbert space of states as $L^2(\bR,\bC)$.
    \item $f\mapsto Z_{t_2;t_1}f$ given by $(Z_{t_2;t_1}f)(Y) = \int
        Z(Y,t_2;X,t_1)f(X)dX$
    \item Time evolution operator $Z_t = e^{-itH}$.
    \item On the circle: $Z_E(\beta) = \int dX_1 Z_{E,\beta}(X_1,X_1) =
        \Tr\exp(-\beta H)$.
    \item Example: Harmonic oscillator
        \begin{enumerate}
            \item To compute partition function in two different way: operator
                formalism $\Tr e^{-\beta H}$ and directly using
                path-integral requires zeta function regularization.
        \end{enumerate}
    \item Example: Sigma model with target space: $S^1$ with radius $R$.
        \[ S(X) = \int \FR{1}{2}\dot X^2 dt.\]
    \item Example: Sigma model with target space: $\bR$.
    \item Example: Sigma model with target space: $(M,g)$.
        \[ S = \FR{1}{2}\int dt g_{ij}(X) \FR{dX^i}{dt}\FR{dX^j}{dt},
        g_{ij} = \delta_ij + C_{ijkl}X^kX^l+\cdots.\]
        \begin{enumerate}
            \item Operator formalism
            \item Ambiguity of Hamiltonian: related to the measure in the
                path integral.
            \item Supersymmetry imposes constraint: fixes a particular
                Hamiltonian
        \end{enumerate}
    \item Complicated actions require approximations to evaluate. The
        semi-classical approximations take the classical solution: $\delta
        S/\delta X\bigr|_{X=X_{cl}} = 0$ and expand the action around that
        critical point:
        \[ S(X) = S(X_{cl}) + \FR{(\delta X)^2}{2}\FR{\delta^2 S}{\delta
        X^2}\biggr|_{X=X_{cl}} + \mc O((\delta X)^3).\]
        Then the partition function becomes
        \begin{align*}
            Z &= \int \mc DX e^{iS(X)}
= e^{iS(X_{cl})} \int \mc D(\delta X) \exp\left(i \FR{(\delta
X)^2}{2}\FR{\delta^2 S}{\delta X^2}\right)\\
            &= \FR{\exp\left(iS(X_{cl})\right)}{\sqrt{\det\PFR{\delta^2
                S(X_{cl})}{\delta X^2}}}
        \end{align*}
\end{enumerate}

\subsection{Structure of Hilbert Space}
\begin{definition}
A \textbf{supersymmetric quantum mechanical system} is $\bZ_2$-graded
Hilbert space $\mc H$ with an even operator $H$ and two odd operators
$Q,Q^\dag$ called supercharges satisfying: $Q^2=(Q^\dagger)^2=0$,
$\{Q,Q^\dag\}=2H$.
\end{definition}
A consequence is $[H,Q]=[H,Q^\dag]=0$. The operator defining the $\bZ_2$
grading is denoted $(-1)^F$. Moreover, $H = \FR{1}{2}\{Q,Q^\dagger\}\ge0$
with equality if and only if the state is annihilated by both supercharges.
If a state is annihlated by a charge then it is invariant under that
symmetry, which means that ground states of a SUSY quantum mechanical
system are automatically supersymmetric.

Since $Q,\cnj Q, (-1)^F$ commute with the Hamiltonian, we first put a
grading on the Hilbert space, by energy levels, and then grade the
resulting subspaces by the $\bZ_2$ grading: $\mc H = \bigoplus \mc
H_n^+\dsum \mc H_n^-$. If we define $Q_1=Q+Q^\dagger$ then \[Q_1^2=2H.\]
For $E_n > 0$, $Q_1^2 = 2E_n$ so $Q_1$ is invertible and defines an
isomorphism \[\mc H^+_n \cong \mc H^-_n.\] 
This has an important consequence for adiabatic/continuous deformations.
Degenerate energy levels may split but the dimension of positive energy
bosonic states is the same as the fermionic states. Some of the degenerate
ground states may split off to form positive energy states, but they again
must come in pairs. The invariant that we should consider is the
difference:
\begin{align} \dim \mc H^B_0 - \dim \mc H^F_0 
= \Tr \bigl[(-1)^Fe^{-\beta H}\bigr].\label{WittenIndex} 
\end{align}
This is called the \textbf{supersymmetric index} or the \textbf{Witten
index} and is often written as $\Tr(-1)^F$. 

To touch base with mathematics let us consider $Q$ as a coboundary operator
of the complex: $\cdots \to \mc H^B\to \mc H^F \to \mc H^B \to \mc H^F \to
\cdots$. This complex is naturally graded by the energy levels, and $Q$
preserves this grading. However, the cohomology is trivial: take a
$Q$-closed state $Q\ket{\alpha}=0$ and apply $\ket\alpha = \FR{1}{2E_n}H
\ket\alpha = \FR{1}{2E_n}\{Q,Q^\dagger\}\ket\alpha = Q \left( \FR{1}{2E_n}
Q^\dagger\ket\alpha \right) \in \im Q$. Therefore the cohomology group on
the bosonic and fermionic Hilbert spaces gets a contribution only from the
(supersymmetric) ground states:
\begin{align} H^B(Q) \cong \mc H^B_{(0)},\qquad H^F(Q) \cong \mc H^F_{(0)}.
\label{groundStateCohom}
\end{align}
Note this is an isomorphism not an equality as $H^\bullet(Q)$ is obtained
by a quotient, where as $\mc H^B_{(0)}$ is an honest subgroup of the
Hilbert space.

In this form, the Witten index $\Tr(-1)^F$, \eqref{WittenIndex}, resembles
the Euler characteristic for the complex. It is also not surprising that
because of the appearance of the trace we may write down path integral
expressions for $\Tr(e^{\beta H})$ and $\Tr\left[ (-1)^Fe^{-\beta H}
\right]$. \todo{Understand the periodic boundary conditions in path
integral for fermions.}
\subsection{Supersymmetric Hilbert space: Example} 
% Example: single-variable potential theory 

\begin{example} Let us consider a supersymmetric theory with a single bosonic
variable $x$ and a complex superpartner $\psi$. The Lagrangian is given by:
\[ L = \FR{1}{2} \dot x^2 - \FR{1}{2} h'(x)^2 + \FR{i}{2}\left( \cnj\psi
\dot\psi - \dot{\cnj\psi}\psi \right) - h''(x)\cnj\psi\psi,\]
where $\cnj\psi=\psi^\dagger$, and the second term plays the role of the
potential $-V(x)$. Now the following transformations change the Lagrangian
by a total time derivative:
\begin{align}
    \delta x = \epsilon\cnj\psi-\cnj\epsilon\psi,\qquad
    \delta \psi=\epsilon(i\dot x+h'(x)),\qquad
    \delta\cnj\psi=\cnj\epsilon(-i\dot x+h'(x)),
    \label{singlepotentialSUSYtransformations}
\end{align}
where $\epsilon=\epsilon_1+i\epsilon_2$ and $\cnj\epsilon=\epsilon^*$ is
the complex conjugate. As long as the boundary variation vanishes (which
will be the case on an open manifold) then the action is invariant. Let
$\delta_1,\delta_2$ be variations as in \eqref{singlepotentialSUSYtransformations}
but with $\epsilon_2=0$ and $\epsilon_1=0$ respectively. Then:
\[[\delta_1,\delta_2]x =2i(\epsilon_1\cnj\epsilon_2 -
\epsilon_2\cnj\epsilon_1)\dot x,\qquad\qquad [\delta_1,\delta_2]\psi
=2i(\epsilon_1\cnj\epsilon_2-\epsilon_2\cnj\epsilon_1)\dot\psi.\]
Applying the Noether procedure with $\epsilon=\epsilon(t)$ to get: $\delta
L = -i\dot\epsilon Q - i\dot{\cnj\epsilon}\cnj Q$ where\todo{Derive this
Noether charge!}
\[Q=\cnj\psi(i\dot x+h'(x)),\qquad\qquad\cnj Q = \psi(-i\dot x+h'(x)).\]

\textbf{Quantizion:} Compute conjugate variables, promote to operators,
impose commutation relations.

$H=\FR{1}{2}p^2+\FR{1}{2}h'(x)^2+\FR{1}{2}h''(x)(\cnj\psi\psi-\psi\cnj\psi)$.

Check:
\begin{enumerate}
    \item $[H,Q]=[H,\cnj Q] = 0$, note $Q^\dagger = \cnj Q$.
    \item Let $\mc O(x,\psi,\cnj\psi)$ be any observable (self-adjoint
        operator). Show that $\delta \mc O = [\hat\delta,\mc O]$ where
        $\hat\delta = \epsilon Q+\cnj\epsilon\cnj Q$.
    \item The Hilbert space decomposes as $L^2(\bR,\bC)\ten\bC^2$ as a
        result of the bosonic and fermionic degrees of freedom.
\end{enumerate}
Define the fermionic number operator, $F = \cnj\psi\psi$ which measures the
number of fermions of a state (either 0 or 1). A more convenient operator
that (does not annihilate the state without fermions) is given by $(-1)^F$.
Check:
\begin{enumerate}
    \item $[F,Q]=Q, [F,\cnj Q]=-\cnj Q$
    \item $\left\{ Q,(-1)^F \right\}=\left\{ \cnj Q,(-1)^F \right\}=0$
    \item $\left\{ Q,Q \right\}=\left\{ \cnj Q,\cnj Q \right\}=0$.
    \item $\left\{ Q,\cnj Q \right\} = 2H$
\end{enumerate}
\end{example}
\subsection{Supersymmetric Ground States: Towards Morse Theory}
As discussed before, the grounds states of a supersymmetric Hamiltonian are
precisely the states which span the intersection of $\ker Q$ and $\ker\cnj
Q$. In the basis $\{\ket 0, \cnj\psi\ket0\}$, the supercharges are
\[Q=\cnj\psi(ip+h'(x))=\begin{pmatrix}0&0\\\FR{d}{dx}+h'(x)&0\end{pmatrix},\qquad
\cnj Q=\psi(-ip+h'(x))=\begin{pmatrix}0&-\FR{d}{dx}+h'(x)\\0&0\end{pmatrix}.\]
Therefore we look for $f_1,f_2$ so that $\Psi= f_1(x) \ket0 + f_2(x)
\cnj\psi\ket0$ is annihilated by $Q$ and $\cnj Q$. This leads to two
differential equations with exact solutions: \[f_1(x)=c_1e^{-h(x)},\qquad
f_2(x)=c_2 e^{h(x)}.\] We also want $f_1,f_2 \in L^2$ which means that
they are either $0$ or exponentially decaying at \emph{both} $x=\pm\infty$.
Therefore we have two solutions:
\[ e^{-h(x)}\ket 0, \text{if $h\to\infty$, as $x\to\pm\infty$,}\qquad
e^{h(x)}\cnj\psi\ket 0, \text{if $h\to-\infty$, as $x\to\pm\infty$.}\]

In the case of a harmonic oscillator, $h(x) = \FR{\om}{2}x^2$, with
potential $V(x) = \FR{1}{2}(h'(x))^2 = \FR{\om^2x^2}{2}$. Therefore the
ground states become:
\[ \Psi_{\om>0} = e^{-\FR{1}{2}\om x^2}\ket 0,\qquad \Psi_{\om<0}=
e^{-\FR{1}{2}|\om|x^2}\cnj\psi\ket 0.\]


\subsection{Semi-classical analysis}
We now switch to the Hamiltonian formulation, and at the same time rescale
$h\to \lam h$ for a large $\lam \gg 1$. This goes back to the deformation
invariance result that the action is invarinat under deformations of this
form:
\[H=\FR{1}{2}p^2+\FR{\lam^2}{2}(h'(x))^2+\FR{\lam}{2}h''(x)[\cnj\psi,\psi].\]
The ground states will be localized at the minimum of $(h'(x))^2$.
Expanding $h$ around a critical point, and at the same time rescale the
coordinates $x-x_i = \FR{1}{\sqrt\lam}(\tilde x-\tilde x_i)$ the expansion
becomes \[h(x) = \FR{1}{2\lam}h''(x_i)(\tilde x-\tilde
x_i)^2+\FR{1}{6\lam^{3/2}}h'''(x_i)(\tilde x-\tilde x_i)^3 +
O(\lam^{-2}).\]
The Hamiltonian the becomes:
\[ H = \lam\left( \FR{1}{2}\tilde p^2 + \FR{1}{2}h''(x_i)^2(\tilde x-\tilde
    x_i)^2+\FR{1}{2}h''(x_i)\left[ \cnj\psi,\psi \right]\right)
+\lam^{1/2}(\cdots)+(\cdots)+\mc O(\lam^{-1/2}),\]
where $\tilde p = -i\FR{d}{d\tilde x}$.
The $\mc O(\lam)$ term is a supersymmetric harmonic oscillator with
$\om=h''(x_i)$. The ground state around $x_i$ are:
\begin{align*}
\Psi_i &= e^{-\FR{\lam}{2}h''(x_i)(x-x_i)^2}\ket 0+\mc O(\lam^{-1/2}),\ \ \text{if
$h''(x_i)>0$},\\
\Psi_i &= e^{-\FR{\lam}{2}|h''(x_i)|(x-x_i)^2}\ket 0+\mc O(\lam^{-1/2}),\ \text{if
$h''(x_i)<0$}
\end{align*}
This perturbation theory, tells us there is exactly one supersymmetric
ground state for every critical point and so the Witten index is given by
\[\Tr(-1)^F = \sum_{i=1}^N \sign(h''(x_i)).\]

If the target space is $\bR^n$, with $n$ bosonic and $2n$ fermionic
variables then the Hamiltonian becomes a sum $H = \FR{1}{2}\sum_I
p_I^2+\FR{1}{2}(\di_Ih(x))^2+\FR{1}{2}(\di_I\di_J h)[\cnj\psi^I,\psi^J]$,
and supercharges: $Q = \cnj\psi^I(ip_I+\di_Ih)$, $\cnj Q =
\psi^I(-ip_I+\di_Ih)$. In this case, $h:\bR^n \to \bR$. Expanding about a
critical point of $h$ and choosing the right coordinate system
$\xi^I=\xi^I_{(i)}$: \[ h(x)= h(x_i)+\sum c_I(\xi^I)^2+\cdots\]
In particular, this means that the Hamiltonian simplifies and we again pick
out the $\mc O(\lam)$ term to get the ground state:
\[\Psi_i = \left(\bigotimes_{I\ :\ c_I>0} \exp(-\lam c_I\xi^I)\ket 0\right)
\ten\left(\bigotimes_{I\ :\ c_I<0} \exp(-\lam|c_I|\xi^I)\bar\psi^I\ket0
\right).\] In particular, if we denote the Morse indices of $h$ by $\mu_i$,
the Witten index is given by $\Tr(-1)^F = \sum_{i=1}^N (-1)^{\mu_i}$.

\subsection{Landau--Ginzburg, $n=2m$}
The Landau--Ginzburg model is complex analogue of our discussion above.
The target space will now be $\bC^m \equiv \bR^{2m}$ and we will take the
Lagrangian to be: \[ L = \sum_{i=1}^m \left(|\dot
    z_i|^2+i\bar \psi^i\di_t\psi^{\bar i}+i\bar\psi^{\bar
    i}\di_t\psi^i-\FR{1}{4}|\di_iW|^2\right)
    -\FR{1}{2}\sum_{ij}(\di_i\di_jW\psi^i\bar\psi^j+\di_{\bar i}\di_{\bar
    j}\cnj W\psi^{\bar i}\bar\psi^{\bar j}).\]
Suppose $W$ has $N$ nondegenerate critical points $p_1,\dots,p_N$. That is,
$\det \di_i\di_j W(p_a)\ne 0$. Expanding $W$ around these critical points
and rewriting in real coordinates we get:
\begin{align*}
    W(z) &= \sum_{k=1}^m (z^k)^2+\mc O((z^k)^3)\\
    &= \sum_{k=1}^m\left[ (x^k)^2-(y^k)^2 \right] + \cdots
\end{align*}
Therefore the Morse indices for \emph{any} critical point are equal to the
complex dimension $m$.

\todo{Show that this Landau--Ginzburg model is $\mc N=2$ supersymmetric.}
\begin{comment}
This Landau--Ginzburg model that we are considering here actually has
two pairs of supersymmetry charges $Q_\pm$ and $\cnj Q_\pm$ that generate
the following transformations:
\begin{align*}
    \delta z^i = \epsilon_+\bar\psi^i-\epsilon_-\psi^i
\end{align*}
\end{comment}

\subsection{Sigma Models}
% topology <---> ground states of SUSY-\sg-model
% superpotentials on target <---- physical realization of Morse theory
Let $(M,g)$ be a Riemannian manifold and denote by $\mc T$ the one
dimensional manifold one which our QFT lives. The bosonic and fermionic
variables are maps of the form
\begin{align}
    \phi \colon \mc T\to M,\qquad \psi,\bar\psi\in\Gamma(\mc
    T,\phi^*TM\ten\bC)
    \label{sigmamodelvariables}
\end{align}
The Lagrangian is given by:
\begin{align}
    L &= \FR{1}{2}g_{IJ}\dot\phi^I\dot\phi^J + \FR{i}{2}(\bar\psi^I
    D_t\psi^J-D_t\bar\psi^I\psi^J) - \FR{1}{2} R_{IJKL}\psi^I\bar\psi^J
    \psi^K\bar\psi^L\\
    &=\FR{1}{2}\braket{\dot\phi,\dot\phi} +
    \FR{i}{2}(\braket{\bar\psi,\nabla^{LC}_t\psi}
    -\braket{\nabla^{LC}_t\bar\psi,\psi})
    -\FR{1}{2}R(\psi,\bar\psi,\psi,\bar\psi),
    \label{sigmaModelRiemLag}
\end{align}
The supersymmetry transformations are given by:
\[ \delta\phi^I = \epsilon\bar\psi^I-\bar\epsilon\psi^I,\qquad
\delta\psi^I = \epsilon(i\dot\phi^I-\Gamma^I_{JK}\bar\psi^J\psi^K),\qquad
\delta\bar\psi^I =\bar\epsilon(-i\dot\phi^I-\Gamma^I_{JK}\bar\psi^J\psi^K)s.
\]
The corresponding supercharges are given by:\todo{Work out the Noether
charge for Riemannian manifold sigma model.}
\[ Q = ig_{IJ}\bar\psi^I\dot\phi^J = i\braket{\bar\psi,\dot\psi},\qquad
\bar Q=-ig_{IJ}\psi^I\dot\psi^J=-i\braket{\psi,\dot\psi}.\]
There is an extra symmetry of this free Lagrangian: $(\psi,\bar\psi) \to
(e^{-i\theta}\psi,e^{i\theta}\bar\psi)$. The corresponding Noether charge
is \[F = g_{IJ}\bar\psi^I\psi^J = \braket{\bar\psi,\psi}.\]
Quantizing this system, requires the conjugate momenta: \[p_I = \FR{\di
L}{\di\dot\phi^I}=g_{IJ}\dot\phi^J,\qquad {\pi_\psi}_I =
ig_{IJ}\bar\psi^J.\] Finally, imposing commutation relations:
\[ [\phi^I,p_J] = i\delta^I_{\ J},\qquad \{\psi^I,\bar\psi^J\} = g^{IJ},\]
with all other commutators vanishing. The supercharges are conveniently
write $Q=i\bar\psi^Ip_I, \bar Q = -i\psi^Ip_I$. The ambiguitiy in the
choice of Hamiltonian is fixed if we impose $\{Q,\bar Q\}=2H$.
Finally, note that $[F,Q]=Q$ and $[F,\bar Q]=-\bar Q$ from which we get
$[H,F]=0$.

There exists a natural representation of the observables on the space:
\[\mc H = \Omega(M)\ten\bC,\ \ \text{with}\ \ \braket{\omega_1,\omega_2}_{\mc
H} = \int_M \bar\om_1 \wedge *\om_2.\]
The observables can be assigned the following operators:
\[\phi^I = x^I\cdot,\qquad p_I = -i\nabla_I,\qquad
\bar\psi^I=dx^I\wedge,\qquad \psi^I = g^{IJ} \FR{\di}{\di x^J}\hook\]
The ground state, corresponding to the intersection of all $\ker\psi^I$.
The $F$-charge, or fermion number, is the degree of the form and so the
natural grading on the Hilbert space \[\mc H = \bigoplus
\Omega^p(M)\ten\bC\]
corresponds to the fermion number grading. The supercharge $Q$ is the extrior
derivative, $Q = i\bar\psi^Ip_I = dx^I\wedge\FR{\di}{\di x^I}=d$, and
Hermitian conjugate is $\bar Q = Q^\dagger =d^\dagger$. Finally, we compute
the Hamiltonian $H$ by the supersymmetry relation \[H = \FR{1}{2}\{Q,\bar
Q\} = \FR{1}{2}(dd^\dagger+d^\dagger d) = \FR{1}{2}\Delta,\] the
Laplace--Beltrami operator. The ground states are then just the harmonic
forms: \[\mc H_{(0)} = \Harm(M,g) = \bigoplus_{p=0}^n \Harm^p(M,g).\]

Now we are starting to see the correspondence between supersymmetric
quantum mechanics and differential geometry. In fact, Hodge theory is not
too far away! As we showed before, the ground states can be characterized
by cohomology (see \eqref{groundStateCohom}). Since $[F,Q]=Q$ the
$Q$-complex may be graded by the fermion number, or in more mathematical
language, the de Rham cohomology is graded by the degree of the forms.
In fact, since we know the ground states correspond to the the harmonic
forms we use the same proof as for \eqref{groundStateCohom} to show that
\[\mc H_{(0)}=\Harm(M,g) \cong H^\bullet(Q) = H^\bullet_{dR}(M),\]
and even more that with respect to the fermion number grading,
\[\Harm(M,g) \cong H^p_{dR}(M).\]
The Witten index, $\Tr(-1)^F$, counting the parity of the fermion numbers
becomes: \[\Tr(-1)^F = \sum_{p=0}^n (-1)^p \dim H^p_{dR}(M) = \chi(M).\] 

\todo{Prove Gauss Bonnet using the path integral: Compute the Witten index
$\Tr[(-1)^Fe^{-\beta H}]$ in terms of the Riemann curvature tensor and use
the localization principle in the limit of $\beta\to0$.}

\todos

\addcontentsline{toc}{chapter}{Bibliography}
\bibliographystyle{unsrt}
\bibliography{mirrorsym-notes}



\end{document}
