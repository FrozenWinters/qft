\documentclass{report}
\usepackage{amsmath}
\usepackage{amssymb}
\usepackage{amsthm}
\usepackage{amscd}
\usepackage{fullpage}
\usepackage{braket}
\usepackage{dsfont}
\usepackage{slashed}
\usepackage{comment}
\usepackage{mathtools}
\usepackage[usenames,dvipsnames,svgnames,table]{xcolor}

\theoremstyle{plain}
\newtheorem{theorem}{Theorem}[section]
\newtheorem{lemma}[theorem]{Lemma}
\newtheorem{proposition}[theorem]{Proposition}
\newtheorem{corollary}[theorem]{Corollary}

\theoremstyle{definition}
\newtheorem{definition}[theorem]{Definition}
\newtheorem{axiom}{Axiom}
\newtheorem{example}[theorem]{Example}
\newtheorem{exercise}{Exercise}[section]

\theoremstyle{remark}
\newtheorem*{remark}{Remark}
\newtheorem*{note}{Note}

\newcommand{\FR}[2]{\frac{#1}{#2}}
\newcommand{\PFR}[2]{\left(\frac{#1}{#2}\right)}
\newcommand{\SFR}[2]{\sqrt{\frac{#1}{#2}}}

\newcommand{\mc}{\mathcal}
\newcommand{\lam}{\lambda}
\newcommand{\vphi}{\varphi}
\newcommand{\om}{\omega}
\newcommand{\Om}{\Omega}
\newcommand{\gam}{\gamma}
\newcommand{\sg}{\sigma}
\newcommand{\di}{\partial}
\newcommand{\sdi}{\slashed\partial}
\newcommand{\ddi}[2]{\FR{\partial {#1}}{\partial {#2}}}
\newcommand{\hp}{\hat p}
\newcommand{\ha} { a}
\newcommand{\hb} { b}
\newcommand{\hbd} { b^\dagger}
\newcommand{\had}{ a^\dagger}
\newcommand{\iden}{\mathds{1}}
\newcommand{\colr}[1]{ {\color{red} #1 } }
\newcommand{\colb}[1]{ {\color{blue} #1 } }
\newcommand{\colf}[1]{ {\color{Fuchsia} #1 } }

\newcommand{\elaborate}{{\color{blue} \textbf{Elaborate.}}}
\newcommand{\CHECK}{{\color{blue} \textbf{CHECK}}}

\newcommand{\bC}{\mathbb{C}}
\newcommand{\bR}{\mathbb{R}}
\newcommand{\bP}{\mathbb{P}}
\newcommand{\bN}{\mathbb{N}}
\newcommand{\bZ}{\mathbb{Z}}
\newcommand{\cA}{\mathcal{A}}
\newcommand{\cB}{\mathcal{B}}
\newcommand{\cC}{\mathcal{C}}
\newcommand{\cD}{\mathcal{D}}
\newcommand{\cF}{\mathcal{F}}
\newcommand{\cG}{\mathcal{G}}
\newcommand{\cH}{\mathcal{H}}
\newcommand{\cI}{\mathcal{I}}
\newcommand{\cJ}{\mathcal{J}}
\newcommand{\cL}{\mathcal{L}}
\newcommand{\cM}{\mathcal{M}}
\newcommand{\cP}{\mathcal{P}}
\newcommand{\fg}{\mathfrak{g}}
\newcommand{\fso}{\mathfrak{so}}
\newcommand{\diff}{\mathrm{diff}}
\newcommand{\Weyl}{\mathrm{Weyl}}
\DeclareMathOperator{\id}{id}
\DeclareMathOperator{\Vol}{Vol}
\DeclareMathOperator{\diag}{diag}
\DeclareMathOperator{\im}{Im}
\DeclareMathOperator{\re}{Re}
\DeclareMathOperator{\Ad}{Ad}
\DeclareMathOperator{\Sp}{Sp}
\DeclareMathOperator{\SO}{SO}
\DeclareMathOperator{\GL}{GL}
\DeclareMathOperator{\Hom}{Hom}
\DeclareMathOperator{\Aut}{Aut}
\newcommand{\eff}{\text{eff}}
\newcommand{\pp}{\text{pp}}
\newcommand{\dder}[2]{\frac{d #1}{d #2}}
\newcommand{\pder}[2]{\frac{\partial #1}{\partial #2}}
\newcommand{\pdder}[2]{\frac{\partial^2 #1}{\partial #2^2}}

\begingroup
    \makeatletter
    \@for\theoremstyle:=definition,remark,plain\do{%
        \expandafter\g@addto@macro\csname th@\theoremstyle\endcsname{%
            \addtolength\thm@preskip\parskip
            }%
        }
\endgroup

\edef\restoreparindent{\parindent=\the\parindent\relax}
\usepackage{parskip}
\restoreparindent

\title{String Theory and Supersymmetry\\Winter 2016 Seminar Notes}
\author{Anton Borissov, Henry Liu}
\date{\today}

\begin{document}

\maketitle

\tableofcontents

\chapter{Introduction to Strings}

We asked ``why fields?'' when we started QFT; now we ask, why strings?
Here are some potentially convincing reasons.
\begin{enumerate}
\item If we allow one more degree of freedom than particles, many
  IR/UV divergences disappear; we require less renormalization. If we
  allow more than one degree of freedom, new divergences arise from
  the increased internal degrees of freedom.
\item Every consistent string theory contains a massless spin-$2$
  state, i.e. a graviton, whose interactions at low energies reduce to
  general relativity.
\item The Standard Model, based on QFT, has $25$ adjustable constants.
  String theory has none, and leads to gauge groups big enough to
  include the Standard Model.
\item Consistent string theories force upon us supersymmetry and
  extra dimensions, which have arisen naturally from several different
  attempts to unify the Standard Model.
\end{enumerate}

Regardless of whether they are convincing, we start in this chapter,
as with any other physical model, by writing down an action.
Specifically, we first write the action for a relativistic string by
generalizing that of a relativistic point particle, and then we
quantize the action. As with QFT there are different ways to quantize.
For exposure and convenience, we use the analogue of canonical
quantization for now, in order to quickly compute the spectrum of a
string.

As usual, we take $\hbar = c = 1$, and use {\bf Einstein summation
  convention}: indices that appear as both superscripts and subscripts
are implicitly summed over.

\section{Review of Relativity}

We work in $\bR^{D-1,1}$ where $D$ is the {\bf number of dimensions}.
Recall that coordinates are written $x^\mu = (x^0, x^1, \ldots, x^D) =
(ct, x^1, \ldots, x^D)$, and the metric is
\[ -ds^2 \coloneqq \eta_{\mu\nu} dx^\mu dx^\nu, \quad \eta_{\mu\nu} = \diag(-1, 1, 1, \ldots, 1). \]
Note that $\eta^\mu{}_\mu = D$. We use the dot product to stand for
the {\bf Lorentz inner product}, e.g. $-ds^2 = dx \cdot dx$.

\begin{definition}
  Define the {\bf proper time} of a system as the time elapsed
  measured by a clock traveling in the same Lorentz frame as the
  system itself. In such a Lorentz frame, $dx^i = 0$ and $dt$ is the
  proper time elapsed, so $-ds^2 = -dt_p^2$; define
  \[ ds \coloneqq \sqrt{ds^2} = dt_p \qquad \text{whenever } ds^2 > 0, \]
  i.e. for timelike intervals. Hence $ds$ is the {\bf proper time
    interval}. The {\bf relativistic momentum} is $p^\mu \coloneqq
  m(dx^\mu/ds)$. Conveniently,
  \[ p^\mu p_\mu = m^2\dder{x^\mu}{s} \dder{x_\mu}{s} = -m^2 \dder{s^2}{s^2} = -m^2. \]
\end{definition}

\begin{definition}
  A {\bf Lorentz transformation} $\Lambda^\mu{}_\nu$ is an element of
  the Lorentz group, the collection of all linear isometries of
  $\bR^{D-1,1}$. We say $a^\mu$ is a {\bf vector} if under Lorentz
  transformations, it changes as $a'^\mu = \Lambda^\mu{}_\nu a^\nu$. A
  {\bf Poincar\'e transformation} is a Lorentz transformation possibly
  followed by a translation.
\end{definition}

\begin{definition}
  The {\bf world line} of a point particle is the path in spacetime
  $\bR^{D-1,1}$ traced out by the particle as it evolves in time.
\end{definition}

The underlying principle of relativity says that physical laws are
independent of Lorentz frame. In other words, any action we write down
that we want to be compatible with relativity must have external
symmetries: it must be invariant under Lorentz transformations. We
call this {\bf Lorentz invariance}. As long as superscripts and
subscripts match up, we do not have to worry about Lorentz invariance.

The {\bf action} for a free relativistic {\bf point particle} is
obtained by writing down the simplest Lorentz invariant action, and
then making sure dimensions work out. If $\gamma$ is the path taken by
the particle, the action is therefore
\[ S_{\pp}[x] \coloneqq -m\int_\gamma ds = -m\int_\gamma d\tau \, \sqrt{-\eta_{\mu\nu} \dder{x^\mu}{\tau} \dder{x^\nu}{\tau}} = -m\int_\gamma d\tau \, \sqrt{-\dot{x}^\mu \cdot \dot{x}_\mu} \]
where a dot denotes a $\tau$-derivative. Because $ds$ is
coordinate-independent, it does not what how we pick the
parametrization $\tau$. Physicists like to call this {\bf
  reparametrization invariance}. This invariance is very important:
without it, we have actually introduced a completely new parameter
$\tau$, thus increasing the number of degrees of freedom from $D-1$ to
$D$.

\begin{exercise}
  By computing $\delta(ds^2)$ in two different ways, show that
  \[ \delta S_{\pp}[x] = m\int_\gamma \delta(dx^\mu) \frac{dx_\mu}{ds} = \int_\gamma d\tau \left(\dder{}{\tau} \delta x^\mu\right) p_\mu = \delta x^\mu p_\mu\bigg|_{\tau_i}^{\tau_f} - \int d\tau \delta x^\mu \dder{p_\mu}{\tau}. \]
  Argue that the first term vanishes if we specify {\bf initial and
    final conditions}. Hence deduce the equation of motion
  $dp_\mu/d\tau = 0$.
\end{exercise}

The action $S_{\pp}$ seems simple in the $\int_\gamma ds$ form, but is
messy when parametrized. Later when we quantize using path integrals,
$S_{\pp}$ is difficult to work with because of the derivatives under
the square root. There is a different, classically-equivalent action
we can work with. Introduce an additional field
$\gamma_{\tau\tau}(\tau)$ (sometimes called an {\bf einbein} in
general relativity), which we can view as a metric on the world line,
and take the action
\[ S'_{\pp} \coloneqq -\frac{1}{2} \int_\gamma d\tau \sqrt{-\gamma_{\tau\tau}} (\gamma^{\tau\tau} \dot{x}^\mu \dot{x}_\mu + m^2) = -\frac{1}{2} \int_\gamma d\tau \, (\eta^{-1} \dot{x}^\mu \dot{x}_\mu - \eta m^2), \quad \eta \coloneqq \sqrt{-\gamma_{\tau\tau}(\tau)}. \]
It seems like we have arbitrarily added an extra degree of freedom,
but in fact $\gamma$ is completely specified by the equation of
motion. The action $S'_{\pp}$ is much better to work with in a path
integral, because it is {\bf quadratic} in $\dot{x}^\mu$.

\begin{exercise}
  Vary $S'_{\pp}$ with respect to $\gamma_{\tau\tau}$ to get the
  equation of motion $\gamma_{\tau\tau} = \dot{x}^\mu
  \dot{x}_\mu/m^2$. Substitute this expression back into $S'_{\pp}$ to
  obtain $S_{\pp}$, and therefore conclude that the two actions are
  classically equivalent.
\end{exercise}

\section{Nambu--Goto and Polyakov Actions}

We graduate to {\bf one-dimensional strings}; in this section we write
down an action for them. There are two kinds of strings: those with
two distinct endpoints, called {\bf open strings}, and those which are
loops, called {\bf closed strings}. Because closed strings are just
open strings with the extra constraint that the two endpoints match,
we focus on open strings.

The action for the relativistic point particle is proportional to the
proper time elapsed on the particle's world line. But the proper time,
when multiplied by $c$, can be viewed as the ``proper length'' of the
world line. The natural generalization, then, is to consider the
surface in space-time traced out by the string as it evolves in time,
called the {\bf world sheet} $M$, and to define an action proportional
to the ``proper area'' of the world sheet. The world sheet $M$ is a
two-dimensional surface, and therefore requires charts modeled on
$\bR^2$.

\begin{definition}
  The {\bf coordinates} we use on $\bR^2$, the parameter space, are
  denoted $(\tau, \sigma)$, and so the {\bf world sheet} $M$ is
  locally a surface given by functions denoted $X^\mu(\tau, \sigma)$
  (capitalized to disambiguate from the coordinates $x^\mu$), called
  {\bf string coordinates}. The lowercase Latin characters $a, b,
  \ldots$ are used to denote {\bf indices} that run over values $\tau,
  \sigma$. In particular, sometimes we write $\sigma^0 = \tau$ and
  $\sigma^1 = \sigma$, i.e. we write the coordinates $(\tau, \sigma)$
  as $(\sigma^0, \sigma^1)$. Two notes:
  \begin{enumerate}
  \item The choice of parametrization $(\tau, \sigma)$ is, again, up
    to us, but usually we take the coordinate $\tau$ to be the proper
    time, and $\sigma$ the position along the string.
  \item For our purposes, $M = X^\mu$, i.e. the single chart $X^\mu$
    describes the entire world sheet for the region of spacetime we
    care about.
  \end{enumerate}
\end{definition}

\begin{exercise}
  Show that the metric $\eta_{\mu\nu}$ on spacetime $\bR^{D-1,1}$
  induces a metric $g$ on the world sheet via pullback along the
  inclusion $\iota\colon M \to \bR^{D-1,1}$. Compute $g$ and the area
  element:
  \[ g_{ab} = \di_a X^\mu \di_b X_\mu, \quad dA = d\tau \, d\sigma \, \sqrt{-\det g}. \]
\end{exercise}

A relativistic particle has a parameter we call mass. It turns out
mass is not the appropriate physical interpretation of the
corresponding parameter for strings. Instead, we interpret it as a
{\bf tension}, and denote it $T_0$.

\begin{definition}
  The {\bf Nambu--Goto action} for a relativistic string is given by
  \[ S_{\text{NG}}[X] \coloneqq -T_0 \int_M dA = -T_0 \int_M d\tau \, d\sigma \, \sqrt{-\det g}. \]
  Again, note that it satisfies {\bf reparametrization invariance},
  literally by construction.
\end{definition}

But again, we have a square root and derivatives inside it, and now we
know how to get rid of it: introduce an independent world sheet metric
$\gamma_{ab}(\tau, \sigma)$. This time the metric is on a surface, so
we need to specify the signature. We take Lorentzian signature $(-,
+)$.

\begin{definition}
  The {\bf Polyakov action} for a relativistic string is given by
  \[ S_{\text{P}}[X, \gamma] \coloneqq -\frac{T_0}{2} \int_M d\tau \, d\sigma \, \sqrt{-\gamma} \, \gamma^{ab} \di_a X^\mu \di_b X_\mu, \]
  where $\gamma$ without indices stands for $\det(\gamma_{ab})$. From
  now on, we always refer to $\gamma_{ab}$ as the {\bf metric}, and
  $g_{ab}$ as the {\bf induced metric}. Indices are raised/lowered
  using the metric $\gamma_{ab}$, not the induced metric $g_{ab}$. (In
  fact, from now on we basically forget about $g_{ab}$; we use it only
  to introduce the Nambu--Goto action, and the following exercise.)
\end{definition}

\begin{exercise}
  Show that $\delta \sqrt{-\gamma} = (1/2)\sqrt{-\gamma} \,
  \gamma^{ab} \delta \gamma_{ab}$, and therefore that
  \[ \delta_\gamma S_{\text{P}}[X, \gamma] = -\frac{T_0}{2} \int_M d\tau \, d\sigma \, \sqrt{-\gamma} \, \delta \gamma^{ab} \left(g_{ab} - \frac{1}{2} \gamma_{ab} \gamma^{cd} g_{cd}\right). \]
  Rearrange the obtained equation of motion and conclude that
  $g_{ab}\sqrt{-g} = \gamma_{ab}\sqrt{-\gamma}$. Hence replace
  $\gamma$ in $S_{\text{P}}[X, \gamma]$ with $g$, and obtain that
  $S_{\text{P}}[X, \gamma] = S_{\text{NG}}[X]$.
\end{exercise}

\begin{definition}
  As in general relativity, define the {\bf stress-energy tensor}
  \[ T^{ab}(\tau, \sigma) \coloneqq -\frac{4\pi}{\sqrt{-\gamma}} \delta_{\gamma} S_{\text{P}}[X, \gamma] = -2\pi T_0 \left(\di^a X^\mu \di^b X_\mu - \frac{1}{2} \gamma^{ab} \di_c X^\mu \di^c X_\mu\right), \]
  so that the {\bf equation of motion} arising from varying $\gamma$
  says $T_{ab} = 0$.
\end{definition}

\begin{exercise} (Important!)
  Now vary $S_{\text{P}}[X, \gamma]$ with respect to $X^\mu$ to obtain
  \begin{align*}
    \delta_X S_{\text{P}}[X, \gamma]
    &= -T_0 \int_M d\tau \, d\sigma \, \sqrt{-\gamma} \, \gamma^{ab} \left(\di_a(\delta X^\mu \di_b X_\mu) - \di_a \di_b X_\mu \delta X^\mu\right) \\
    &= -T_0 \int_0^{\ell} d\sigma \, \sqrt{-\gamma} \left[\delta X^\mu \di^\tau X_\mu\right]_{\tau=\tau_i}^{\tau=\tau_f} - T_0 \int_{\tau_i}^{\tau_f} d\tau \, \sqrt{-\gamma} \left[\delta X^\mu \di^\sigma X_\mu\right]_{\sigma=0}^{\sigma=\ell} \\
    &\qquad + T_0 \int_M d\tau \, d\sigma \, \sqrt{-\gamma} \, \delta X^\mu \nabla^2 X_\mu.
  \end{align*}
\end{exercise}

A careful inspection of the terms in the variation $\delta_X
S_{\text{P}}[X, \gamma]$ yield interesting insights. For this
variation to vanish, each of the terms must vanish independently,
since they control different aspects of the string's behavior.
\begin{enumerate}
\item The last term is determined by the motion of the string in the
  domain $(0, \ell) \times (\tau_i, \tau_f)$, and therefore $\delta
  X^\mu$ is not constrained by any boundary conditions there. Hence we
  have the {\bf equation of motion} $\sqrt{-\gamma} \, \nabla^2 X_\mu
  = 0$.
\item The first term is determined by the configuration of the string
  at times $\tau_i$ and $\tau_f$. If we specify these configurations
  as {\bf initial and final conditions}, then $\delta X^\mu$ is zero
  for the first term, so the term vanishes.
\item The second term is determined by the configuration of the
  endpoints of the string when $\tau \in (\tau_i, \tau_f)$. It does
  not vanish automatically, and we have to impose {\bf boundary
    conditions} in order for it to do so.
\end{enumerate}

\begin{definition}
  Let $\sigma_*$ denote the $\sigma$-coordinate of an endpoint, i.e.
  either $\sigma_* = 0$ or $\sigma_* = \ell$.
  \begin{itemize}
  \item The {\bf free (Neumann) boundary condition} is $\di^\sigma
    X^\mu(\tau, \sigma_*) = 0$.
  \item The {\bf Dirichlet boundary condition} is $\delta X^\mu(\tau,
    \sigma_*) = 0$.
  \end{itemize}
  Alternatively, if the string is {\bf closed}, i.e. we have the {\bf
    periodicity} conditions
  \[ X^\mu(\tau, 0) = X^\mu(\tau, \ell), \quad \di^a X^\mu(\tau, 0) = \di^a X^\mu(\tau, \ell), \quad \gamma_{ab}(\tau, 0) = \gamma_{ab}(\tau, \ell), \]
  no additional boundary conditions are necessary.
\end{definition}

For a long time, string theorists did not seriously consider the
Dirichlet boundary condition. Why should the endpoints of an open
string be fixed, and if they were, where would they be fixed onto? In
particular, this fixing of endpoints would violate momentum
conservation. Then Polchinski, in the 1990s, suggested that the
endpoints are attached to {\bf D-branes}, which should themselves be
thought of as dynamical objects alongside strings. Conceptually, then,
\begin{enumerate}
\item a D$0$-brane is a particle, a D$1$-brane is a string, and so on,
  and they interact non-trivially;
\item the Dirichlet boundary condition says that a given D$1$-brane
  has fixed endpoints on a higher D$p$-brane;
\item any momentum lost by the D$1$-brane is absorbed by the
  D$p$-brane; and
\item the Neumann boundary condition is just saying there is a
  D-dimensional D-brane permeating all of space-time, i.e. the string
  endpoints are not fixed at all.
\end{enumerate}
We return to this D-brane perspective much later on. It is hard enough
to quantize strings without more dynamical objects floating around.

\section{Gauge Fixing}

There is another reason the Polyakov action is preferable over the
Nambu--Goto action: it has more symmetries, and these symmetries make
it easier to gauge fix (using Faddeev--Popov or otherwise) when we try
to quantize. The Polyakov action is invariant under the following
symmetries:
\begin{enumerate}
\item $D$-dimensional {\bf Poincar\'e transformations}:
  \[ X^\mu(\tau, \sigma) \mapsto \Lambda^\mu{}_\nu X^\nu(\tau, \sigma) + a^\mu, \quad \gamma_{ab}(\tau, \sigma) \mapsto \gamma_{ab}(\tau, \sigma); \]
\item {\bf Reparametrization} (i.e. diffeomorphisms): for new
  coordinates $\tilde{\sigma}^a(\tau, \sigma)$,
  \[ X^\mu(\tau, \sigma) \mapsto X^\mu(\tau,\sigma), \quad \gamma_{ab}(\tau, \sigma) \mapsto \pder{\sigma^c}{\tilde{\sigma}^a} \pder{\sigma^d}{\tilde{\sigma}^b} \gamma_{cd}(\tau, \sigma); \]
\item $2$-dimensional {\bf Weyl transformations}: for arbitrary
  $\omega(\tau, \sigma)$,
  \[ X^\mu(\tau, \sigma) \mapsto X^\mu(\tau,\sigma), \quad \gamma_{ab}(\tau, \sigma) \mapsto \exp(2\omega(\tau, \sigma)) \gamma_{ab}(\tau, \sigma). \]
\end{enumerate}
The Nambu--Goto action is not invariant under Weyl transformations.

\begin{exercise}
  Verify all these statements. (This should be quite straightforward.)
\end{exercise}

\begin{definition}
  Let $\diff$ denote the group of diffeomorphisms acting on $\Sigma$,
  and $\Weyl$ the group of Weyl transformations acting on $\Sigma$.
\end{definition}

Before we continue, let's make sure our Lagrangian is as general as
possible, to reduce future work. If $\cL$ has more terms, they need to
satisfy all the above symmetries. In particular, Weyl invariance is
very odd: it prevents us from adding terms such as
\[ \int_M d\tau \, d\sigma \, \sqrt{-\gamma} \, V(X), \quad \mu \int_M d\tau \, d\sigma \, \sqrt{-\gamma}. \]

\begin{exercise}
  Convince yourself that $\cL$ must contain one more $\gamma^{ab}$
  than $\gamma_{ab}$ in order to satisfy Weyl invariance and
  counteract the change in $\sqrt{-\gamma}$. Since such a
  $\gamma^{ab}$ can only pair up indices with derivatives, we need a
  second-order Lorentz-invariant term that is coordinate-independent.
  Convince yourself that other than $\di_a X^\mu \di_b X_\mu$, this
  term can only involve $\gamma^{ab}$ and $\gamma_{ab}$, and that in
  fact it must be the {\bf scalar curvature} $R$. Show that under a
  Weyl transformation,
  \[ \sqrt{-\gamma} \, R \mapsto \sqrt{-\gamma} (R - 2\nabla^2 \omega). \]
  Hence argue that we need another term integrated over $\di M$ to
  counteract $\nabla^2(\sqrt{-\gamma} \, \omega)$. Putting everything
  together, conclude that
  \[ \chi \coloneqq \frac{1}{4\pi} \int_M d\tau \, d\sigma \, \sqrt{-\gamma} \, R + \frac{1}{2\pi} \int_{\di M} ds \, k \]
  is Weyl invariant. Here $ds$ is proper time along $\di M$ using the
  metric $\gamma_{ab}$, and $k \coloneqq \pm t^a n_b \nabla_a t^b$ is
  the {\bf geodesic curvature} of the boundary, where $t^a$ is a unit
  vector tangent to the boundary, and $n_b$ an outward-pointing unit
  vector, and we choose $\pm$ depending on whether the boundary is
  timelike or spacelike.
\end{exercise}

Let's proceed with gauge fixing: we need to use up the internal
degrees of freedom in $\diff \times \Weyl$. The transformation of the
scalar curvature computed in the exercise above says we can use Weyl
invariance to locally set the scalar curvature to zero, by solving
$2\nabla^2\omega = R$ and then applying the Weyl transformation
$\exp(2\omega)$. But we are in two dimensions, where the symmetries of
the Riemann curvature tensor determine it from $R$:
\[ R_{abcd} = R_{cdab}, \; R_{abcd} = -R_{bacd} = -R_{abdc} \implies R_{abcd} = \frac{1}{2}(\gamma_{ac}\gamma_{bd} - \gamma_{ad}\gamma_{bc}) R. \]
Hence not only can we locally set $R = 0$, we can locally set
$\gamma_{ab} = \eta_{ab}$, the flat Minkowski metric.

\begin{definition}
  If we consider only reparametrization and not Weyl transformations,
  the metric $\gamma_{ab}$ can always be brought to the form
  $\exp(2\omega)\eta_{ab}$. Forcing the metric to be of that form is
  known as {\bf conformal gauge}. Performing the additional Weyl
  transformation to obtain $\gamma_{ab} = \eta_{ab}$ is known as {\bf
    unit gauge}. In general, the form of the metric we choose to put
  $\gamma_{ab}$ in is called the {\bf fiducial metric}.
\end{definition}

How many internal degrees of freedom have we used up if we put the
metric $\gamma_{ab}$ in unit gauge? Well, $\diff$ has two degrees of
freedom, one for each coordinate, and $\Weyl$ has one, for the scale
of the metric. But the metric itself has three independent components,
being symmetric. Hence we expect to be done: we have canonically
chosen a representative of each gauge equivalence class.

But, perhaps unexpectedly, there is more gauge freedom: there are
non-trivial transformations in $\diff \times \Weyl$ that preserve unit
gauge! The key to finding these transformations is to realize that
$\Sigma$ is actually a {\bf Riemann surface}: let $z \coloneqq \tau +
i\sigma$, so that $ds^2 = dz d\bar{z}$. Now if $f(z)$ is a holomorphic
change of coordinates, then
\[ z \mapsto f(z), \quad ds^2 \mapsto |\di_z f|^{-2} dz d\bar{z}, \]
so now applying the Weyl transformation $\exp(2 \ln |\di_z f|)$
recovers $ds^2$. Clearly the composition of the two transformations is
non-trivial.

What went wrong? Well, just because dimensions match up does not mean
we have spanned the whole space of gauge transformations! The
holomorphic diffeomorphisms above actually have {\bf measure zero} in
$\diff$. When we stop working locally and work globally instead, these
extra bits of freedom are canceled by boundary conditions.

\section{Quantization via Path Integral}

Recall from QFT that we have a giant machine for quantizing classical
theories: the path integral. Before we begin plugging the Polyakov
action into the machine, however, we need to make a modification. From
now on, the world sheet is equipped with a {\bf Euclidean metric}
$g_{ab}$, instead of a Lorentzian one $\gamma_{ab}$. This is so that
the path integral over metrics is better defined. The transition from
Euclidean to Minkowski is, formally, done via {\bf Wick rotation}:
$x^0 \mapsto ix^0$ and similarly for the metric. The {\bf Euclidean
  path integral}, and the Euclidean action (with the additional terms
on top of the Wick-rotated Polyakov action), is therefore
\[ Z \coloneqq \frac{1}{\Vol} \int \cD g \, \cD X \, \exp(-S_{\text{P}}[X, g]), \]
\[ S_{\text{P}}[X, g] = \frac{T_0}{2} \int_M d\tau \, d\sigma \, \sqrt{g} \, g^{ab} \di_a X^\mu \di_b X_\mu + \lambda\left(\frac{1}{4\pi} \int_M d\tau \, d\sigma \, \sqrt{g} \, R + \frac{1}{2\pi}\int_{\di M} ds \, k\right) \]
where $\Vol$ is the volume of the gauge action on the {\bf
  configuration space} consisting of all possible $X^\mu$ and
$g$). More explicitly, we can imagine partitioning configuration
space into gauge orbits; we actually want to integrate on a path
through these gauge orbits. But now recall from QFT that we have
another giant machine for doing so: the Faddeev-Popov method.

\subsection{The Faddeev-Popov Method}

Let's first recall that the idea behind Faddeev-Popov is very natural:
we want to do a change of coordinates in configuration space so that
instead of integrating over a mish-mash of $g$ and $X$, we integrate
such that one variable goes along gauge orbits, and the other goes
along the gauge-fixed path. Although this sounds technical, we perform
procedures like this quite often without realizing it! For example,
consider the calculation
\[ \iint dx \, dy \, e^{-x^2 - y^2} = \int d\theta \int dr \, r e^{-r^2} = 2\pi \int dr \, r e^{-r^2} = \pi. \]
What is really happening here is that we recognized the $U(1)$
symmetry of the original integrand, and changed variables in order to
factor out that symmetry. Instead of integrating over $(x, y)$, we
integrated over $(r, \theta)$, with $\theta$ parametrizing the gauge
orbits. Furthermore, we picked out the $y = 0$ representative of
each gauge orbit for the remaining integral.

Armed with this motivation, we can proceed. Let $\hat{g}_{ab}$ be the
fiducial metric; it represents our choice of gauge fixing, just like
the choice $y = 0$. Let $\zeta$ be shorthand for a combined coordinate
and Weyl transformation:
\[ \zeta\colon g \mapsto g^\zeta \coloneqq \exp(2\omega(\tau,\sigma)) \pder{\sigma^c}{\sigma'^a} \pder{\sigma^d}{\sigma'^b} g_{cd}(\tau,\sigma). \]

\begin{definition}
  Let $\cD \zeta$ be a gauge invariant measure on $\diff \times
  \Weyl$. (Whether such a measure exists is very relevant for us, but
  we disregard it for now.) Define the {\bf Faddeev-Popov determinant}
  \[ \Delta_{\text{FP}}^{-1}(g) \coloneqq \int \cD \zeta \, \delta(\hat{g}^\zeta - g). \]
  Here the $\delta$ is really the {\bf Dirac functional}, i.e.
  $\hat{g}^\zeta$ and $g$ must agree at every point $(\tau, \sigma)$.
\end{definition}

\begin{exercise}
  Show that $\Delta_{\text{FP}}(g)$ is gauge-invariant by computing
  that $\Delta_{\text{FP}}(g^\zeta)^{-1} =
  \Delta_{\text{FP}}(g)^{-1}$.
\end{exercise}

Now it is time to do the calculation to factor out the integral over
the gauge orbits. The first step is to add a $1$ to the integral:
\[ Z = \int \frac{\cD g \, \cD X}{\Vol} \exp(-S_{\text{P}}[X, g]) = \int \frac{\cD g \, \cD X \, \cD \zeta}{\Vol} \Delta_{\text{FP}}(g) \delta(\hat{g}^\zeta - g) \exp(-S_{\text{P}}[X, g]). \]
The second step is to do the integral over $g$, which, due to the
$\delta(\hat{g}^\zeta - g)$, amounts to substituting $g$ with
$\hat{g}^\zeta$:
\[ Z = \int \frac{\cD X \, \cD \zeta}{\Vol} \Delta_{\text{FP}}(\hat{g}^\zeta)\exp(-S_{\text{P}}[X, \hat{g}^\zeta]). \]
Finally, since both $\Delta_{\text{FP}}$ and $S_{\text{P}}$ are
gauge-invariant, we can substitute $\hat{g}$ for $\hat{g}^\zeta$. Then
nothing in the integrand depends on $\zeta$ anymore, so it factors out
and cancels the volume normalization:
\[ Z = \int \frac{\cD \zeta}{\Vol} \int \cD X \, \Delta_{\text{FP}}(\hat{g}) \exp(-S_{\text{P}}[X, \hat{g}]) = \int \cD X \, \Delta_{\text{FP}}(\hat{g}) \exp(-S_{\text{P}}[X, \hat{g}]). \]

\begin{exercise}
  Evaluate $\iint dx \, dy \, e^{-x^2 - y^2}$ by applying the
  Faddeev-Popov method to its $U(1)$ symmetry and the gauge-fixing
  condition $y = 0$. Conclude that the Faddeev-Popov method is
  completely rigorous in finite dimensions, and that
  $\Delta_{\text{FP}}$ is actually a Jacobian (hence the name
  Faddeev-Popov determinant).
\end{exercise}

It remains to compute the Faddeev-Popov determinant
$\Delta_{\text{FP}}$. To do so, we make the simplifying assumption
that $\diff \times \Weyl$ actually acts freely on metrics $g$, i.e.
for each $g$, there is exactly one $\zeta$ such that
$\delta(\hat{g}^\zeta - g) = 0$. Obviously this assumption is false:
we showed earlier that the action has fixed points. But it is true
locally, so we deal with the global issues later.



\end{document}
